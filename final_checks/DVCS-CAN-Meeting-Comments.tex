\section*{\centering{Answers to the final questions sent on October 7th, 2016}}


"Dear Mohammad,

After consultation of Raphael (who contacted you), the CAN review committee 
would like to request that the study of varying cut in the missing energy 
eHe4gX be repeated with a proper, bin dependent, pi0 subtraction. We understand 
that this may take several days, but feel it is a necessary step to raise any 
doubt about the stability of the result. Optionally, if time allows, add a 
fourth cut which would only get rid of the tails: [-0.45, 0.50].

I also discussed with Raphael the necessity to explain in the section on 
Systematic uncertainties how the corresponding numbers in the numerical tables 
are generated. So far, the discussion is exclusively on the fitted ALU(90°), 
while the numbers in the Tables are bin-dependent.

Thank you and best regards,

Michel."

\begin{enumerate}
   \item \textcolor{blue}{Regarding the stability of the coherent $A_{LU}$ with 
      respect to the missing energy cut. I first remind you about the previous 
   study from the first round comments and the results we got. Then, I will 
present the new results after the CAN review committee suggestions. }
   
\begin{enumerate}
   \item The first round comment, where the background subtraction have been 
      performed based on the 4D background subtractions without binning in the 
      missing energy distribution: \\
      " 40) Fig.  4.2: I'm bothered by the wide distribution in Emiss (top 
      middle), and don't understand why you use such wide cuts.  Surely, 
      anything above Emiss = 0.4 must have additional particles in it, and even 
      if not, why not cut this tail?  Note that the same tail is absent for the 
      p (Fig. 4.6) where you use indeed a tighter cut, although you don't even 
      account for the proton's initial momentum.\\
   \textcolor{blue}{ We applied 3$\sigma$ based on the comparison between data 
      and simulation, see figure 4.4 in the note. In the following, figure 
      \ref{fig:coherent_ME_bins}, we performed bins in the missing energy 
      distribution and we watch the reconstructed beam-spin asymmetries. As a 
      conclusion, we see that the reconstructed asymmetries are compatible 
   within the given error bars.}" \\

\begin{figure}[tbp]
   \centering
   \includegraphics[height=7.0cm]{fig/coh_ME_bins.png}
   \includegraphics[height=7.0cm]{fig/BSA_coherent_ME.png}
   \includegraphics[height=7.0cm]{fig/coh_ME_alpha.png}
   \caption{Coherent missing energy distributions (top), the reconstructed 
   beam-spin asymmetries as a function of $\phi$ (middle), and the extracted 
asymmetry at $\phi = 90 ^{\circ}$ from fitting the asymmetries as a function of 
the missing energy in each bin. }
   \label{fig:coherent_ME_bins}
    \end{figure}


 \item Following the CAN committee suggestions, we applied the requested cuts 
    to remove the tails and we extracted R for the different bins in the 
    missing energy. The results are presented in figure 
    \ref{fig:coherent_ME_bins_2}. Within the given error bars, the two sets of 
    results are compatible.   
    
    
    \begin{figure}[tbp]
   \includegraphics[height=6.0cm]{fig/new_ME_Coh_bins.png}
   \includegraphics[height=6.0cm]{fig/e4Hegamma_e4Hepi0_Phi_ME.png}
   \hspace{-3.0cm} \includegraphics[height=6.5cm]{fig/new_BSA_coherent_ME.png}
   \hspace{-1.0cm} \includegraphics[height=6.5cm]{fig/new_coh_ME_alpha.png}
   \caption{from top to bottom and from left to right, coherent missing energy 
      binning, the corresponding background acceptance ratio from simulation, 
      the reconstructed beam-spin asymmetries as a function of $\phi$ (middle), 
   and the extracted asymmetry at $\phi = 90 ^{\circ}$ from fitting the 
asymmetries as a function of the mean missing energy in each bin.  }
   \label{fig:coherent_ME_bins_2}
    \end{figure}
\end{enumerate}

\item \textcolor{blue}{Regarding the listed bin by bin systematic errors, they 
   were extracted from the variation of the reconstructed asymmetries such as 
in figures 4.23 and 4.29 in the analysis note. The text is modified to include 
the extraction details, see the end of section 4.7.}

\end{enumerate}

