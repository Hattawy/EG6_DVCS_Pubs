\documentclass[a4paper,11pt,twoside]{article}

\date{\today}
\usepackage{amsmath,amssymb}             % AMS Math
% \usepackage[french]{babel}
\usepackage[latin1]{inputenc}
\usepackage[OT1]{fontenc}
\usepackage[left=2.7cm,right=1.7cm,top=1.6cm,bottom=1.6cm,includefoot,includehead,headheight=13.6pt]{geometry}
\usepackage{setspace}
\usepackage{epigraph}
\usepackage{lineno}


%\usepackage{arev}
%\usepackage[bitstream-charter]{mathdesign}
%\usepackage[urw-garamond]{mathdesign}
%\usepackage[sfmath]{kpfonts} %% sfmath option only to make math in sans serif. Probablye only for use when base font is sans serif.
%\renewcommand*\familydefault{\sfdefault} %% Only if the base font of the document is to be sans serif
\usepackage[sc]{mathpazo}
\linespread{1.05}   
\usepackage[T1]{fontenc}



% Table of contents for each chapter

\usepackage[nottoc, notlof, notlot]{tocbibind}
\usepackage{minitoc}
\setcounter{minitocdepth}{2}
\mtcindent=15pt
% Use \minitoc where to put a table of contents

\usepackage{aecompl}

% Glossary / list of abbreviations

\usepackage[intoc]{nomencl}
\renewcommand{\nomname}{List of Abbreviations}

\makenomenclature

% My pdf code

\usepackage{graphicx,type1cm,eso-pic,color}
\usepackage{lscape}

  \usepackage[pagebackref,hyperindex=true]{hyperref}

%\geometry{letterpaper}
%\graphicspath{{.}{images/}}

% nicer backref links
\renewcommand*{\backref}[1]{}
\renewcommand*{\backrefalt}[4]{%
\ifcase #1 %
(Not cited.)%
\or
(Cited on page~#2.)%
\else
(Cited on pages~#2.)%
\fi}
\renewcommand*{\backrefsep}{, }
\renewcommand*{\backreftwosep}{ and~}
\renewcommand*{\backreflastsep}{ and~}

% Links in pdf
\usepackage{color}
\definecolor{linkcol}{rgb}{0,0,0.4} 
\definecolor{citecol}{rgb}{0.5,0,0} 

% Change this to change the informations included in the pdf file

% See hyperref documentation for information on those parameters

\hypersetup
{
bookmarksopen=true,
pdftitle="",
pdfauthor="", pdfsubject="", %subject of the document
%pdftoolbar=false, % toolbar hidden
pdfmenubar=true, %menubar shown
pdfhighlight=/O, %effect of clicking on a link
colorlinks=true, %couleurs sur les liens hypertextes
pdfpagemode=None, %aucun mode de page
pdfpagelayout=SinglePage, %ouverture en simple page
pdffitwindow=true, %pages ouvertes entierement dans toute la fenetre
linkcolor=linkcol, %couleur des liens hypertextes internes
citecolor=citecol, %couleur des liens pour les citations
urlcolor=linkcol %couleur des liens pour les url
}

% definitions.
% -------------------

\setcounter{secnumdepth}{3}
\setcounter{tocdepth}{2}

% Some useful commands and shortcut for maths:  partial derivative and stuff
\newcommand{\xbp}{$x_{Bj}$}
\newcommand{\xb}{$x_{Bj}~$}
\newcommand{\ptp}{$p_\perp^2$}
\newcommand{\pt}{$p_\perp^2~$}
\newcommand{\dptp}{$\Delta \langle p_\perp^2 \rangle$}
\newcommand{\dpt}{$\Delta \langle p_\perp^2 \rangle ~$}

\brokenpenalty10000\relax

\newcommand{\pd}[2]{\frac{\partial #1}{\partial #2}}
\def\abs{\operatorname{abs}}
\def\argmax{\operatornamewithlimits{arg\,max}}
\def\argmin{\operatornamewithlimits{arg\,min}}
\def\diag{\operatorname{Diag}}
\newcommand{\eqRef}[1]{(\ref{#1})}

\usepackage{rotating}                    % Sideways of figures & tables
%\usepackage{bibunits}
%\usepackage[sectionbib]{chapterbib}          % Cross-reference package (Natural BiB)
%\usepackage{natbib}                  % Put References at the end of each chapter
                                         % Do not put 'sectionbib' option here.
                                         % Sectionbib option in 'natbib' will do.
\usepackage{fancyhdr}                    % Fancy Header and Footer

% \usepackage{txfonts}                     % Public Times New Roman text & math font
  
%%% Fancy Header %%%%%%%%%%%%%%%%%%%%%%%%%%%%%%%%%%%%%%%%%%%%%%%%%%%%%%%%%%%%%%%%%%
% Fancy Header Style Options

\pagestyle{fancy}                       % Sets fancy header and footer
\fancyfoot{}                            % Delete current footer settings

%\renewcommand{\chaptermark}[1]{         % Lower Case Chapter marker style
%  \markboth{\chaptername\ \thechapter.\ #1}}{}} %

%\renewcommand{\sectionmark}[1]{         % Lower case Section marker style
%  \markright{\thesection.\ #1}}         %

\fancyhead[LE,RO]{\bfseries\thepage}    % Page number (boldface) in left on even
% pages and right on odd pages
\fancyhead[RE]{\bfseries\nouppercase{\leftmark}}      % Chapter in the right on even pages
\fancyhead[LO]{\bfseries\nouppercase{\rightmark}}     % Section in the left on odd pages

\let\headruleORIG\headrule
\renewcommand{\headrule}{\color{black} \headruleORIG}
\renewcommand{\headrulewidth}{1.0pt}
\usepackage{colortbl}
\arrayrulecolor{black}

\fancypagestyle{plain}{
  \fancyhead{}
  \fancyfoot{}
  \renewcommand{\headrulewidth}{0pt}
}

%\usepackage{algorithm}
%\usepackage[noend]{algorithmic}

%%% Clear Header %%%%%%%%%%%%%%%%%%%%%%%%%%%%%%%%%%%%%%%%%%%%%%%%%%%%%%%%%%%%%%%%%%
% Clear Header Style on the Last Empty Odd pages
\makeatletter

\def\cleardoublepage{\clearpage\if@twoside \ifodd\c@page\else%
  \hbox{}%
  \thispagestyle{empty}%              % Empty header styles
  \newpage%
  \if@twocolumn\hbox{}\newpage\fi\fi\fi}

\makeatother
 
%%%%%%%%%%%%%%%%%%%%%%%%%%%%%%%%%%%%%%%%%%%%%%%%%%%%%%%%%%%%%%%%%%%%%%%%%%%%%%% 
% Prints your review date and 'Draft Version' (From Josullvn, CS, CMU)
\newcommand{\reviewtimetoday}[2]{\special{!userdict begin
    /bop-hook{gsave 20 710 translate 45 rotate 0.8 setgray
      /Times-Roman findfont 12 scalefont setfont 0 0   moveto (#1) show
      0 -12 moveto (#2) show grestore}def end}}
% You can turn on or off this option.
% \reviewtimetoday{\today}{Draft Version}
%%%%%%%%%%%%%%%%%%%%%%%%%%%%%%%%%%%%%%%%%%%%%%%%%%%%%%%%%%%%%%%%%%%%%%%%%%%%%%% 

\newenvironment{maxime}[1]
{
\vspace*{0cm}
\hfill
\begin{minipage}{0.5\textwidth}%
%\rule[0.5ex]{\textwidth}{0.1mm}\\%
\hrulefill $\:$ {\bf #1}\\
%\vspace*{-0.25cm}
\it 
}%
{%

\hrulefill
\vspace*{0.5cm}%
\end{minipage}
}

\let\minitocORIG\minitoc
\renewcommand{\minitoc}{\minitocORIG \vspace{1.5em}}

\usepackage{multirow}
%\usepackage{slashbox}

\newenvironment{bulletList}%
{ \begin{list}%
	{$\bullet$}%
	{\setlength{\labelwidth}{25pt}%
	 \setlength{\leftmargin}{30pt}%
	 \setlength{\itemsep}{\parsep}}}%
{ \end{list} }

\newtheorem{definition}{D�finition}
\renewcommand{\epsilon}{\varepsilon}

% centered page environment

\newenvironment{vcenterpage}
{\newpage\vspace*{\fill}\thispagestyle{empty}\renewcommand{\headrulewidth}{0pt}}
{\vspace*{\fill}}


\begin{document}

\section*{Reviewers' comments (Oct. 11th, 2017)}

Reviewer \#1: Dear Author, please find here a list of comments from my side.

Summary: the article describes the layout, operation and performances of a 
Radial Time Projection Chamber with the electron amplification stage provided 
by three layers of GEM. The main novelty is the application of the electron 
multiplier for the amplification stage in a TPC on a semi-cylindrical shaped 
plane. The detector, originally built for the BoNuS experiment, is in fact 
formed by two semi-cylindrically shaped halves connected together, with the 
axis coincident with the beam line.\\
 
The detector was implemented to measure DVCS of electrons on different light 
nuclei and the performances were evaluated on electron-$^4$He elastic 
scattering runs.\\

Chapter 1 describes the CLAS@JLab detector, inside which the RTPC was installed 
and took data. Chapter 2 focuses on the RTPC mechanical and electrical layout.  
From inner to outer, there are: the $^4$He gas target, a $^4$He gas gap, the 
actual time projection chamber. The time projection chamber is a 200~mm long 
detector composed by different gas gaps limited by concentric electrodes around 
the axis: cathode (R= 30~mm), G1(R= 60~mm), G2(R= 63~mm), G3(R= 66~mm), readout 
plane (R= 69~mm). Everything fits inside a 5T solenoidal magnetic field. It 
provides both time and charge measurements. Chapter 3 is a short description of 
the electronics and readout, mainly inherited from ALICE and BoNuS. The 
improvement stands in the higher readout rate capability. Chapter 4 talks about 
calibration of drift velocity, drift paths and energy loss measurements by the 
usage of simulations with MAGBOLTZ and tuning the results on elastic scattering 
data. Chapter 5 concentrates on track finding/fitting procedure and noise 
evaluation and subtraction. Chapter 6 reports on resolution and efficiency from 
the elastic scattering reconstruction of 1 GeV electrons on 4He and Chapter 7 
concludes the paper.\\

The article main points of interest are the followings:

a) The usage of GEMs as avalanche creators inside the time projection chamber.  
As far as I know other TPC with the wires substituted by GEM layers have been 
(or are being) realized, like ALICE TPC upgrade or the FOPI one (proposed for 
PANDA), but in those cases the GEM are planar and on the end plane of the TPC, 
not around the beamline.

b) The shaping of the GEM foils around the target. However, since now a fully 
cylindrical GEM foil has been realized (for the KLOE-2 detector and for the 
next BESIII inner tracker) I would not call these "cylindrical" GEMs, but 
"semi-cylindrical" GEMs to point out the difference (which reflects in the 
construction procedure). Maybe these other experiments using GEMs should be 
mentioned due to the similarities, thought they are successive in time to this 
RTPC construction.\\ 

Concerning the work description, it is mainly exhaustive, but in some points 
there are open questions that I would like to submit to the authors in the 
following.

\newpage
\section*{Chapter by chapter comments}

\begin{enumerate}

\subsection*{Chapter 2}
\item As already said I would not speak of cylindrical GEMs but 
   semi-cylindrical. Moreover, please add the reference to Sauli's paper [11] 
   when speaking the first time about GEM in the text.\\
\textcolor{blue}{cylindrical GEMs-> semi-cylindrical. The reference has been added.} 

\item In my opinion, this chapter should highlight better which are the 
   improvements w.r.t. BoNuS detector to underline the originality of this 
   work. For example, better acceptance? Higher gain? Higher electric 
   stability? Lower material budget? Please quote some old and new values, 
   maybe in a table, to make them evident.\\
\textcolor{blue}{The main difference is in the design of the structure and in the
firmware of the electronics, this has been made clearer.} 

\item Relevant length and thicknesses of the various elements are reported in 
   the text and in the figure, can you please add also the z position of the 
   target w.r.t the detector? To give the complete information, I would also 
   add the value of the induction gap electric field, which is the only missing 
   field value and the value of the reached amplification gain with the three 
   GEM.\\
\textcolor{blue}{ The text has been modified to clarify the z position of the 
target and the RTPC with respect to CLAS sectrometer.\\
Regarding the electric field it is unclear what you call the induction gap. We
provide these fields for the drift region and the inter-GEM regions which we 
find most relevant. If extra information is needed, please let us know.\\
The gain of such triple GEM configuration has been studied by 
CERN's Gas Detector Development Group (Fabio Sauli, Progress with the Gas 
Electron Multiplier, 2nd Workshop on Advanced Transition Radiation Detectors 
for Accelerator and Space Applications (Bari, Sept. 4-7, 2003), Nucl. Instr. 
and Meth. A522(2004)93). We evaluated only the overall gain in our chamber, 
the lack of calibration of some electronics elements makes it impossible for 
us to extract a gain specific to the GEM system at this point.}

\item When speaking of low spark rate the reference [12] is quoted, which is 
   the PDG. Maybe it is better to quote more specific studies like the ones by 
   Bachmann et al.\\
\textcolor{blue}{Thanks. It has been replaced.} 

\item line 119: is the Kapton of the GEM foil really 300 mum thick? It is 
   usually very thin, like 50 mum, in order to have low material budget and 
   very high fields.\\
\textcolor{blue}{Sorry typo. It is 50 $mu$m} 

\subsection*{ Chapter 3}
\item The description of the electronics looks to me a little not well 
   organized, like a collection of info from previous descriptions from ALICE 
   and BoNuS. It is a series of details, some of which not really relevant for 
   the good understanding of the work described in the paper. So, I would put 
   only some general info (the number of channels/pad, the number of time 
   samples and their width for each channel and the number of  ADC samples and 
   their width for each channel) with the reference where to find more material 
   and just write explicitly the novelties, such as the fact that the 
   acquisition rate was increased (footnote 2).\\
\textcolor{blue}{This section has been completely reworked.} 

\item Also, I understand that fig. 5 comes from reference [14], updated with 
   the values of the RTPC instead of the ALICE-TPC, but it seems to me that 
   there are some parts of the picture which have to be fixed. I list them 
   here:\\
- why does it say 768 pads?\\
- why 7.6 microsec?\\
- the anode wire e grid make no sense here\\
- why L1 and L2 (trigger levels?) \\
\textcolor{blue}{Based on previous comment, we shortened the discussion and 
droped this figure. } 

\subsection*{ Chapter 4}
	/paragraph 4.1
\item I would really prefer to call it drift "velocity" instead of drift 
   "speed"\\
\textcolor{blue}{What is calibrated here is mainly the drift time, based 
on your next question as well, we clarified the text about these.} 
\textcolor{red}{To be implemented}

\item I don't really understand the procedure to find the drift velocity 
   described here and fig. 7. What is $Nb_{max}$ and $Nb_{max/2}$? Why does 
   $T_{max/2}$ correspond to $Nb_{max/2}$ and why is it used as $T_{max}$? If I 
   make a quick calculation and use $T_{max}$ ~ 65 * 100ns and a drift length 
   of ~ 3cm /cos(23$^{\circ}$) ~ 7.5 cm I get a $v_{drift}$ ~ 1.15 cm/microsec 
   << 5 cm/microsec which is the average value of drift velocity in drift 
   chambers.\\
The procedure according to me would be:\\
- for each hit associated to a track, whose time information is available, take 
the drift time $t_{drift_{i}}$;\\
- plot the distribution of these $t_{drift_{i}}$;\\
- find the $T_{min}$ = minimum $t_{drift_{i}}$ and $T_{max}$ = maximum 
$t_{drift_{i}}$\\
- compute $delta_{t} = T_{max} - T_{min}$ and extract the drift velocity by 
dividing the track length / $delta_{t}$. Figure 8 would have the $delta_{t}$ 
instead of the $T_{max}$ vs z.\\
Please provide more details on the applied procedure, since I have the open 
questions I wrote before and I could not understand it properly. Moreover, 
please quote the value of the found drift velocity. Is it in accordance with 
the simulations?\\
\textcolor{blue}{The text has been updated to simplify and clarify the procedure 
(in particular, some of the variable were not necessary). What we are doing is pretty 
similar to what you describe, but we might have overlooked some informations in the
original description. Hopefully it is now better.\\
Trigger time is at time bin 15, so T = 50*100~ns = 5$\mu$s.
The drift length is equal to 3cm/ cos(23$^{\circ}$ = 3.26 cm (I assume the 7.5 is a typo).
Giving $v_{drift}$ = 0.65 cm/$\mu$s. These values are exactly what was anticipated by
our early comparison of gaz mixtures based on MAGBOLTZ, so we assume there is no 
problem there.\\} 
\textcolor{red}{Actually the text still needs some work}

\item Figure 6: why are $T_{min}$ and $T_{max}$ not at the drift gap 
   extremities?\\
   \textcolor{blue}{The figure is updated.} 

        /paragraph 4.2

\item line 306: "in one bin" should be explained better. I understood it is the 
   subdivision in bins of the z coordinate, but it was not so obvious at the 
   first reading\\
\textcolor{blue}{The text is cleared.} 

\item Has the stability of the drift path been counter checked with simulations 
   varying the field according to the cathode radius variation? How much is the 
   field variation if the radius have some mm of uncertainty?\\
\textcolor{blue}{We did not study this case. } 

        /paragraph 4.3
\item The gain calibration is very quick. I would like more details, e.g. 
   "setting the simulation properly" intends only GEANT4 or also MAGBOLTZ or 
   GARFIELD? Were they used to simulate the drift and avalanche formation of 
   the electrons? And the diffusion from the gas? Was a specific code written 
   to simulate the electronics?\\
\textcolor{blue}{The section has been extended to provide more details. We 
implemented specific codes in our GEANT4 simulation to have a realistic 
description of the electron drifts and of the DAQ process.} 

\item line 330: "correction factors" are mentioned. On what? Why haven't only 
   good tracks been used for calibration? Lack of statistics?\\
\textcolor{blue}{We improved the gains by using a method based only on data,
the text has been clarified on the topic. All these calibration procedures
are performed with good tracks only.} 

\item When it says that the pad charge is "compared" to the neighboring ones, 
   what happens next? Is it the noise reduction step described later? If so, 
   please write it. \\
\textcolor{blue}{The gain extraction is following this order: \\
- pass 1: the gain of each pad is defined as the ratio between the mean 
recorded experimental ADCs to the mean simulated ADCs in the same track. Then 
for each pad, the gain ratios were collected from all the identified elastic 
tracks and these gains were fitted by a Landau function to give a
  first pass gain for each pad as the most probable value of the fit.\\
 - pass2: to make sure that the recorded ADCs by a given pad are similar to the 
 recorded ADCs by other pads in the same track, we refined this calibration. For this, 
 we compared the mean ADCs of each pad to the mean ADCs of the whole track. 
 This ratio is collected from all the elastic events and a gain correction 
 factor is extracted and applied giving a final gain for each pad.
  } 

\item Figure 11: I don't understand it. How many tracks does the histo contain? 
   If the single pad has 128 channels, how many channels does the single bin 
   contain? Moreover, is the dE calculated on pad by pad basis? In that case, 
   why don't you put just one point in the histo for each pad? What is the mean 
   number of channels or pads fired by one track? Sorry for the list of 
   questions, but I could not understand the figure.\\
   \textcolor{blue}{This figure only displays one track. Pads are single channels, 
the bins correspond to time (100 ns per bin). The dE is calculated overall in general,
we use this pad by pad method only for calibration. The average number of pad per 
track is 9, with most of the good tracks between 6 and 14.}

\item Figure 12 - please, put it after figure 11. Moreover, please change dEdx 
   to dE/dx on y axis. Why are the left and right pictures so different?  
   Talking about the deuteron band which is present only in the left figure and 
   the smear on the top part of the right picture. \\
\textcolor{red}{The figures layout will be updated.\\} 
   
\textcolor{blue}{The y-axis label on figure 12 is updated. Regarding the lower 
   peak in the left module of the RTPC, it is likely not deuterons. We added
   a footnote to clarify the question.} 

\subsection*{ Chapter 5}
        /paragraph 5.1	
     
\item Please detail a little bit more the procedure to remove the oscillatory 
   noise. It says "event by event and channel by channel": if the ADC threshold 
   is passed then a noise level is subtracted? If so, how is it calculated?\\
\textcolor{blue}{The oscillatory noise was removed just by rejecting all hits 
from a channel if too many of these hits fall on the noise curve in a given 
event. We did not use a threshold or pedestal concept, just a pad rejection 
if we observe the oscillating noise in the event, the text was clarified.} 

/paragraph 5.2
\item How was "10.5 mm" cut chosen? is it in xy plane or in 3-dimensions?\\
\textcolor{blue}{The distance between hits are calculated in 3D. This cut has 
been chosen because it enables to jump a one pad gap in case of a dead 
channel.} 

\item How was the "10 hits" cut chosen? What is the mean number of primary 
   ionization in this gas mixture?\\
   \textcolor{blue}{This cut is set to avoid tracks made out of random noise and
to limit computing time. The number of hits in good tracks at the end of the 
analysis is always above 30, with a mean around 135, so this cut has no impact 
on the good track selection.}

\item < dE/dx > is computed here as corrected Etot divided total track length. 
   Usually the mean dE/dx is computed with the truncated mean of the specific 
   energy losses of each hit. Isn't it possible to compute it like this here? 
   dx for each hit can be (maybe) computed from the difference in the hit 
   times. Sum $dE_{i}$/Sum $dx_{i}$ is not equal to Sum ($dE_{i}/dx_{i}$).\\
\textcolor{blue}{We compute dEdx with path-length and ADC- 
truncations, and a linear fit to ADC vs path-length to get slope and fit 
quality and dedx from fit. Unfortunately, it never seemed to give us much 
improvement compared to Sum $dE_{i}$/Sum $dx_{i}$. Note that, by the nature of 
the FADC electronics used, our hits always correspond to 100 ns time bins such that
each measured ADCs are almost dE/dx measurements already.}

\subsection*{ Chapter 6}

        /paragraph 6.2
\item I understand that efficiency is a ratio between two integrals, each 
   integral being the area of the Gaussian which fits the recoil mass 
   distribution in the inclusive and exclusive case. If so, would it be 
   possible to please draw both fits on the figure 14 and provide the integrals 
   to compute the mean efficiency? If I misunderstood, please comment.\\
\textcolor{blue}{This study has been performed by a student a long time
ago and we do not have all the individual fits anymore.} 

\subsection*{ Chapter 7} 

\item Is it "high rate environment" or "high readout rate"? They are connected, 
   but what is the rate the RTPC underwent during data taking?\\
\textcolor{red}{The rate of protons is of MHz in the chamber with a readout of
about 3kHz, so one could argue that it is both. Our point here is to emphasis
the high rate however, as these slow protons deposit a lot of energy in the 
chamber.} 

\item I understand the resolution and efficiency have been evaluated with the 2 
   GeV electrons. Is it foreseen to analyze also the 6 GeV electron runs, which 
   were the actual data taking? Is there an evaluation of the foreseen RTPC 
   performances in that case?\\
\textcolor{blue}{It is not possible to do such studies with the 6 GeV beam, because 
of the lack of an exclusive process to use as elastic cross sections drops dramatically 
in our acceptance at such high energy. We would have loved to and tried many alternate
channels, but none was detectable with reasonnable statistics.} 

\subsection*{ Some general remarks on the layout}

\item I would add to the keywords also GEM or gas electron multipliers or multi 
   pattern gas detectors (it depends on what is available).\\
\textcolor{blue}{Updated} 

\item Please put the units on the axis of every figure, together with the 
   physical quantity. In some figures it is only written in the caption or in 
   the text.\\
\textcolor{blue}{ } 

\item Sometimes to explain the dimension of something something like "250 mm 
   long" (line 67) while some other times "84-mm-long" (on line 83) is written. 
   Please make this uniform in the whole text (I just quoted two examples, but 
   there are more in the text).\\
\textcolor{blue}{Updated everywhere} 

\item In footnote 2, BoNuS is written BONUS and somewhere else in the text also 
   BoNus. Please uniform this to the correct one, which I think is BoNuS.\\
\textcolor{blue}{Updated to BoNuS everywhere} 

\item Please make the bibliography entries uniform: they are not all written 
   with the same layout.\\
\textcolor{blue}{Updated} 

List of in text corrections:

\item line 17: "GeV" --> "GeV/c"\\
\textcolor{blue}{Corrected} 

\item line 26: "nuclei" --> "nucleus"\\
\textcolor{blue}{Corrected} 

\item line 40: "to track" --> "to bend tracks"(since the magnetic field just 
   bends the track and then the detector tracks them)\\
\textcolor{blue}{Added} 

\item line 91: "second gap" --> "second gas gap"\\
\textcolor{blue}{Added} 

\item line 110: "from the GEMs" -->  something like "after they have been 
   multiplied by the GEMs" or "induced in the last gas gap" since the GEM is 
   the amplification system.\\
\textcolor{blue}{Added} 

\item line 131: "axial magnetic field" --> "solenoidal magnetic field"\\
\textcolor{blue}{Replaced} 

\item line 133: "challenge" --> "challenges"\\
\textcolor{blue}{Added} 

\item line 133-134: "radial TPC" it would be better to uniform the way to write 
   it in the whole text (like, "RTPC")\\
\textcolor{blue}{Uniformed} 

\item line 167: "stage" --> "stages"\\
\textcolor{blue}{Corrected} 

\item line 176: "ReadOut" should always be written in the same way (see line 
   168) unless this was specifically written in a different way in the cited 
   articles\\
\textcolor{blue}{Uniformed} 

\item line 208: cm$^{-1}$ . s$^{-1}$ --> cm$^{-1}$ cdot s$^{-1}$ (the dot is 
   not in the center vertically)\\
\textcolor{blue}{Corrected} 

\item line 208 again: "6.067" before it said "6.064", please fix the wrong one
\textcolor{blue}{Corrected. It is 6.064} \\

\item line 263: "2\%" --> I would put "around 2\%"\\
\textcolor{blue}{Added} 

\item line 268: "reconstructions" --> "reconstruction"\\
\textcolor{blue}{Corrected } 

\item line 269: "Paths" --> "Path"\\
\textcolor{blue}{Corrected} 

\item line 308: "codes" --> "code"\\
\textcolor{blue}{Corrected} 

\item line 405: "electron's" --> "electron"\\
\textcolor{blue}{Corrected} 

\item fig14 - caption: "W distribution" --> I would put "recoil mass W 
   distribution"\\
\textcolor{blue}{Added} 

\item table 1: $sigma_{p}$ is actually $sigma_{p/p}$, since it is in 
   percentage\\
\textcolor{blue}{Yes it is. Corrected } 

\item line 432: "Helium-4" --> "4He" (for uniformity, as was written before)\\
\textcolor{blue}{Changed to $^4$He} 

\item line 488: There is an extra "," at the beginning of the item\\
\textcolor{blue}{Cleaned}\\
~\\
\end{enumerate}

The topic of the article is interesting but I would recommend a revision.  
Please I would like the authors to address the questions asked and provide the 
required changes or explain why they think those changes are not 
necessary/correct. Thank you. Best regards.\\

\textcolor{blue}{We would like to thank the reviewer for the detailed remark that
helped improving the manuscript significantly.}

\end{document}
