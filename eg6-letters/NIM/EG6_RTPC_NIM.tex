
\documentclass[twocolumn,showpacs,superscriptaddress,groupedaddress]{revtex4}
\usepackage{graphicx}
\usepackage{dcolumn}
\usepackage{bm}  
\usepackage{amssymb}  
\usepackage{hyperref}
\hypersetup{colorlinks=true, urlcolor=blue, citecolor=blue}


\begin{document}


% the following line is for submission, including submission to the arXiv!!
%\hspace{5.2in} \mbox{Fermilab-Pub-04/xxx-E}

\title{\vspace{-15mm}\fontsize{24pt}{10pt}\selectfont\textbf{A New Radial Time 
Projection Chamber for CLAS at JLab}}
\input author_list.tex  

\date{\today}

\begin{abstract}
A new Radial Time Projection Chamber (RTPC) was developed at Jefferson 
laboratory (JLab) to track low-energy nuclear recoils for the purpose of 
measuring nuclear exclusive channels, such as, coherent Deeply Virtual Compton 
Scattering (DVCS) and coherent production of mesons on $^4$He. In such 
processes, $^4$He nucleus remains intact and recoils as a whole in the final 
state. The CEBAF Large Acceptance Spectrometer (CLAS) at JLab has been used for 
similar studies on proton targets, however it cannot track recoils with momenta 
less than 250 MeV/c. In 2009, we carried out measurements on $^4$He target 
using CLAS spectrometer, which was upgraded with the RTPC during the CLAS-EG6 
experimental run. The detector was positioned directly around a gaseous target, 
allowing a detection threshold of $~$100  MeV/c for $^4$He. This article 
discuses the RTPC design, work principle, developed calibration methods and 
detector performance.
\end{abstract}

\maketitle

%\tableofcontents

\section{Introduction} \label{sec:level1}

The CEBAF Large Acceptance Spectrometer (CLAS)~\cite{CLASref} is installed in 
the Hall-B of the Thomas Jefferson National Accelerator Facility (TJNAF), 
Newport News, Virginia, USA. The accelerator of TJNAF provides, simultaneously, 
a high power electron beam of energy up to 6 GeV and 100$\%$ duty cycle to 
three experimental Halls (A, B, C). The CLAS spectrometer, located in Hall-B, 
is composed of several sub-detectors and two magnets. The spectrometer was 
designed to track charged particles of momentum greater than 250 MeV/c as well 
as detecting real photons. Figure~\ref{fig:CLAS} shows a three dimensional 
representation of the CLAS detector, ordered with respect to their radial 
distance from the beam line as:

\begin{itemize}
 \item Three regions of Drift Chambers (DC) for tracking of charged 
    particles~\cite{DCref}.
 \item Superconducting toroidal magnet which enables the tracking in the DC by 
    bending the trajectories.
 \item Cherenvkov Counters (CC) to help in separating electrons from negative 
    particles, such as pions ~\cite{CCref}.
 \item Scintillation Counters (SC) to identify hadrons via measuring their 
    Time-Of-Flight (TOF)~\cite{TOFref}.
 \item Forward Electromagnetic Calorimeter (EC) which measures the energy of 
    the showering particles (electrons and photons) and neutrons~\cite{ECref}.
\end{itemize}

\begin{figure}[tbp]
\centering \includegraphics[scale=0.3]{fig/test_clas.png}
\caption{A three dimensional representation of the primary CLAS setup 
   represents the geometry and the relative position of each subsystem. From 
   outside, the electromagnetic calorimeter (green), time-of-flight system 
   (red), Cherenkov counters (purple), three drift chamber regions (blue) and 
the superconducting torus coils (yellow).} \label{fig:CLAS}
\end{figure}

The DVCS process is the exclusive electroproduction of a real photon from a 
quark of a hadronic target ($H$), $eH \rightarrow e' H' \gamma$. In the basic 
setup of CLAS, the real photons are detected by the forward EC, which covers 
polar angles from 8$^{\circ}$ to 45$^{\circ}$. With a 6 GeV electron beam, the 
majority of the DVCS photons are produced at polar angles below 15$^{\circ}$, 
where the acceptance of the EC is poor. In CLAS-E1DVCS experiment (2005), CLAS 
was upgraded with the addition of an Inner Calorimeter (IC). The IC is 
constructed from 424 lead-tungstate (PbWO$_{4}$) crystals, covering polar 
angles between 5$^{\circ}$ and 15$^{\circ}$. Each crystal is 16 cm long ,
corresponding to 17 radiation lengths, with a 1.33~$\times$~1.33 cm$^2$ front 
surface and a 1.6~$\times$~1.6 cm$^2$ back surface. The achieved energy 
resolution is around 3 to 4$\%$ for photon energies between 2 GeV and 5 GeV and 
the angular resolution is between 3 to 5 mrad for the same energy range 
\cite{Hyon-suk}.

The developed RTPC has been used during a three months experimental run in 2009 
with a longitudinally polarized electron beam of energy 6.064 GeV projected 
onto a gaseous $^{4}$He target to measure asymmetries in Deep Virtual Compton 
Scattering (DVCS)~\cite{proposal} and exclusive meson production. Moreover, a 
5~T solenoid was placed around the RTPC-target system to shield the detectors 
close to the beam line from Mollers electrons. The solenoid field plays also a 
major rule in finding the momentum of the charged particles in the RTPC by 
bending their tracks curvature.

In this paper we report the progress in designing and calibrating a radial TPC 
that had been used with JLab-CLAS spectrometer. This paper is organized as 
follows: section \ref{prev_tpc} presents an overview of the TPCs history and 
previous usage, section \ref{sec_design} details the physics considerations for 
the RTPC design and its internal structure, while section \ref{sec_readout} 
points the properties of the read-out system. The calibration strategies for 
such a detector are discussed in section \ref{sec_calib}. Finally, the 
performance of the RTPC is described in section \ref{sec_perf}.

\section{TPC technology} \label{prev_tpc}
In the late 1970s, David R. Nygren presented the concept of the Time projection 
Chambers (TPCs) as a new tool to track charged particles based on combining 
electric and magnetic fields within a sensitive gas volume. The ionized 
electrons are collected by a position-sensitive electron collection system.  
Later in 1997, F.~Sauli introduced the Gas Electron Multiplier (GEM) as an 
effective gas-gain element to improve the tracking efficiency of the TPCs 
\ref{gem_sauli}. Nowadays, the TPCs are widely used in physics experiments, 
such as high-energy physics (ALEPH, DELPHI), heavy-ion physics (STAR, ALICE) 
and rare-event searches (XENON, NEXT).

The first implementation of a radial TPC with JLab-CLAS spectrometer has been 
performed by the BoNuS experiment \cite{BONUS} in 2005. The success of the 
BoNuS-RTPC in detecting the slow protons motivated the development of this new 
RTPC. For the channels of interest in CLAS-EG6 experiment, the recoiled 
$^{4}He$ nuclei have momentum per charge around 150 MeV/c. The detection of 
these low-energy nuclei was the primary motivation for designing this detector. 

 
\section{CLAS-EG6 RTPC design} \label{sec_design}
The design of CLAS-EG6 RTPC has been chosen according to physics and 
experimental considerations. At 6 GeV incident electron beam energy, the recoil 
$^{4}He$ nuclei, from the coherent DVCS channel, have an average momentum (per 
charge) around 125 MeV/c, while the "standard" CLAS detects charged particles 
with a threshold of 250 MeV/c. On the experimental site, the chosen design has 
to fit within the available space inside a 5 Tesla Solenoid magnet that 
surrounds the target \cite{Hyon-suk}. Figure \ref{fig:RTPC2} shows an 
experimental view of the RTPC before being inserted into the solenoid. 


The CLAS-EG6 RTPC is a 200~mm long with a 150~mm diameter TPC. While a charged 
particle traverses a gas-sensitive volume, the ionized electrons are released 
and drift, under the effect of an applied electromagnetic field, outward 
perpendicularly to the beam direction. These electrons are then amplified by 
three layers of GEMs and detected by a curved readout system on the external 
shell as can be seen in the figure~\ref{fig:RTPC_1_4}. As for CLAS-BoNuS, the 
EG6-RTPC has two detection halves but unlike BoNuS they are part of a single 
chamber. Each half has independent GEM amplification systems that cover around 
80\% of the azimuthal angle.

\begin{figure}[tb]
\centering
\includegraphics[scale=0.19]{fig/RTPC_exp.png}
\caption{View of the RTPC ready to be inserted in the solenoid. } 
\label{fig:RTPC2}
\end{figure}

\begin{figure}[tb]
\centering
\includegraphics[scale=0.28]{fig/RTPC_1_all.png}
\caption{Schematic drawing of CLAS-EG6 RTPC taken on a plane perpendicular to 
the beam line, see text for description of the elements.} 
\label{fig:RTPC_1_4}
\end{figure} 

As seen in figure~\ref{fig:RTPC_1_4}, the CLAS-EG6 RTPC has the following 
substructure, from the beam line to the exterior:\\
\begin{itemize}
   \item The target extends along the RTPC's central z-axis, with a diameter of 
      6 mm. It is enclosed in a 27-$\mu$m-thick Kapton wall.

   \item The first gas gap extends from 3 mm to 20 mm radial distances. It is 
      filled with $^{4}He$ gas at 1~atm. During run, this region is swarmed by 
      M\o ller electrons induced by the beam, such that filling this region 
      with a light gas like $^{4}He$ minimizes secondary interactions. This 
      region is surrounded by a 4 $\mu$m thick window made of aluminized Mylar 
      and connected to ground.                                                                
   \item The gap region extends between 20~mm and 30~mm and is filled with a 
      gas mixture of 80$\%$ Neon and 20$\%$ Dimethyl Ether (DME). This region 
      is surrounded by a 4~$\mu$m thick window made of aluminized Mylar, which 
      serves as the cathode.

   \item The drift region is filled with the Ne-DME gas mixture, it extends 
      from the cathode to the first Gas Electron Multiplier(GEM), 60 mm away 
      from the beam axis. The electric field in this region is around 500~V/cm 
      (???) and perpendicular to the beam. Field cages are installed at both 
      ends of the detector in this region. Progressive voltage depending on 
      radial position is applied on these to maintain a uniform electric field 
      perpendicular to the beam line at the edges of the RTPC.
   
   \item The electron amplification system is composed of three GEMs located at 
      radii, 60 mm, 63 mm and 66 mm. In this configuration, the first GEM layer 
      represents the anode.

   \item The readout board has internal radius of 69~mm and collects the 
      charges. Preamplifiers are plugged directly on the outer side and are 
      transmitting the signal to the data acquisition electronics.
\end{itemize}

\begin{figure}[tbp]
\centering
\includegraphics[scale=0.70]{fig/GEM_photo.jpg}
\caption{A schematic layout of the GEMs system in one module of the RTPC.} 
\label{fig:GEMs}
\end{figure}

The GEM system has been mainly chosen for this detector for its flexibility, 
allowing the curved amplification surface. Also GEMs are known to create less 
sparks than in other gaseous detectors, which provides a cleaner detection of 
slow nuclei like $^4$He. GEMs are made from an insulator layer, Kapton in our 
case, sandwiched between two 5 $\mu$m copper layers. The mesh of each GEM layer 
is chemically etched with 50$\mu$m diameter holes in double-conical cross 
section shapes as can be seen in figure~\ref{fig:GEMs}. A potential difference 
of 400 V is applied between the two copper layers of each GEM creating a very 
strong electric field in the holes. Such strong fields lead to high ionization 
from initial electrons and therefore amplification of the signal. Also, a 
potential difference of 150 V is set between the GEMs to push the amplified 
electrons toward the readout pads. The gain of each GEM layer is of order 100, 
making a total gain of about $10^{6}$.\\

\section{Readout System} \label{sec_readout}
The RTPC electron collecting system has 3200 readout pads. These collection 
elements are located at the end of the amplification region, 69 mm from the 
central axis.

\begin{figure}[tb]
   \centering
   \includegraphics[scale=0.55]{fig/PADs.png}
   \caption[]{A schematic representation of a part of the readout system.  The 
   shaded sixteen pads are a group of pads that are connected to the same 
pre-amplifier.} \label{fig:PADs}
\end{figure}

Figure \ref{fig:PADs} shows a schematic drawing of the size and the alignment 
configuration of the pads. Each readout pad is 5 mm in length and 4.45 mm in 
width the shift between the rows allows to reduce aliasing. Each half of the 
RTPC has 40 rows and 40 columns of pads. The shaded region in the 
figure~\ref{fig:PADs} shows how pads are grouped to the 16 channels 
pre-amplifiers. The Time to Digital Converter (TDC) units are 114~ns and 
indicates the time taken by the electron to drift from the ionization point to 
the readout board. On the the other side, the charge information is measured in 
Analog-to-Digital-Converter (ADC) units without specific normalization. 

Missing items in description:

Detail the electronic setup: Andrea ? or other Italians?

The gas mixture choice: ???

Discuss HV setting: ???

\section{Calibration} \label{sec_calib}
Two quantities are recorded for each detected electron at the readout board, 
time (TDCs) and signal amplitude (ADCs) (/!\\ are those amplitude or charge 
preamps ???). The timing information provides trajectory determination 
resulting in momentum per unit charge measurement. Going from a collection of 
the TDCs to a momentum measurement requires good knowledge of the drift speed 
and paths followed by the electrons released in the gas. The recorded ADCs give 
the deposited energy per unit of length ($\small{\frac{dE}{dX}}$) which, 
together with the momentum calculated from the trajectory, enables particle 
identification.


\subsection{Drift Speed Parametrization}

We can measure the drift speed using the tracks detected in the RTPC, in figure 
\ref{fig:RTPC_signals} we represent a typical $^{4}He$ track (in green). After 
it causes ionization in the drift region, the released electrons (in black) 
drift to the detection plane under the effect of the electric field. The 
electrons released close to the cathode take the most time to reach the readout 
pads while the geometrical symmetry insure they always travel the same 
distance. So by identifying the maximum TDC measured, we can infer the drift 
speed of the electrons in the RTPC.\\

\begin{figure}[tb]
\centering
\includegraphics[scale=0.35]{fig/RTPC_2.png}
\caption[]{A schematic drawing of a $^{4}He$ track (in green) traversing the 
drift region, with the drift paths followed by the electrons (in black). } 
\label{fig:RTPC_signals}
\end{figure}

We first measure the drift speed along the 200 mm RTPC's length to take into 
account variations in the electric and magnetic field, some of which are known 
(see figure \ref{fig:B_MAP}) some might be due to slight shifts in the 
geometry. We verified this by looking at the TDC profile generated in the 
chamber by all our tracks (see figure \ref{fig:TDC_profile}). The time profile 
of registered hits clearly show the dropping edge expected from the geometrical 
considerations previously mentioned. We define a value named $TDC_{Max/2}$ at 
which the dropping edge passes half the maximum number of hits in the 
histogram, this value is inversely proportional to the drift speed. The result 
can be seen in figure \ref{fig:RunNumber_61551_TDCmax_Zslice} showing the 
dependence of drift speed with position along the beam line axis in the 
chamber. 

\begin{figure}[tb]
   \centering
   \includegraphics[scale=0.35]{fig/B_MAP.png}
   \caption[]{Schematic draw in of the solenoid magnetic vectors inside the 
   RTPC. In this draw, r is the radial distance from the target's axis and z is 
the longitudinal position along the RTPC with its center at zero.} 
\label{fig:B_MAP}
\end{figure}

\begin{figure}[tb]
   \centering
   \includegraphics[scale=0.3]{fig/TDC_profile.png}
   \caption[]{Time profile of the collected hits in one experimental run.  } 
   \label{fig:TDC_profile}
\end{figure}

\begin{figure}[tb]
\centering
\includegraphics[scale=0.36]{fig/RunNumber_61551_TDCmax_Zslice.png}
\caption{Time profile distribution for the collected hits in one experimental 
run. } \label{fig:RunNumber_61551_TDCmax_Zslice}
\end{figure}

Due to the non perfect experimental conditions and in particular changes in the 
gas proportions, the drift speed might also change over the 3 months run.  
Figure \ref{fig:Drift_run_number_1} shows the $TDC_{Max/2}$ values for 
individual runs (approximately 2 hours long). We observe a significant 
variation over time that we parametrized with a fit in order to take it into 
account in the track reconstruction code. Dealing with such situation is 
possible by adapting the right drift speed to the electrons as a function of 
geometry and time (we did not see any correlation between the effects). These 
functions were extracted for our entire data sets and implemented in the 
reconstructions code.

\begin{figure}[tb]
\hspace*{-1.8cm}
\includegraphics[scale=0.26]{fig/Drift_run_number_1.png}
\caption{$TDC_{Nbmax/2}$ versus the experimental run numbers (time).  } 
\label{fig:Drift_run_number_1}
\end{figure}

   
\subsection{Drift Paths Parametrization}

The drift paths are the trajectories which an electron follows in the gas. The 
standard software to calculate drift paths is the MAGBOLTZ program 
~\cite{MAGBOLTZ}. For its calculations, MAGBOLTZ needs to know the detector's 
geometry, the exact composition of the gas mixture and of course the electric 
and the magnetic fields. We used such calculations as a first calibration, but 
as can be seen with the drift speed, conditions are far from stable in the 
chamber. Moreover, because we use of a thin foil as a cathode, the geometry 
knowledge is limited to a millimeter precision. These problems, already 
encountered for the BoNuS RTPC calibration (ref), have motivated the 
acquisition of specific calibration runs. These are taken with lower energy 
electron beam (1.20 and 1.27 GeV) to enhance the cross section of the elastic 
process ($e ^{4}He \rightarrow e ^{4}He$). In this process the knowledge of the 
electron kinematic gives us the helium nuclei kinematic and therefore allow us 
to calibrate our drift paths independently of our knowledge of the exact 
conditions in the chamber.

The drift paths' parametrization is possible using identified elastic events 
from our experiment and simulate copy of them in a GEANT4 
simulation~\cite{GEANT4}. Then by comparing the GEANT4 calculated position of 
the helium nuclei to the list of hits effectively measured in the chamber, we 
can reconstruct the electrons drift path in the RTPC. However, because of the 
magnetic field, the drift paths are not linear. So to perform the extraction, 
we make a first assumption of a linear dependence between the radius of 
emission and the TDC of detection and then refine our result. As it happens, 
the curvature is minimal and this process converge already on the second 
iteration. Here are the step of the extraction procedure (we work in a polar 
coordinate system ($R$,$\phi$), $\Delta\phi$ being the difference between the 
$\phi$ of the pad measuring a hit and the $\phi$ of the simulated helium):

\begin{figure}[tb]
\centering
%\hspace{-0.95cm}
\includegraphics[scale=0.37]{fig/TdcR_check_p1_10.png}
\caption{R versus TDC distribution for the simulated hits locating in a single 
$z$ bin. The maximum $R$ equals 60 mm (close to the anode) at minimum TDCs 
(Trigger time is 15), and minimum R equals 30 mm (close to the cathode) at 
maximum TDCs ($\sim75$).}
\label{fig:R_correlation}
\end{figure} 

\begin{itemize}

   \item The linear assumption between R and time(TDC) is necessary to link the 
      generated hits from GEANT4 to physical hits. Figure 
      \ref{fig:R_correlation} shows our initial $R$ versus TDC. 

   \item For these selected hits, we look at $\Delta\phi$ ($\Delta \phi$ = 
      $\phi_{sim.}$ - $\phi_{hit\_pad}$) versus TDC. We clearly obtain a 
      correlation shown in figure~\ref{fig:1st_pass_delta_phi}. This $\Delta 
      \phi$ change with TDC shows the drift in $phi$ of the electrons due to 
      the Lorentz angle. 

      \begin{figure}[tb]
         \centering
         \includegraphics[scale=0.37]{fig/TdcPhi_p1_10.png}
         \caption{$\Delta \phi$ versus TDC distribution for the simulated hits 
         locating in one slice centered at 35 mm.}
         \label{fig:1st_pass_delta_phi}
      \end{figure}

   \item Using this first measurement, we can calculate a new correlation 
      between $R$ and TDC. The result, shown in figure~\ref{fig:R_TDC}, is very 
      close to linear explaining the fast convergence of our procedure. 


      \begin{figure}[tb]
         \centering
         \includegraphics[scale=0.37]{fig/R_TDC.png}
         \caption{The calculated R versus the time is corrected by the 
         parameters of $\Delta \phi$ from the first pass.}
         \label{fig:R_TDC}
      \end{figure}


   \item In the second pass, we associate simulated helium positions to 
      physical hits according to their calculated correlation and obtain new 
      $\Delta \phi$ distributions.

   \item We finally parametrize the $\Delta\phi$ distributions with a fit as 
      can be seen in figure~\ref{fig:DELTA_PHI_TDC}, which shows the final 
      $\Delta \phi$ versus TDC with a red line representing our drift paths 
      fit.
\end{itemize}


\begin{figure}[tb]
\centering
\includegraphics[scale=0.37]{fig/FitResult_p2_10.png}
\caption{$\Delta \phi$ versus TDC distribution in the second pass for the slice 
centered at 35 mm longitudinal position along the RTPC. The red line represents 
the final drift paths in this slice.}
\label{fig:DELTA_PHI_TDC}
\end{figure}

To verify the stability of the drift paths, this procedure was carried out 
using both the 1.204 GeV data set from the beginning of the run period and the 
1.269 GeV data from the end of the run period. As a result, we found very 
similar drift paths and reached the conclusion that whatever changed in the 
setting affected only the drift speed.

\subsection{Gain Calibration}

The parametrization of the drift speed and the drift paths were carried out 
using only the TDCs, gain calibration goal is to equalize the gains in ADC/MeV 
over the full RTPC. The gain of each pad is the ratio between the actual 
deposited energy and the registered ADC value. We identified three different 
methods to extract such gains. The first is to compare the experimental energy 
deposit to the expected value calculated from Bethe-Block formula (ref). The 
second method is to compare the experimental ADCs to the energy deposited in 
GEANT4 by simulated tracks (using the same elastic events than for the drift 
paths calibration). In the third method, the ADCs of each pad is compared to 
the ADCs of the other pads inside each track.

(Detail of 1st method: Nathan)

To perform the second method it requires a very good GEANT4 simulation 
including drift paths, but also drift and amplification spreads of the charges 
before reaching the pad, so that the simulated hits match the experimental 
ones. Moreover, the simulation has to match the data acquisition (DAQ) features 
that can lead to cutting out hits. After setting the simulation properly, we 
can compare simulation to data on an event by event basis as in 
figure~\ref{fig:EVENT_adc_tdc}. In this step, the gain for each pad is defined 
as the ratio of the measured ADCs to the simulated deposited energy.

(Detail of 3rd method: Mohammad)

\begin{figure}[tb]
\includegraphics[scale=0.350]{fig/EVENT_adc_tdc.png}
\caption{Simulated (upper) and experimental (lower) ADCs and TDCs distributions 
of a track. The colors indicate the pads, same color in top and bottom indicate 
that they are the same pad.}
\label{fig:EVENT_adc_tdc}
\end{figure}

\begin{figure}[tb]
   \includegraphics[scale=0.26]{fig/dedx_p_exp_1st.png}
   \caption{$\small{\frac{dE}{dX}}$ vs. p distribution for the left half of the 
      RTPC (left) and for the right half (right). Here, $\small{\frac{dE}{dX}}$ 
   is calculated using the gains of the first method.  The lines are 
theoretical expectations from Bethe-Bloch formula for $^4$He, $^3$He, $^3$H and 
$^2$H (d).}
\label{fig:dedx_p_exp_1st}
\end{figure}

\begin{figure}[tb]
\centering
\includegraphics[scale=0.26]{fig/dedx_p_exp_2nd.png}
\caption{$\small{\frac{dE}{dX}}$ vs. p distribution for the left half of the 
   RTPC (left) and for the right half (right). Here, $\small{\frac{dE}{dX}}$ is 
   calculated using the gains of the second method.  The lines are theoretical 
   expectations from Bethe-Bloch formula for $^4$He, $^3$He, $^3$H and $^2$H 
(d).}
\label{fig:dedx_p_exp_2nd}
\end{figure}

A set of gains has been extracted from each method, in 
figure~\ref{fig:dedx_p_exp_1st}, we have $\small{\frac{dE}{dX}}$ calculated 
using the first method's gains, while in Figure~\ref{fig:dedx_p_exp_2nd} we use 
the second method's gains. From these, we concluded that the gains of the 
second method match best the theoretical lines. This is not surprising since 
calibrating with full tracks leads to mixing the gains from the different pads.  
So it seems that even with several iterations, we could not decipher properly 
the effects of individual pads with this method.

\subsection{Noise Rejection}

(To be written by Nathan) 

\begin{figure}[tb]
\hspace{-0.4cm}
\includegraphics[scale=0.27]{fig/noisy_pad_before_rejection.png}
\caption{}
\label{fig:noisy_pad_before_rejection}
\end{figure}

\begin{figure}[tb]
\hspace{-0.4cm}
\includegraphics[scale=0.27]{fig/noisy_pad_after_rejection.png}
\caption{}
\label{fig:noisy_pad_after_rejection}
\end{figure}

\section{Track Reconstruction}\label{sec_perf}

In order to reconstruct tracks we first select good hits. This means rejecting 
out-of-time hits and hits linked to the electronic noise. The second step is 
space-reconstructing the hits using the extracted drift speed and drift path 
parameters. For each registered hit, we calculate a position of emission from 
the recorded time (TDC) and the position of the recording pad. The third step 
is to create chains of hits. The maximum distance between two close adjacent 
hits has to be less than 10.5 mm to chain them, this roughly correspond to 
neighbors and next to neighbors. Then, we fit the chains that have a minimum of 
10 hits. We make the fit in two iterations, first, all the hits of the chain 
together with the beam line are fitted with a helix. For the second iteration, 
the hits that are 5 mm or farther from the first fit are excluded.

For the energy deposit, $\frac{dE}{dx}$ is calculated in this way:
\begin{equation}
 \left\langle \frac{dE}{dX} \right\rangle= \frac{\sum\limits_{i} \frac{ADC_{i}}{Gi}}{vtl}
\end{equation}
Where the sum runs over all the hits of the track and $G_{i}$ is the gain of 
the associated pad. The $vtl$ is the visible track length in the active drift 
volume. 

(We need some performance studies here: Mohammad)

\section{Conclusion}




\input acknowledgement.tex  

\begin{thebibliography}{99}

\bibitem{CLASref}
   B.A. Mecking et al., The CEBAF large acceptance spectrometer, Nucl. Inst. 
   and Meth. A 503, 513 (2003).

\bibitem{proposal}
   K.Hafidi et al., Deeply virtual Comton scattering off $^{4}$He, Jlab 
   proposal to PAC33 (2007).

\bibitem{DCref}
   M.D. Mestayer et al., The CLAS drift chamber System, Nucl. Inst.  and Meth.  
   A 449, 81 (2000).

\bibitem{CCref}
   G. Adams et al., The CLAS Cerenkov detector", Nucl. Inst. and Meth. A 465, 
   414 (2001).

\bibitem{TOFref}
   E.S. Smith et al., The time-of-flight system for CLAS, Nucl.  Inst. and 
   Meth. A 432, 265 (1999).

\bibitem{ECref}
   M. Amarian et al., The CLAS forward electromagnetic calorimeter, Nucl.  
   Inst. and Meth. A 460, 239 (2001). 

\bibitem{Hyon-suk}
   Hyon-Suk Jo, Etude de la Diffusion Compton Profond{\'e}ment Virtuelle Sur le 
   Nucl{\'e}on avec le D{\'e}tecteur CLAS de Jefferson Lab: Mesure des Sections 
   Efficaces polaris{\'e}es et non polaris{\'e}es, IPNO-Thesis, 2007.

\bibitem{gem_sauli}
   F. Sauli, GEM: A new concept for electron amplification in gas detectors, 
   Nucl. Instr. and Meth. A 386, 531 (1997).

\bibitem{BONUS}
   S. Tkachenko et al., Measurement of the nearly free neutron structure 
   function using spectator tagging in inelastic $^{2}H(e,e'p)X$ scattering 
   with CLAS,	Phys. Rev. C 89, 045206 (2014).

\bibitem{MAGBOLTZ}
   S. Biagi, Monte Carlo simulation of electron drift and diffusion in counting 
   gases under the influence of electric and magnetic fields, Nucl.  Inst. and 
   Meth. in Phy. Res. A, vol. 421, pp. 234-240, 1999.

\bibitem{GEANT4}
http://geant4.cern.ch

\end{thebibliography}

\end{document}

