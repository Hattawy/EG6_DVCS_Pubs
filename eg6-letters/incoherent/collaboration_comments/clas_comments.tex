
%\documentclass[a4paper,11pt,twoside]{ThesisStyle}
\documentclass[a4paper,11pt,twoside]{article}




\date{\today}
\usepackage{amsmath,amssymb}             % AMS Math
% \usepackage[french]{babel}
\usepackage[latin1]{inputenc}
\usepackage[OT1]{fontenc}
\usepackage[left=2.7cm,right=1.7cm,top=1.6cm,bottom=1.6cm,includefoot,includehead,headheight=13.6pt]{geometry}
\usepackage{setspace}
\usepackage{epigraph}
\usepackage{lineno}


%\usepackage{arev}
%\usepackage[bitstream-charter]{mathdesign}
%\usepackage[urw-garamond]{mathdesign}
%\usepackage[sfmath]{kpfonts} %% sfmath option only to make math in sans serif. Probablye only for use when base font is sans serif.
%\renewcommand*\familydefault{\sfdefault} %% Only if the base font of the document is to be sans serif
\usepackage[sc]{mathpazo}
\linespread{1.05}   
\usepackage[T1]{fontenc}



% Table of contents for each chapter

\usepackage[nottoc, notlof, notlot]{tocbibind}
\usepackage{minitoc}
\setcounter{minitocdepth}{2}
\mtcindent=15pt
% Use \minitoc where to put a table of contents

\usepackage{aecompl}

% Glossary / list of abbreviations

\usepackage[intoc]{nomencl}
\renewcommand{\nomname}{List of Abbreviations}

\makenomenclature

% My pdf code

\usepackage{graphicx,type1cm,eso-pic,color}
\usepackage{lscape}

  \usepackage[pagebackref,hyperindex=true]{hyperref}

%\geometry{letterpaper}
%\graphicspath{{.}{images/}}

% nicer backref links
\renewcommand*{\backref}[1]{}
\renewcommand*{\backrefalt}[4]{%
\ifcase #1 %
(Not cited.)%
\or
(Cited on page~#2.)%
\else
(Cited on pages~#2.)%
\fi}
\renewcommand*{\backrefsep}{, }
\renewcommand*{\backreftwosep}{ and~}
\renewcommand*{\backreflastsep}{ and~}

% Links in pdf
\usepackage{color}
\definecolor{linkcol}{rgb}{0,0,0.4} 
\definecolor{citecol}{rgb}{0.5,0,0} 

% Change this to change the informations included in the pdf file

% See hyperref documentation for information on those parameters

\hypersetup
{
bookmarksopen=true,
pdftitle="",
pdfauthor="", pdfsubject="", %subject of the document
%pdftoolbar=false, % toolbar hidden
pdfmenubar=true, %menubar shown
pdfhighlight=/O, %effect of clicking on a link
colorlinks=true, %couleurs sur les liens hypertextes
pdfpagemode=None, %aucun mode de page
pdfpagelayout=SinglePage, %ouverture en simple page
pdffitwindow=true, %pages ouvertes entierement dans toute la fenetre
linkcolor=linkcol, %couleur des liens hypertextes internes
citecolor=citecol, %couleur des liens pour les citations
urlcolor=linkcol %couleur des liens pour les url
}

% definitions.
% -------------------

\setcounter{secnumdepth}{3}
\setcounter{tocdepth}{2}

% Some useful commands and shortcut for maths:  partial derivative and stuff
\newcommand{\xbp}{$x_{Bj}$}
\newcommand{\xb}{$x_{Bj}~$}
\newcommand{\ptp}{$p_\perp^2$}
\newcommand{\pt}{$p_\perp^2~$}
\newcommand{\dptp}{$\Delta \langle p_\perp^2 \rangle$}
\newcommand{\dpt}{$\Delta \langle p_\perp^2 \rangle ~$}

\brokenpenalty10000\relax

\newcommand{\pd}[2]{\frac{\partial #1}{\partial #2}}
\def\abs{\operatorname{abs}}
\def\argmax{\operatornamewithlimits{arg\,max}}
\def\argmin{\operatornamewithlimits{arg\,min}}
\def\diag{\operatorname{Diag}}
\newcommand{\eqRef}[1]{(\ref{#1})}

\usepackage{rotating}                    % Sideways of figures & tables
%\usepackage{bibunits}
%\usepackage[sectionbib]{chapterbib}          % Cross-reference package (Natural BiB)
%\usepackage{natbib}                  % Put References at the end of each chapter
                                         % Do not put 'sectionbib' option here.
                                         % Sectionbib option in 'natbib' will do.
\usepackage{fancyhdr}                    % Fancy Header and Footer

% \usepackage{txfonts}                     % Public Times New Roman text & math font
  
%%% Fancy Header %%%%%%%%%%%%%%%%%%%%%%%%%%%%%%%%%%%%%%%%%%%%%%%%%%%%%%%%%%%%%%%%%%
% Fancy Header Style Options

\pagestyle{fancy}                       % Sets fancy header and footer
\fancyfoot{}                            % Delete current footer settings

%\renewcommand{\chaptermark}[1]{         % Lower Case Chapter marker style
%  \markboth{\chaptername\ \thechapter.\ #1}}{}} %

%\renewcommand{\sectionmark}[1]{         % Lower case Section marker style
%  \markright{\thesection.\ #1}}         %

\fancyhead[LE,RO]{\bfseries\thepage}    % Page number (boldface) in left on even
% pages and right on odd pages
\fancyhead[RE]{\bfseries\nouppercase{\leftmark}}      % Chapter in the right on even pages
\fancyhead[LO]{\bfseries\nouppercase{\rightmark}}     % Section in the left on odd pages

\let\headruleORIG\headrule
\renewcommand{\headrule}{\color{black} \headruleORIG}
\renewcommand{\headrulewidth}{1.0pt}
\usepackage{colortbl}
\arrayrulecolor{black}

\fancypagestyle{plain}{
  \fancyhead{}
  \fancyfoot{}
  \renewcommand{\headrulewidth}{0pt}
}

%\usepackage{algorithm}
%\usepackage[noend]{algorithmic}

%%% Clear Header %%%%%%%%%%%%%%%%%%%%%%%%%%%%%%%%%%%%%%%%%%%%%%%%%%%%%%%%%%%%%%%%%%
% Clear Header Style on the Last Empty Odd pages
\makeatletter

\def\cleardoublepage{\clearpage\if@twoside \ifodd\c@page\else%
  \hbox{}%
  \thispagestyle{empty}%              % Empty header styles
  \newpage%
  \if@twocolumn\hbox{}\newpage\fi\fi\fi}

\makeatother
 
%%%%%%%%%%%%%%%%%%%%%%%%%%%%%%%%%%%%%%%%%%%%%%%%%%%%%%%%%%%%%%%%%%%%%%%%%%%%%%% 
% Prints your review date and 'Draft Version' (From Josullvn, CS, CMU)
\newcommand{\reviewtimetoday}[2]{\special{!userdict begin
    /bop-hook{gsave 20 710 translate 45 rotate 0.8 setgray
      /Times-Roman findfont 12 scalefont setfont 0 0   moveto (#1) show
      0 -12 moveto (#2) show grestore}def end}}
% You can turn on or off this option.
% \reviewtimetoday{\today}{Draft Version}
%%%%%%%%%%%%%%%%%%%%%%%%%%%%%%%%%%%%%%%%%%%%%%%%%%%%%%%%%%%%%%%%%%%%%%%%%%%%%%% 

\newenvironment{maxime}[1]
{
\vspace*{0cm}
\hfill
\begin{minipage}{0.5\textwidth}%
%\rule[0.5ex]{\textwidth}{0.1mm}\\%
\hrulefill $\:$ {\bf #1}\\
%\vspace*{-0.25cm}
\it 
}%
{%

\hrulefill
\vspace*{0.5cm}%
\end{minipage}
}

\let\minitocORIG\minitoc
\renewcommand{\minitoc}{\minitocORIG \vspace{1.5em}}

\usepackage{multirow}
%\usepackage{slashbox}

\newenvironment{bulletList}%
{ \begin{list}%
	{$\bullet$}%
	{\setlength{\labelwidth}{25pt}%
	 \setlength{\leftmargin}{30pt}%
	 \setlength{\itemsep}{\parsep}}}%
{ \end{list} }

\newtheorem{definition}{D�finition}
\renewcommand{\epsilon}{\varepsilon}

% centered page environment

\newenvironment{vcenterpage}
{\newpage\vspace*{\fill}\thispagestyle{empty}\renewcommand{\headrulewidth}{0pt}}
{\vspace*{\fill}}


\begin{document}


\section{Comments from Daniel Carman}


\begin{enumerate}

\item Page 1:\\
 - Abstract:\\
   - Line 4. Use "... azimuthal angle ...".
   \textcolor{blue}{Done} 

   - Line 5. Use "... of the beam spin asymmetry (BSA) was ..."
   \textcolor{blue}{is -> was, corrected!}

   - Line 6. Use "$Q^2$".
   \textcolor{blue}{Done}

   - Line 9. Use "... and for exploring the origin ...".
   \textcolor{blue}{Done.}

 - Line 36. Use "... a three-dimensional (3-D) ..."
   \textcolor{blue}{Added.}

 - Line 38. Use "The measurement of free proton DVCS has been ...".
   \textcolor{blue}{Modified.}

 - Line 45. Use "... similar 3-D picture of the ...".
   \textcolor{blue}{Corrected}

 - Line 46. Use "... case, however, two ...".
   \textcolor{blue}{Done}

 - Line 65. Use "Fig. 1".
   \textcolor{blue}{Kept as FIG.1. We show it in the text as shown in the 
      caption.}

 - Line 67. Use "i.e.".
   \textcolor{blue}{Done}
~\\
\item Page 2:\\
 - Fig. 1 caption.\\
   - Line 5. Use "... is the nucleon longitudinal ...".
   \textcolor{blue}{Done}

 - Line 72. Use "chirally even".
   \textcolor{blue}{Done}

 - Eq. (1). Use comma after equation, not a period.
   \textcolor{blue}{Done}

 - Line 84. Use "... is the virtual photoproduction differential cross section 
      ...".
   \textcolor{red}{We do not agree here.}

 - Eq. (2). Add a comma after the equation.
   \textcolor{blue}{Done}
 
 - Line 93. Use "abovementioned".
   \textcolor{blue}{Done}
 
 - Eq. (3). Add a comma after the equation.
   \textcolor{blue}{Done}
 
 - Line 103. Use "... where $\xi$ is the ...".
    \textcolor{blue}{Done}
 
 - Line 105. Use "Similar expressions apply for the GPDs ...". Provide a 
   reference here.
     \textcolor{blue}{Done}
 
 - Line 108. Use "longitudinally polarized".
      \textcolor{blue}{Done}
 
- Line 113. Use "... centered 64~cm upstream ...".
      \textcolor{blue}{Done}
 
- Line 115. Use "... [40] was supplemented ...".
      \textcolor{blue}{Done}
 
- Line 117. Use "The IC extended ...".
      \textcolor{blue}{Done}
 
- Line 123. Use "beamline".
      \textcolor{blue}{Done}
 
- Line 138. Use "... timing cut was used to separate the EC ...".
      \textcolor{blue}{Done}
 
- Line 142. Use "... photons were mostly soft ..."
     \textcolor{blue}{Done}
~\\
  \item Page 3:\\
 - Line 144. Use "... cuts easily eliminated.".
    \textcolor{blue}{Done}

- Line 145. Use "... photon was considered ...".
     \textcolor{blue}{Done }

- Line 146. Use "... on the particle identification."
     \textcolor{blue}{Done }

- Line 151. Use "$Q^2$".
     \textcolor{blue}{Done}

- Line 153. Use "$W$".
     \textcolor{blue}{Done }

- Line 155. Use "The squared transferred momentum to the recoil proton $t$, 
  calculated from the four-momentum vectors of the incoming ...".
     \textcolor{blue}{Done }

- Line 158. Use "... at a given $Q^2$ and $W$ ...".
     \textcolor{blue}{Done}

- Line 161. Use "specifically used the ...".
     \textcolor{blue}{Done}

- Line 173. Use "Fig. 2".
     \textcolor{blue}{We keep it as in the caption.}

- Line 178. Use "... and we saw no sizable effect that could be ...".
     \textcolor{blue}{Done}

- Line 183. Use "Fig. 3".
     \textcolor{blue}{We keep it as in the caption.}

- Line 184. Use "... that contributed ...".
     \textcolor{blue}{Done}

- Line 187. Use "... escapes detection.".
     \textcolor{blue}{Done }

- Line 190. Use "... to be 6.5\% by ...".
     \textcolor{blue}{Done }

- Line 195. Use "... that were wrongly identified ...".
     \textcolor{blue}{Done}

- Fig. 2 caption.\\
   a- Line 2. Use "... to bottom are: ...".
    \textcolor{blue}{Done}\\
   b- Line 7. Use "... event candidates ..."
    \textcolor{blue}{Done}\\
   c- Line 8. Use "... events that passed ...".
    \textcolor{blue}{Done}\\
   d- Line 9. Use "... all of these cuts ...".
    \textcolor{blue}{Done}

 - Line 199. Use "... events were calculated as ...".
    \textcolor{blue}{Done}
~\\
\item Page 4:\\
 - Line 203. Use "... to be 8 to 10\%".
     \textcolor{blue}{Done}

- Eq. (6). Equation should end with a comma, not a period.
      \textcolor{blue}{Done}

- Line 206. Use "... DVCS events for the positive ...".
      \textcolor{blue}{Done}

- Line 207. Use "... beam-helicity states, ...".
      \textcolor{blue}{Done.}

- Line 215. Use "... $a_0$ term dominates the shape ...".
      \textcolor{blue}{Done}

- Line 220. Use "Fig. 4".
      \textcolor{blue}{We prefer to keep it as in the caption.}

- Line 222. Use "... integrated over the full ...".
      \textcolor{blue}{Done}

- Line 223. Use "... integrated over the full  ... integrated over the full 
  ...".
      \textcolor{blue}{Done}

- Line 225. Use "sin" and "cos" should be in rm font. In latex use "$\sin$" and 
  "$\cos$" in math mode.
      \textcolor{blue}{Done}

- Line 230. Use "statistical uncertainties".
      \textcolor{blue}{Done}

- Line 231. Use "Fig. 5".
      \textcolor{blue}{We prefer to keep it as in the caption.}

- Line 233. Use "Fig. 4".
      \textcolor{blue}{We prefer to keep it as in the caption.}

- Line 234. Use "... does not show a strong ...".
      \textcolor{blue}{Done}

- Line 236. Use "... are compared to the theoretical ...".
      \textcolor{blue}{Done}

- Line 238. Use "Their model used a nuclear spectral function and considers 
   mainly off-shell effects." This is a statement of jargon that is not 
   appropriate for the general PRL audience.  Please reconsider.
      \textcolor{red}{We kept it as it is.}

- Line 243. Use "... effects that are not taken ...".
      \textcolor{blue}{Done}

- Line 245. Use "... knocked-out proton ...".
      \textcolor{blue}{Done}

- Line 246. Use "On the graph for the $-t$ ...".
      \textcolor{blue}{Done}

- Line 247. Use "HERMES Collaboration".
      \textcolor{blue}{Done}

- Line 254. Use "... we constructed ...".
      \textcolor{blue}{Done}

- Fig. 4. The labels are all too small on this plot (axis labels, titles, plot 
  kinematic labels)
      \textcolor{red}{Will be revisited.}

- Fig. 4 caption.\\
   a- Line 2. Use "(top row)".
      \textcolor{blue}{Done}\\
   b- Line 3. Use "(middle row)" and "(bottom row)".
      \textcolor{blue}{Done}\\
   c- Line 7. "sin" and "cos" should be in rm font. In latex use "$\sin$" and 
   "$\cos$" in math mode.
      \textcolor{blue}{Done}

- Line 255. Use "Fig. 6".
      \textcolor{blue}{We prefer to keep it as in the caption.}

- Line 259. Use "... lower asymmetries that are independent of ...".
      \textcolor{blue}{Done}

- Line 262. Use "medium-modified".
      \textcolor{blue}{Done}

- Line 268. Use "... will be important to differentiate ...".
      \textcolor{blue}{Done}

- Line 273. Use "Our results are compared ...".
     \textcolor{blue}{Done}
~\\
\item Page 5:\\
 - Fig. 5 caption.\\
   - Line 5. Use "On the middle plot the curves ...".
     \textcolor{blue}{Done}\\
   - Line 6. Use "On the bottom plot the solid ...".
     \textcolor{blue}{Done}\\
   - Line 8. It is not clear what you mean by "enriched region".  
     \textcolor{blue}{They were not detecting the final state hadron and the 
     were separating the coherent from the incoherent DVCS events based on the 
     assumption that the coherent dominates the low $t$ region and the 
     incoherent dominate the high $t$ region, and they avoided calling it pure 
     coherent or incoherent by labeling them as coherent or incoherent enriched 
     regions. }\\
   - Line 8. Use "... [30]; the curves represent ...".
     \textcolor{blue}{Done}
~\\
References:\\
 - End the following references with a period: [2], [28], [29], [30], [31], 
   [32], [33], [34], [38].
     \textcolor{blue}{Done. All the references have been double checked for 
     this.}

 - Ref. [40] is not a CLAS Collaboration paper.
     \textcolor{blue}{Cleaned}

 - Refs. [42] and [43] are not cited in the paper.
     \textcolor{blue}{Cleaned.}
   
  
\end{enumerate}



\section{Comments from Michel Gar\c{c}on}


\begin{enumerate}

\item Since HERMES already measured DVCS BSA on "coherent enriched" and "
   incoherent enriched" event samples off He4, one has to be more careful i) in 
      the use of ''first'' in the abstract and on lines 62 and 270, ii) on the 
      way to quote or qualify ref. [30]. In any case, the way it is done now is 
      not OK if only because Figs 5 and 6 show that there were previous 
      measurements, and this is contradictory with the use of "first". My 
      suggestion would be to keep the word "first" (at the risk of having 
      problems with a referee), except maybe in line 62, to remove the ref 30 
      in line 49 and to add at the end of the paragraph in line 61 the 
      sentence: "The previous (or pioneering?) measurements of DVCS off nuclei, 
      and in particular off 4He, performed at HERMES [30] yielded results with 
      both "coherent enriched" and "incoherent enriched" event samples, hence 
      not fully exclusive, but significant enough to be compared with our 
      results below".
\textcolor{blue}{We took your suggestion. Modified.}

   \item Abstract line 3: "within" or "close to the center of" instead of "in 
      front of".
\textcolor{blue}{in front of -> within.}

   \item Abstract next to last line: "for studying the structure of bound 
      nucleons".
\textcolor{blue}{Done.}

   \item Line 50: happens -> occurs.
\textcolor{blue}{Done.}

   \item Fig. 1 caption: is not it -2ksi, and not 2ksi ?
\textcolor{blue}{Done.}

   \item Line 158: ... minimum kinematically allowed ...
\textcolor{blue}{Done.}

   \item Line 188-192: one wonders what is done with this 6.5\%. If nothing is 
      done, should be stated explicitly and/or justified
\textcolor{blue}{We have corrected our asymmetries for these accidentals.
      }\\
       \textcolor{blue}{This sentence has been added to the paper: Our final 
       asymmetries have been corrected for this contamination in the form
      $A_{LU~corrected}~=~\frac{1}{1-contamination~percentage}~A_{LU~measured}$ 
      }

   \item Figs 2, 3 and 4: tick labels too small.
\textcolor{red}{will be revisited.}

   \item Line 277: I would drop "exciting"; "novel" is enough.
\textcolor{blue}{Done.}

   \item Ref. [3]: why not quote the latest 2018 issue? It also contains a 
      section on DIS.
\textcolor{blue}{Done.}

   \item Ref. [20]: this work was superseded by PRC92, 055202 (2015), a 
      reanalysis of the same data. I would either replace Munoz Camacho et al.  
      by Defurne et al, or add to CMC: updated in ....
      \textcolor{blue}{Replaced by: M. Defurne et al. (Jefferson Lab Hall A 
      Collaboration). Phys. Rev. C 92, 055202 (2015).}

   \item Ref[38]: add B+M, PRD82, 0704010 (2010).
\textcolor{blue}{Added.}

   \item Ref.[39]: why not give the publicly accessible link to the proposal?
   \textcolor{blue}{Added.}
  
\end{enumerate}



\section{Comments from David Irland}


\begin{enumerate}

\item Page 2, Equation 3: I don't see the definition of epsilon (or is it a 
   curly E?)
\textcolor{blue}{It is the Compton form factor associated with the GPD E.}

\item Page 3, line 154: Is a lower limit of 2 GeV enough to avoid the nucleon 
   resonance region? There are several known resonances in the range between 2 
      and 3 GeV. Perhaps you can argue that the effect of these is negligible.
\textcolor{blue}{Right, all the free nucleon DVCS analysis have made this 
      assumption.}

\item Page 3, line 159-160: the equations are quite hard to follow, 
   particularly with the braces. It may look better as a displaymath equation.
\textcolor{blue}{Tested. It is not looking much better.}

\item Page 4, figure 4: the font size is way too small!  \textcolor{red}{Will 
   be revisited.}
  
\end{enumerate}



\end{document}
