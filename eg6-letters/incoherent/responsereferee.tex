\documentclass{article}
\usepackage[utf8]{inputenc}

\title{LZ15969:  Authors' Reply}

\usepackage{natbib}
\usepackage{graphicx}

\begin{document}

\maketitle

\section{Summary of Changes}
Below is a summary list of the changes to the paper and their location in the text.  The following sections contain responses to the referees, where we feel we have addressed the expressed concerns.  Note, we would like to refute the claims regarding the invalidity of the $Q^2$ selection and factorization assumptions, particularly regarding this letter's relationship to a large history of published research in this field.

\begin{itemize}
    \item{The title and abstract were completely rewritten.}
    \item{The summary was updated with future plans including an upcoming experiment.}
    \item{Removed some unnecessary experimental details that are standard to CLAS analyses, and added a corresponding reference, in the right column of page 3.}
    \item{Added a sentence on the relationship between the GPD $H$ and PDFs, after equation 5.}
    \item{Explicitly mentioned the standard nature and justification of the $Q^2>2$ GeV selection, at the beginning of page 4.}
    \item{Motivation for the $t>t_{min}$ selection was clarified, in the left column of page 4.}
    \item{Treatment of $\pi^0$ contamination and subtraction was clarified, in the paragraph starting at the end of page 4.}
    \item{The dominance of $a_0$, the Compton Form Factor, and the GPD $H$, was clarified and supported with references, in the left column of page 5.}
    \item{Minor grammar and style updates.}
\end{itemize}

\section{Responses to Referees}

\subsection{Referee A}

{\em The title needs work, to try to grab the attention of the reader.
At present it reads like the title of an internal collaboration
report. Here is one suggestion, "Medium Modification of Nucleon 
Properties, Explored with Deeply Virtual Compton Scattering" but 
I am sure the many senior people in the author list can come up
with something even better.

Similar comments apply to the abstract; the motivation for the
experiment is buried at the end. Bring it to the front. Start the
abstract with something like "Deeply virtual Compton scattering of
electrons from protons not only affords a view into the internal
structure of protons but also provides a method of exploring the
modification of that structure when the proton is bound within a
nucleus." Again, I am sure you can come up with something better than
that.

If you have any plans for additional uses of this technique, it
would be a good idea to mention them in your final summary paragraph.}

\noindent \\
Answer: Following the comments from the Referee A, we revisited the title and abstract completely. We
also made an addition to the summary paragraph to document future related plans, the approved ALERT experiment at JLab.

\subsection{Referee B}

{\em I do not support publication of this manuscript in Physical Review Letters. Both the introduction and experimental aspects are well described, but the article does lack in general accessibility in its science description, and oversimplifies in its physics interpretation. The data are worthwhile to publish and of interest, but I fail to see the portrayed science conclusions at this time.

My criticism related to general accessibility is that no clear meaning or physics insights are attributed to the Compton form factor H, it is simply postulated. Similarly, that the $a_0$ term dominates the shape of the distribution is derived from an earlier publication, but its relevance not explained. An 8 to 10\% contamination of the $\pi^0$ channel is reported, but its impact on the results or systematics is never further elucidated. To interpret some of the figures and textual description, the reader needs to fold in the $t$-dependence. Perhaps more critical, the results and the conclusions drawn are oversimplified and it is unclear if they are correct. First, it is certainly not true that selection of events with $Q^2>1$  GeV$^2$ suffices to ensure the applicability of the DVCS handbag diagram. Even in the inclusive DIS case the argument is still out if $Q^2>1$ GeV$^2$ suffices for a (leading-twist) partonic interpretation. There are no experimental data that cleanly indicate such low $Q^2$ cut would suffice for a pure handbag diagram description. It is further stated that t was required to be greater than a minimum kinematically allowed value, which seems an empty statement without further clarification.  A further serious objection I have is that it seems the authors assumed an Impulse Approximation, and draw their main conclusions from that assumption.

The authors assume a factorization of their measured process as a DVCS process on a single quasi-free proton and a nuclear spectral function, and then compare with an inclusive EMC effect and with a QMC calculation, and conclude that they find medium-modified GPDs beyond these. But there is no one-to-one comparison possible, as the nuclear ratio they measure is (at best!) a convolution of electron-quark scattering with the handbag mechanism, embedded in a nuclear medium, at $Q^2 > 1 $ GeV$^2$. It is certainly not a given that this at these scales is separable in a quasi-free DVCS-like process convoluted with a nuclear spectral function. The conclusion that the measured BSA is quenched and that nuclear effects exist is fair, but to link those at such low-$Q^2$ scales to possible medium modifications of the bound-proton’s partonic structure or, separately, to a new avenue to explore the origin of the EMC effect is a stretch. It reminds of initial $(e,e’p)$ experiments where quenching was also postulated to be related to bound-nucleon EMC-like effects. It is just too early to draw such conclusions, even more if it is not clear if one can interpret the data in this low-$Q^2$ region with a leading-twist handbag mechanism, and if this changes in the nuclear medium.}

\noindent \\
Answer:

We clarified the letter to make it clearer to the general reader where it is possible without adding too much material. Particularly, the physics behind the H GPD is clarified, the reasoning behind the minimum t cut, as well as the $\pi^0$ contamination and how we accounted for it (i.e. it is measured and subtracted at each kinematic). It is unclear what the referee is pointing at when mentioning that we assume impulse approximation for our conclusion. We do not draw any strong conclusion in favor of a particular interpretation of the result, beyond the fact that the result is at odds with existing models. Actually, as would be expected from any surprising first measurement, we point that this result should lead to further exploration of the question both on the theoretical and experimental levels. 

We strongly disagree with the suggestion that the cut $Q^2 > 1$ GeV$^2$ is unjustified and improper. While some of the criticisms raised by the referee are sound, they may be forgetting that a significant aspect of this work involves comparing our results to previous experimental DVCS results from the past decade. Those results have systematically used such a cut, and a different choice for us would defeat the purpose of our comparison. Moreover, this cut is widely used in previously published results cited in the letter that are all consistent with the handbag diagram description: PRL 87, 182001; 97, 072002; 100, 162002; 114, 032001; 115, 212003; 119, 202004 just to quote PRL references. Such extended literature strongly establish this cut as the standard of the field, we fail to see how this letter could possibly be a reasonable place to revisit this standard. 

About the publication in PRL, we still feel strongly that such a large deviation (20 to 40\%) from the predictions of theoretical models, whatever its source, is worth communicating to a large audience of physicists, thus our choice of PRL to publish this work.

\end{document}
