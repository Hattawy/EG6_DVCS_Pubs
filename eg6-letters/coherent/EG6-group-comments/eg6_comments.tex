
%\documentclass[a4paper,11pt,twoside]{ThesisStyle}
\documentclass[a4paper,11pt,twoside]{article}




\date{\today}
\usepackage{amsmath,amssymb}             % AMS Math
% \usepackage[french]{babel}
\usepackage[latin1]{inputenc}
\usepackage[OT1]{fontenc}
\usepackage[left=2.7cm,right=1.7cm,top=1.6cm,bottom=1.6cm,includefoot,includehead,headheight=13.6pt]{geometry}
\usepackage{setspace}
\usepackage{epigraph}
\usepackage{lineno}


%\usepackage{arev}
%\usepackage[bitstream-charter]{mathdesign}
%\usepackage[urw-garamond]{mathdesign}
%\usepackage[sfmath]{kpfonts} %% sfmath option only to make math in sans serif. Probablye only for use when base font is sans serif.
%\renewcommand*\familydefault{\sfdefault} %% Only if the base font of the document is to be sans serif
\usepackage[sc]{mathpazo}
\linespread{1.05}   
\usepackage[T1]{fontenc}



% Table of contents for each chapter

\usepackage[nottoc, notlof, notlot]{tocbibind}
\usepackage{minitoc}
\setcounter{minitocdepth}{2}
\mtcindent=15pt
% Use \minitoc where to put a table of contents

\usepackage{aecompl}

% Glossary / list of abbreviations

\usepackage[intoc]{nomencl}
\renewcommand{\nomname}{List of Abbreviations}

\makenomenclature

% My pdf code

\usepackage{graphicx,type1cm,eso-pic,color}
\usepackage{lscape}

  \usepackage[pagebackref,hyperindex=true]{hyperref}

%\geometry{letterpaper}
%\graphicspath{{.}{images/}}

% nicer backref links
\renewcommand*{\backref}[1]{}
\renewcommand*{\backrefalt}[4]{%
\ifcase #1 %
(Not cited.)%
\or
(Cited on page~#2.)%
\else
(Cited on pages~#2.)%
\fi}
\renewcommand*{\backrefsep}{, }
\renewcommand*{\backreftwosep}{ and~}
\renewcommand*{\backreflastsep}{ and~}

% Links in pdf
\usepackage{color}
\definecolor{linkcol}{rgb}{0,0,0.4} 
\definecolor{citecol}{rgb}{0.5,0,0} 

% Change this to change the informations included in the pdf file

% See hyperref documentation for information on those parameters

\hypersetup
{
bookmarksopen=true,
pdftitle="",
pdfauthor="", pdfsubject="", %subject of the document
%pdftoolbar=false, % toolbar hidden
pdfmenubar=true, %menubar shown
pdfhighlight=/O, %effect of clicking on a link
colorlinks=true, %couleurs sur les liens hypertextes
pdfpagemode=None, %aucun mode de page
pdfpagelayout=SinglePage, %ouverture en simple page
pdffitwindow=true, %pages ouvertes entierement dans toute la fenetre
linkcolor=linkcol, %couleur des liens hypertextes internes
citecolor=citecol, %couleur des liens pour les citations
urlcolor=linkcol %couleur des liens pour les url
}

% definitions.
% -------------------

\setcounter{secnumdepth}{3}
\setcounter{tocdepth}{2}

% Some useful commands and shortcut for maths:  partial derivative and stuff
\newcommand{\xbp}{$x_{Bj}$}
\newcommand{\xb}{$x_{Bj}~$}
\newcommand{\ptp}{$p_\perp^2$}
\newcommand{\pt}{$p_\perp^2~$}
\newcommand{\dptp}{$\Delta \langle p_\perp^2 \rangle$}
\newcommand{\dpt}{$\Delta \langle p_\perp^2 \rangle ~$}

\brokenpenalty10000\relax

\newcommand{\pd}[2]{\frac{\partial #1}{\partial #2}}
\def\abs{\operatorname{abs}}
\def\argmax{\operatornamewithlimits{arg\,max}}
\def\argmin{\operatornamewithlimits{arg\,min}}
\def\diag{\operatorname{Diag}}
\newcommand{\eqRef}[1]{(\ref{#1})}

\usepackage{rotating}                    % Sideways of figures & tables
%\usepackage{bibunits}
%\usepackage[sectionbib]{chapterbib}          % Cross-reference package (Natural BiB)
%\usepackage{natbib}                  % Put References at the end of each chapter
                                         % Do not put 'sectionbib' option here.
                                         % Sectionbib option in 'natbib' will do.
\usepackage{fancyhdr}                    % Fancy Header and Footer

% \usepackage{txfonts}                     % Public Times New Roman text & math font
  
%%% Fancy Header %%%%%%%%%%%%%%%%%%%%%%%%%%%%%%%%%%%%%%%%%%%%%%%%%%%%%%%%%%%%%%%%%%
% Fancy Header Style Options

\pagestyle{fancy}                       % Sets fancy header and footer
\fancyfoot{}                            % Delete current footer settings

%\renewcommand{\chaptermark}[1]{         % Lower Case Chapter marker style
%  \markboth{\chaptername\ \thechapter.\ #1}}{}} %

%\renewcommand{\sectionmark}[1]{         % Lower case Section marker style
%  \markright{\thesection.\ #1}}         %

\fancyhead[LE,RO]{\bfseries\thepage}    % Page number (boldface) in left on even
% pages and right on odd pages
\fancyhead[RE]{\bfseries\nouppercase{\leftmark}}      % Chapter in the right on even pages
\fancyhead[LO]{\bfseries\nouppercase{\rightmark}}     % Section in the left on odd pages

\let\headruleORIG\headrule
\renewcommand{\headrule}{\color{black} \headruleORIG}
\renewcommand{\headrulewidth}{1.0pt}
\usepackage{colortbl}
\arrayrulecolor{black}

\fancypagestyle{plain}{
  \fancyhead{}
  \fancyfoot{}
  \renewcommand{\headrulewidth}{0pt}
}

%\usepackage{algorithm}
%\usepackage[noend]{algorithmic}

%%% Clear Header %%%%%%%%%%%%%%%%%%%%%%%%%%%%%%%%%%%%%%%%%%%%%%%%%%%%%%%%%%%%%%%%%%
% Clear Header Style on the Last Empty Odd pages
\makeatletter

\def\cleardoublepage{\clearpage\if@twoside \ifodd\c@page\else%
  \hbox{}%
  \thispagestyle{empty}%              % Empty header styles
  \newpage%
  \if@twocolumn\hbox{}\newpage\fi\fi\fi}

\makeatother
 
%%%%%%%%%%%%%%%%%%%%%%%%%%%%%%%%%%%%%%%%%%%%%%%%%%%%%%%%%%%%%%%%%%%%%%%%%%%%%%% 
% Prints your review date and 'Draft Version' (From Josullvn, CS, CMU)
\newcommand{\reviewtimetoday}[2]{\special{!userdict begin
    /bop-hook{gsave 20 710 translate 45 rotate 0.8 setgray
      /Times-Roman findfont 12 scalefont setfont 0 0   moveto (#1) show
      0 -12 moveto (#2) show grestore}def end}}
% You can turn on or off this option.
% \reviewtimetoday{\today}{Draft Version}
%%%%%%%%%%%%%%%%%%%%%%%%%%%%%%%%%%%%%%%%%%%%%%%%%%%%%%%%%%%%%%%%%%%%%%%%%%%%%%% 

\newenvironment{maxime}[1]
{
\vspace*{0cm}
\hfill
\begin{minipage}{0.5\textwidth}%
%\rule[0.5ex]{\textwidth}{0.1mm}\\%
\hrulefill $\:$ {\bf #1}\\
%\vspace*{-0.25cm}
\it 
}%
{%

\hrulefill
\vspace*{0.5cm}%
\end{minipage}
}

\let\minitocORIG\minitoc
\renewcommand{\minitoc}{\minitocORIG \vspace{1.5em}}

\usepackage{multirow}
%\usepackage{slashbox}

\newenvironment{bulletList}%
{ \begin{list}%
	{$\bullet$}%
	{\setlength{\labelwidth}{25pt}%
	 \setlength{\leftmargin}{30pt}%
	 \setlength{\itemsep}{\parsep}}}%
{ \end{list} }

\newtheorem{definition}{D�finition}
\renewcommand{\epsilon}{\varepsilon}

% centered page environment

\newenvironment{vcenterpage}
{\newpage\vspace*{\fill}\thispagestyle{empty}\renewcommand{\headrulewidth}{0pt}}
{\vspace*{\fill}}


\begin{document}


\section{Comments from Whitney}


\begin{enumerate}
\item In general I think there are too many figures for a PRL.
Figures 1-5 can probably be removed or reduced.\\
\textcolor{blue}{I kept all the figures for now, but thinking in removing 
figure 2, the RTPC. }

\item "Deeply Virtual Compton Scattering" -> "deeply virtual Compton
scattering (DVCS)"\\
\textcolor{blue}{Done. }
  
\item "Radial Time Projection Chamber" -> "radial time projection chamber
(RTPC)"\\
\textcolor{blue}{Done. }
  
\item "Beam Spin Asymmetries" -> "beam spin asymmetries"\\
\textcolor{blue}{Done. }
  
\item "Compton Form Factors" -> "Compton form factors"\\
\textcolor{blue}{Done. }
  
\item "Generalized Parton Distribution" -> "generalized parton distribution"\\
\textcolor{blue}{Done. }
  
\item Line 17: "Quantum Chromodynamics (QCD)" -> "quantum chromodynamics
(QCD)"\\
\textcolor{blue}{Done. }
  
\item Line 19: "development of the Generalized Parton Distributions (GPDs)
..." -> "development of the generalized parton distribution (GPD) ...",
assuming the acronym isn't already defined in abstract.\\
\textcolor{blue}{Done. }
  
\item Line 23: Unnecessary comma.\\
\textcolor{blue}{Done. }
  
\item Line 23: Last sentence in paragraph could be improved. It is at a
critical location in the paper so you might want to reword for clarity.\\
  
 "In impact parameter space, the GPDs are indeed interpreted as a tomography of 
 the transverse plane for partons carrying a certain longitudinal momentum"\\

Perhaps ...\\

"Impact parameter GPDs provide a tomographic image of the partons
carrying fixed values of longitudinal momentum."\\
\textcolor{blue}{We would like to keep the sentence as it is. }
  
\item Line 28: "Deeply Virtual Compton Scattering" -> "deeply virtual Compton
scattering" and remove "(DVCS)" if acronym defined  in  abstract.\\
\textcolor{blue}{Done.}
  
\item Line 29: ", i.e. the " -> ", \textit{i.e.}, the"\\
\textcolor{blue}{Done. }
  
\item Line 34: Technically "JLab" isn't an acronym since it has lower case
letters -- it is just another name. Furthermore, if you are going to
define the JLab acronym anyway, shouldn't you do the same for CERN and
HERA? I suggest just leaving it as "JLab" or "Jefferson Lab".\\
\textcolor{blue}{Removed JLab and keep it as Jefferson Lab. }
  
\item Line 48: "Figure 1" -> "FIG. 1"\\
\textcolor{blue}{Done.}
  
\item Line 50: Sentence is hard to read. Maybe define the invariants $Q^2$ and t
in one sentence. Then describe the kinematic regime for factorization.\\
\textcolor{blue}{Left as it is for now. }
  
\item Line 59: Oxford comma?  "x, $\xi$, and t"\\
\textcolor{blue}{Done. }
  
\item Line 59: "Figure 1" -> "FIG. 1"\\
\textcolor{blue}{Done. }
  
\item Line 61: " with M the proton mass ..." ->
", $\nu = k^0-k^{\prime0}$, and M is the proton mass." Note that t is
already defined defined (see Line 50 comment).\\
\textcolor{blue}{Cleaned and removed $t$ from here.}
  
\item Line 65: "Compton Form Factors" -> "Compton form factors"\\
\textcolor{blue}{Done. }
  
\item Line 66: "... defined as" -> "... defined at leading order as"
The coefficient function is the LO one and it should be noted.\\
\textcolor{blue}{Done.}
  
\item Line 69: Paragraph needs improved. Perhaps get to the point quickly ...
"The HERMES experiment measured this process with a few nuclear targets
(N, Ne, KR, and XE), however, they did not measure the recoil nucleus.
Therefore, the coherence of the reaction possibly suffers from large
contaminations ..."\\
\textcolor{blue}{Kept the same for now. }
  
\item Line 79: Shouldn't "CLAS" have already been defined in the abstract?\\
\textcolor{blue}{ We define it here.}
  
\item Line 83: Is this the first definition of RTPC? (But note the correct
capitalization of "radial time projection chamber")\\
\textcolor{blue}{ Yes and corrected for the capitalization.}
  
\item Line 85:  The sentence beginning here seems out of place. "... while it
is subject to significant nuclear effects". Is this accidental mixing
with the incoherent paper?\\
\textcolor{blue}{I don't see why it is out of place.}
  
\item Line 100: "at energy of " -> "at an energy of"\\
\textcolor{blue}{Done.}
  
\item Line 115: "In Figure 2 a picture of RTPC installed in the experimental
hall, and the rendering showing detector components and the 4 He
detection concept are presented." ->
"Presented in FIG. 2 are a picture of RTPC installed in the experimental
hall and a diagram showing the basic detection technique."\\
\textcolor{blue}{the sentence is cleaned. }
  
\item Line 134: "The photons are detected in either the IC or
the CLAS electromagnetic calorimeter." Is this true?\\
\textcolor{blue}{Yes for the case of the ICEC pio topology needed for the 
background subtraction. }
  

\item Line 146: "events with ... " -> "events with $Q^2$ > 1 GeV$^2$/c$^2$."\\
\textcolor{blue}{Extra spaces? yes done. }
  
\item Line 160: "After these requirements, we ... " -> "About 3200 events pass
these requirements and are shown in FIG. 4..."\\
\textcolor{blue}{Edited.}
  
\end{enumerate}


\section{Comments from Hovanes}

\begin{enumerate}
\item In the abstract there is a statement about the BSA size relative to
the BSA on the proton, but there
is not such statement or comparison in the article (I might have missed
it). I would suggest either removing
this statement from the abstract or add graph(s) comparing the two BSAs
to support such a statement.\\
\textcolor{blue}{What about adding the t-dependence plot of the ALU ratio? }
  
\item line 36 : "allowed extraction of the tomography of the nucleon" could
be changed "allowed for extraction
of the three-dimensional picture of the nucleon" . The way "tomography"
is used in lines 24 and 37 is a little
unusual.\\
\textcolor{blue}{Cleaned. }
  
\item line 49 : "hand bag diagram" -> "hand-bag diagram"\\
\textcolor{blue}{Done here and in the caption.}
  
\item line 78 : I suggest removing "and that the reaction did not occur on
a bound nucleon".\\
\textcolor{blue}{Removed. }
  
\item line 80 : CLAS was not designed for study of DVCS, therefore the
statement that it is optimized for DVCS
is not a justified statement. One can simply say " ... in Hall-B at
Jefferson Lab has been previously used for DVCS
measurements on nucleon."\\
\textcolor{blue}{Cleaned.}
  
\item Line 194 : "and a solenoid." -> "and a solenoid magnet."\\
\textcolor{blue}{Added. }
  
\item Line 198 "... 5 Tesla solenoid ..." -> "... 5 Tesla solenoid magnet
... "\\
\textcolor{blue}{Added. }
  
\item Figure 3 : The histograms titles need to be removed from the top of
the graphs and put as x-axis titles.\\
\textcolor{blue}{I will update them.}
  
\item Line 196 " ... that depends ..." -> "... that depend ..." .\\
\textcolor{blue}{Corrected. }
  
\item Line 152 and line 198 : The font for 'He4 changes throughout the paper.\\
\textcolor{blue}{Corrected.}
  
\item Figures 3, 5, 6, 7 : axis titles labels and legends are totally
illegible. At some point these will have to be improved.
\textcolor{blue}{I will work on them.}
  
\item Line 271. I am not sure how the results in Figure 7 support the
argument that the extraction of CFF is
model independent. The applicability of Eq. 5 presumably determines if
this extraction is model independent or not,
either way the fits would produce some results.
\textcolor{blue}{ }
  
\end{enumerate}





\section{Comments from Eric}
I will concur with other remarks that you may try to concentrate on a reduced 
set of figures. It is always difficult to squeeze everything in the PRL and I 
apologize in advance for my next remark asking to potentially say more. It 
would be valuable to have a direct comparison between proton and He4 on the 
basis of the amplitude of the asymmetries at 90°. The ratio He4/proton would 
give the so-called generalized EMC ratio that already encode some nuclear 
effects. It would be interesting to know whether or not this comparison yields 
conclusive physics input.
\textcolor{blue}{We may remove Figure 2 and add a t-dependence for the ratio! }
  
\section{Comments from Lamiaa}

\begin{enumerate}

\item Would it possible to define some acronyms in the abstract especially the ones that will be used again there like DVCS (some published PRLs did, e.g. the latest CLAS PRL of 2015, they used CLAS acronym w/o. definition (!), defined GPDs acronym in the abstract to avoid writing generalized parton distributions twice, and all others including JLab were defined in the core. 

I believe according to the abbreviation guidelines below, you could either define the acronym DVCS using deeply virtual Compton scattering or Deeply Virtual Compton Scattering but e.g. the above letter used the former!
\textcolor{blue}{Thanks! }
  
\item In Eq. 1 $\&$ 2 you gave the expression of the real and comlplex CFF amplitudes, hence, shouldn't be "real and complex amplitudes" in line 66?
\textcolor{blue}{Edited.}
  
\item While the PRL link below contains some comments about the letter length, number of words, and figures size which It will be nice to review, I have a comment about some figures:
   a. To gain some space you could move a figure's caption a little bit up using:
   

         vglue -xxxcm caption{}               ! e.g. vglue -0.5cm

  
   b. Some figures are more shifted to the left, e.g. Fig. 3, Fig. 5, 6 $\&$ 7, if centering under\\

       begin{figure*}\\
       centering \\
         includegraphics[]{}\\
          vglue -0.55cm caption{}\\
          label\\
         end{figure*}\\

       is not doing the job. You could force it by adding vspace*{-xxxcm} just after begin{figure*}.
       (a simple trick that works!)
\textcolor{blue}{I will work on them. }
  
\item It will be nice to rephrase some sentences to read better such as:
    a. Line 207, "The explicit expressions of these terms can be found in [40] 
    and show that, by using the sin( $\phi$) and cos( $\phi$) contributions, it 
    is possible to extract Im(HA) and Re(HA) from the beam spin asymmetry."\\
    
    b.  Line 236, "However, added quadratically,the total systematic 
    uncertainty is about 10\%, which is significantly smaller than statistical 
    uncertainties in all kinematical bins."\\
    
    c. Line 278, "While the accuracy of our results does not allow to 
    discriminate between the models, they demonstrate possibility of extraction 
    of the CFF of spin 0 target in a model independent way."\\
    \textcolor{blue}{ }
  
\item Line 256, $Q^2$, $x_B$, and t dependencies.........
\textcolor{blue}{Done. }
  
\end{enumerate}


\end{document}
