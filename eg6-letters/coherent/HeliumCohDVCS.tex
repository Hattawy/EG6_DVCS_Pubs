\documentclass[%
 reprint,
%superscriptaddress,
%groupedaddress,
%unsortedaddress,
%runinaddress,
%frontmatterverbose, 
%preprint,
%showpacs,preprintnumbers,
%nofootinbib,
%nobibnotes,
%bibnotes,
 amsmath,amssymb,
 aps,
%pra,
%prb,
%rmp,
%prstab,
%prstper,
%floatfix,
]{revtex4-1}

\usepackage{graphicx}% Include figure files
\usepackage{dcolumn}% Align table columns on decimal point
\usepackage{bm}% bold math

\begin{document}

\title{First Exclusive Measurement of Deep Virtual Compton Scattering off $^4$He: Toward the 3D tomography of nuclei}

\author{M. Hattawy, ...}
\affiliation{...}
\author{R. Dupr\'e}
\affiliation{Institut de Physique Nucl\'eaire d'Orsay, CNRS-IN2P3,\\ 
Universit\'e Paris-Sud, Universit\'e Paris-Saclay, 91406 Orsay, France.}

\date{\today}

\begin{abstract}
We report the first exclusive measurement of deeply virtual Compton scattering 
(DVCS) off a nuclei with the CLAS detector at the Jefferson laboratory. The 
observed beam spin asymmetries are significantly larger than for proton, 
confirming expectations from calculations using the plane wave impulse 
approximation. We present an 
extraction of the single generalized parton distribution (GPD) of the spin-0 
helium-4 nuclei in a completely model independent way. This result demonstrate the 
strength of the method and opens a new and unique access into the partonic structure 
of nuclei.
\end{abstract}

\pacs{Valid PACS appear here}

\maketitle

%\section{Introduction}
% Now many letters do not put any sections. Maybe we should consider doing the same

The generalized parton distribution (GPD) framework developped in the past 
two decades~\cite{}
offers a unique access into hadron 3D structure through deeply virtual Compton
scattering (DVCS), depicted in figure~\ref{fig:DVCS}, and other exclusive 
processes. Many measurements on the proton
at several energies have now be reported~\cite{} and phenomenological work
have demonstrated the solidity of the method to extract the tomography of the
nucleon from these data~\cite{}. 

\begin{figure}[htbp]
\caption{\label{fig:DVCS} Diagram for the coherent DVCS in the plane wave approximation.}
\includegraphics[width=8.6cm]{DVCS.png}
\end{figure}

At the same time, the partonic structure of nuclei has remained a mystery
for the past three decades, as we are still unable to explain properly the 
infamous EMC effect~\cite{}. First observed in 197X by the EMC 
collaboration, it describes the difference between the partonic structure of
free and bound nucleons. Understanding the source of this difference remains 
today a topic of active research~\cite{},
questionning the basics of our understanding of nuclei and their 
structure~\cite{}. Studies based on inclusive measurements of the
effect provide a very detailed description~\cite{} but 
are still unable to tackle the question definitively. Therefore,
using the GPD framework and exclusive channels to study the nuclei in three 
dimensions will give completely new information that will affect our understanding
of the EMC effect in an unprecedented way~\cite{}. 
%In particular, the GPDs 
%allow to isolate the components of the nuclei which are not nucleonic~\cite{}. 

The measurement of DVCS on helium-4 is in this regard the perfect experiment for
several reasons. First, Helium 4 is of spin-0, such that a single GPD ($H$) is
necessary for the description of the DVCS at leading order. This feature allows for a
completely model independant extraction of the CFF from data that is impossible for 
protons and neutrons which have 4 GPDs~\cite{}. Second, 
helium 4 is dense and experience 
already a significant EMC effect, while being simple enough that its nuclear
structure is well known~\cite{}. Third, being 
light enough helium-4 can exit
a light target and be experimentally measured in a coherent DVCS process, 
insuring the exclusivity of the 
measurement. This point is especially important because we are probing with a 
multi-GeV interaction a nuclei bound by only a few MeV, this leads in the 
overwhelming majority of the cases to the incoherent break-up of the target.

%\section{Experiment}
The HERMES collaboration has measured nuclear DVCS~\cite{} without the 
detection of the recoils separating the coherent and incoherent channels 
based on the measurement of the transfered 4-momentum $t$, leading
to controversy in the interpretation of their results~\cite{}. We report
here our exclusive measurement of coherent DVCS off helium-4 at the Jefferson labortory
(JLab), while our results for the incoherent channel will be presented in
another paper. JLab delivers a nearly 100\% duty factor, 6~GeV linearly 
polarized electrons into three experimental halls. Our experiment ran for 3 
months in 2009 using the CEBAF Large Acceptance Spectrometer (CLAS) in 
Hall-B. This detector is large acceptance covering nearly 2$\pi$ and is 
composed of drift chambers, Cherenkov counters, scintillator counters and 
an electro-magnetic calorimeter. It allows to detect accurately electrons 
which provide the trigger for the experiment. It is complemented for DVCS 
experiments with an inner electro-magnetic calorimeter made of PbWO crystals 
to detect the photons emitted at low angle and with a solenoid to keep the 
Moller electrons produced along the beam line to get into our detectors.

\begin{figure*}[tbp]
\caption{\label{fig:exclu} Exclusivity cuts}
\includegraphics[width=17.2cm]{all_coh_exc_cuts.png}
\end{figure*}

In order to insure the coherence of the process, we also detect the recoiling 
helium-4 nuclei. We built a small and light radial time projection chamber 
(RTPC) to detect recoiling helium nuclei down to energies of few MeVs. The 
chamber, shown in figure~\ref{fig:rtpc}, is surrounding the gaseous target 
filled with 6~atm helium-4 and is 20 cm long with a 3 cm radial drift 
length. The detector was specifically calibrated for helium-4 nuclei using 
elastic scattering produced with a 1.2~GeV electron beam.

\begin{figure}[htbp]
\caption{\label{fig:rtpc} RTPC description}
\includegraphics[width=8cm]{RTPC.png}
\end{figure}

%\section{DVCS Events Selection}
Explain exclusivity cuts

We identified several background contributions to the DVCS process, in 
particular accidental events and exclusive deeply virtual $\pi^0$ production (DV$\pi^0$P). The accidental
events happen when an helium nuclei is mistakenly associated with our event, this 
possibility is suppressed by the limited phase space allowed but enhanced by the
low time resolution of the RTPC (~200 ns). We evaluated this contribution by
selecting events passing all our cuts but with particles originating from 
different verteces. With this method, we found this background to represent 
4.1\%. The DV$\pi^0$P events can easily be mistaken
with DVCS events when one of the two photons of $\pi^0$ decay is produced at
low energy in the laboratory frame. In order to estimate the importance of
this background, we developped an event generator DV$\pi^0$P 
that we calibrated to match our experimental yields. We used this generator together 
with a GEANT3~\cite{} simulation of our detection system to estimate the ratio 
of acceptance between measured DV$\pi^0$P and DV$\pi^0$P that would pass our 
DVCS selection cuts. This ratio obtained from simulation is then multiplied by 
the measured yield of DV$\pi^0$P events, indicating a contamination of XX to XX\%. 
%(We NEED a figure of that in the note!!!)

Systematic errors

%\section{Results}

Extraction of assymetries (figure~\ref{fig:assym})

\begin{figure*}[htbp]
\caption{\label{fig:assym} Assymetries}
\includegraphics[width=14cm]{coh_alu_t_phi.png}
\end{figure*}


\begin{figure}[htbp]
\caption{\label{fig:CFF} CFF extraction}
\includegraphics[width=8cm]{HA_t.png}
\end{figure}


Model independant extraction of H (figure~\ref{fig:CFF})


%\section{Summary}

Open new perspectives in nuclear physics, mention the 12 GeV program and the EIC

First fully exclusive measurement: very strong BSA, demonstrate the capability to extract GPDs even with limited statistics and kinematic compared to proton.

% If this is line > 400 this is probably getting too long!

\section*{Acknowledgments}
Guzey and Liuti

\section*{Bibliography}
Biblio

\end{document}

