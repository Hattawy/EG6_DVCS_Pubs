\documentclass[aps,prc,preprint,superscriptaddress]{revtex4-1}

\begin{document}

%Title of paper
\title{Measurement of deeply virtual Compton scattering off helium-4}

% repeat the \author .. \affiliation  etc. as needed
% \email, \thanks, \homepage, \altaffiliation all apply to the current
% author. Explanatory text should go in the []'s, actual e-mail
% address or url should go in the {}'s for \email and \homepage.
% Please use the appropriate macro foreach each type of information

% \affiliation command applies to all authors since the last
% \affiliation command. The \affiliation command should follow the
% other information
% \affiliation can be followed by \email, \homepage, \thanks as well.
\author{CLAS et al.}
%\email[]{Your e-mail address}
%\homepage[]{Your web page}
%\thanks{}
%\altaffiliation{}
\affiliation{JLab et al.}

%Collaboration name if desired (requires use of superscriptaddress
%option in \documentclass). \noaffiliation is required (may also be
%used with the \author command).
%\collaboration can be followed by \email, \homepage, \thanks as well.
\collaboration{CLAS Collaboration}
\noaffiliation

\date{\today}

\begin{abstract}
% insert abstract here
\end{abstract}

% insert suggested PACS numbers in braces on next line
\pacs{}

\maketitle

\section{Introduction}

\section{Theoretical Framework}

  \subsection{The GPD Formalism}

  \subsection{Incoherent Nuclear DVCS}

  \subsection{Coherent Nuclear DVCS}

\section{The E08-024 Experiment}

  \subsection{The CLAS DVCS Setting}

  \subsection{The Radial Time Projection Chamber}

\section{Data Analysis}

  \subsection{Particle Identification}

  \subsection{Monte-Carlo Simulation}

  \subsection{Kinematic Corrections}

  \subsection{Background Reduction}

\section{Incoherent DVCS}

  \subsection{Event Selection}

  \subsection{Systematic Errors}

\section{Coherent DVCS}

  \subsection{Event Selection}

  \subsection{Systematic Errors}

\section{Results}

  \subsection{Incoherent DVCS}

  \subsection{Coherent DVCS}

\section{Summary}

% \begin{figure}
% \includegraphics{}%
% \caption{\label{}}
% \end{figure}

% Surround figure environment with turnpage environment for landscape
% figure
% \begin{turnpage}
% \begin{figure}
% \includegraphics{}%
% \caption{\label{}}
% \end{figure}
% \end{turnpage}

% tables should appear as floats within the text
%
% Here is an example of the general form of a table:
% Fill in the caption in the braces of the \caption{} command. Put the label
% that you will use with \ref{} command in the braces of the \label{} command.
% Insert the column specifiers (l, r, c, d, etc.) in the empty braces of the
% \begin{tabular}{} command.
% The ruledtabular enviroment adds doubled rules to table and sets a
% reasonable default table settings.
% Use the table* environment to get a full-width table in two-column
% Add \usepackage{longtable} and the longtable (or longtable*}
% environment for nicely formatted long tables. Or use the the [H]
% placement option to break a long table (with less control than 
% in longtable).
% \begin{table}%[H] add [H] placement to break table across pages
% \caption{\label{}}
% \begin{ruledtabular}
% \begin{tabular}{}
% Lines of table here ending with \\
% \end{tabular}
% \end{ruledtabular}
% \end{table}


% Specify following sections are appendices. Use \appendix* if there
% only one appendix.
%\appendix
%\section{}

% Surround table environment with turnpage environment for landscape
% table
% \begin{turnpage}
% \begin{table}
% \caption{\label{}}
% \begin{ruledtabular}
% \begin{tabular}{}
% \end{tabular}
% \end{ruledtabular}
% \end{table}
% \end{turnpage}

\begin{acknowledgments}
 Put your acknowledgments here.
\end{acknowledgments}

% Create the reference section using BibTeX:
\bibliography{basename of .bib file}

\end{document}
%
% ****** End of file apstemplate.tex ******

