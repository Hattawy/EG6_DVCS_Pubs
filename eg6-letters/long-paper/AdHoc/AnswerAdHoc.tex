
%\documentclass[a4paper,11pt,twoside]{ThesisStyle}
\documentclass[a4paper,11pt,twoside]{article}

\title{Answer to comments from the Ad-hoc rewiew committee \\
 Measurement of deeply virtual Compton
scattering off Helium-4 with CLAS at Jefferson Lab}


\date{\today}
\usepackage{amsmath,amssymb}             % AMS Math
% \usepackage[french]{babel}
\usepackage[latin1]{inputenc}
\usepackage[OT1]{fontenc}
\usepackage[left=2.7cm,right=1.7cm,top=1.6cm,bottom=1.6cm,includefoot,includehead,headheight=13.6pt]{geometry}
\usepackage{setspace}
\usepackage{epigraph}
\usepackage{lineno}


%\usepackage{arev}
%\usepackage[bitstream-charter]{mathdesign}
%\usepackage[urw-garamond]{mathdesign}
%\usepackage[sfmath]{kpfonts} %% sfmath option only to make math in sans serif. Probablye only for use when base font is sans serif.
%\renewcommand*\familydefault{\sfdefault} %% Only if the base font of the document is to be sans serif
\usepackage[sc]{mathpazo}
\linespread{1.05}   
\usepackage[T1]{fontenc}



% Table of contents for each chapter

\usepackage[nottoc, notlof, notlot]{tocbibind}
\usepackage{minitoc}
\setcounter{minitocdepth}{2}
\mtcindent=15pt
% Use \minitoc where to put a table of contents

\usepackage{aecompl}

% Glossary / list of abbreviations

\usepackage[intoc]{nomencl}
\renewcommand{\nomname}{List of Abbreviations}

\makenomenclature

% My pdf code

\usepackage{graphicx,type1cm,eso-pic,color}
\usepackage{lscape}

  \usepackage[pagebackref,hyperindex=true]{hyperref}

%\geometry{letterpaper}
%\graphicspath{{.}{images/}}

% nicer backref links
\renewcommand*{\backref}[1]{}
\renewcommand*{\backrefalt}[4]{%
\ifcase #1 %
(Not cited.)%
\or
(Cited on page~#2.)%
\else
(Cited on pages~#2.)%
\fi}
\renewcommand*{\backrefsep}{, }
\renewcommand*{\backreftwosep}{ and~}
\renewcommand*{\backreflastsep}{ and~}

% Links in pdf
\usepackage{color}
\definecolor{linkcol}{rgb}{0,0,0.4} 
\definecolor{citecol}{rgb}{0.5,0,0} 

% Change this to change the informations included in the pdf file

% See hyperref documentation for information on those parameters

\hypersetup
{
bookmarksopen=true,
pdftitle="",
pdfauthor="", pdfsubject="", %subject of the document
%pdftoolbar=false, % toolbar hidden
pdfmenubar=true, %menubar shown
pdfhighlight=/O, %effect of clicking on a link
colorlinks=true, %couleurs sur les liens hypertextes
pdfpagemode=None, %aucun mode de page
pdfpagelayout=SinglePage, %ouverture en simple page
pdffitwindow=true, %pages ouvertes entierement dans toute la fenetre
linkcolor=linkcol, %couleur des liens hypertextes internes
citecolor=citecol, %couleur des liens pour les citations
urlcolor=linkcol %couleur des liens pour les url
}

% definitions.
% -------------------

\setcounter{secnumdepth}{3}
\setcounter{tocdepth}{2}

% Some useful commands and shortcut for maths:  partial derivative and stuff
\newcommand{\xbp}{$x_{Bj}$}
\newcommand{\xb}{$x_{Bj}~$}
\newcommand{\ptp}{$p_\perp^2$}
\newcommand{\pt}{$p_\perp^2~$}
\newcommand{\dptp}{$\Delta \langle p_\perp^2 \rangle$}
\newcommand{\dpt}{$\Delta \langle p_\perp^2 \rangle ~$}

\brokenpenalty10000\relax

\newcommand{\pd}[2]{\frac{\partial #1}{\partial #2}}
\def\abs{\operatorname{abs}}
\def\argmax{\operatornamewithlimits{arg\,max}}
\def\argmin{\operatornamewithlimits{arg\,min}}
\def\diag{\operatorname{Diag}}
\newcommand{\eqRef}[1]{(\ref{#1})}

\usepackage{rotating}                    % Sideways of figures & tables
%\usepackage{bibunits}
%\usepackage[sectionbib]{chapterbib}          % Cross-reference package (Natural BiB)
%\usepackage{natbib}                  % Put References at the end of each chapter
                                         % Do not put 'sectionbib' option here.
                                         % Sectionbib option in 'natbib' will do.
\usepackage{fancyhdr}                    % Fancy Header and Footer

% \usepackage{txfonts}                     % Public Times New Roman text & math font
  
%%% Fancy Header %%%%%%%%%%%%%%%%%%%%%%%%%%%%%%%%%%%%%%%%%%%%%%%%%%%%%%%%%%%%%%%%%%
% Fancy Header Style Options

\pagestyle{fancy}                       % Sets fancy header and footer
\fancyfoot{}                            % Delete current footer settings

%\renewcommand{\chaptermark}[1]{         % Lower Case Chapter marker style
%  \markboth{\chaptername\ \thechapter.\ #1}}{}} %

%\renewcommand{\sectionmark}[1]{         % Lower case Section marker style
%  \markright{\thesection.\ #1}}         %

\fancyhead[LE,RO]{\bfseries\thepage}    % Page number (boldface) in left on even
% pages and right on odd pages
\fancyhead[RE]{\bfseries\nouppercase{\leftmark}}      % Chapter in the right on even pages
\fancyhead[LO]{\bfseries\nouppercase{\rightmark}}     % Section in the left on odd pages

\let\headruleORIG\headrule
\renewcommand{\headrule}{\color{black} \headruleORIG}
\renewcommand{\headrulewidth}{1.0pt}
\usepackage{colortbl}
\arrayrulecolor{black}

\fancypagestyle{plain}{
  \fancyhead{}
  \fancyfoot{}
  \renewcommand{\headrulewidth}{0pt}
}

%\usepackage{algorithm}
%\usepackage[noend]{algorithmic}

%%% Clear Header %%%%%%%%%%%%%%%%%%%%%%%%%%%%%%%%%%%%%%%%%%%%%%%%%%%%%%%%%%%%%%%%%%
% Clear Header Style on the Last Empty Odd pages
\makeatletter

\def\cleardoublepage{\clearpage\if@twoside \ifodd\c@page\else%
  \hbox{}%
  \thispagestyle{empty}%              % Empty header styles
  \newpage%
  \if@twocolumn\hbox{}\newpage\fi\fi\fi}

\makeatother
 
%%%%%%%%%%%%%%%%%%%%%%%%%%%%%%%%%%%%%%%%%%%%%%%%%%%%%%%%%%%%%%%%%%%%%%%%%%%%%%% 
% Prints your review date and 'Draft Version' (From Josullvn, CS, CMU)
\newcommand{\reviewtimetoday}[2]{\special{!userdict begin
    /bop-hook{gsave 20 710 translate 45 rotate 0.8 setgray
      /Times-Roman findfont 12 scalefont setfont 0 0   moveto (#1) show
      0 -12 moveto (#2) show grestore}def end}}
% You can turn on or off this option.
% \reviewtimetoday{\today}{Draft Version}
%%%%%%%%%%%%%%%%%%%%%%%%%%%%%%%%%%%%%%%%%%%%%%%%%%%%%%%%%%%%%%%%%%%%%%%%%%%%%%% 

\newenvironment{maxime}[1]
{
\vspace*{0cm}
\hfill
\begin{minipage}{0.5\textwidth}%
%\rule[0.5ex]{\textwidth}{0.1mm}\\%
\hrulefill $\:$ {\bf #1}\\
%\vspace*{-0.25cm}
\it 
}%
{%

\hrulefill
\vspace*{0.5cm}%
\end{minipage}
}

\let\minitocORIG\minitoc
\renewcommand{\minitoc}{\minitocORIG \vspace{1.5em}}

\usepackage{multirow}
%\usepackage{slashbox}

\newenvironment{bulletList}%
{ \begin{list}%
	{$\bullet$}%
	{\setlength{\labelwidth}{25pt}%
	 \setlength{\leftmargin}{30pt}%
	 \setlength{\itemsep}{\parsep}}}%
{ \end{list} }

\newtheorem{definition}{D�finition}
\renewcommand{\epsilon}{\varepsilon}

% centered page environment

\newenvironment{vcenterpage}
{\newpage\vspace*{\fill}\thispagestyle{empty}\renewcommand{\headrulewidth}{0pt}}
{\vspace*{\fill}}



\begin{document}

\maketitle

\section*{}

The committee met on July 2, 2020, to discuss our concerns about the manuscript. We are separating out
our comments on English, wording, typographical errors, etc., as helpful suggestions for you, which do not need
any reponse. However, there are a few areas concern where we would like a response before we can approve the
manuscript. In some cases, we would appreciate a direct answer or some clarifying information. In others, we
recommend changes to the manuscript to make it more clear and easier to read. In any case, please note your
reponse and any manuscript changes in a written document in addition to any discussion between the committee
and the authors by e-mail, so that the response can be preserved in the official record.

\begin{enumerate}
\item  As this is a long paper, giving more details about two published letters, we'd appreciate some statement as
to whether or not anything in the analysis has changed between this paper and the two previous papers? Did
the systematic error decrease? Did the central values of the results move in any case more than 1$\sigma$ of their
former error. Perhaps it would be good to state the changes/improvements achieved in this new publication.

   \textcolor{blue}{There was no change to the results. The statement at the end of the second paragraph
clearly states that we are giving more details and do not mention any change or new analysis.}

\item  We have concerns about how you describe your approach to the $\pi^0$ contamination, especially in the incoherent
channel.
\begin{enumerate}
\item In Figs. 14 and 15, bottom right panels, what is the definition of the $\phi$ angle? Is this the $\phi$ angle of the
$\pi^0$ ?

   \textcolor{blue}{Yes.}

\item What kind of input distributions went into the $\pi^0$ generation to produce the $\phi$-dependence of Figs. 14
and 15.

   \textcolor{blue}{This is given in eq. 3.6 and tab. 3.1 of the analysis note, there is no $\phi$-dependence 
   in the generation. We decided not to go to this level 
   of details in this paper as the parameters have no relevant physics signification.}

\item We feel that it would be more useful to see the $\phi$ distribution calculated (from simulation) from the
single detected photons, i.e., what the $\phi$-distribution of your actual background looks like.

   \textcolor{blue}{The intent of this figure is to compare the simulation to the real data, to assess the 
quality of the simulation. What you propose would not be more useful, it would serve a different purpose:
showing the shape of the background. We do not have a way to show this background from experimental 
data so it would not allow to assess the same information. We provided some of the numbers by email, 
but are not convinced it highlights anything important about the analysis. The article is
already going into a lot of technical details, probably more than most comparable long paper. So we prefer not 
to go to that extend.}

\item In lines 280-282, you write "To make the correction on the DVCS BSA, we assume that the exclusive $\pi^0$
production has no such asymmetry. This has been checked with the exclusive $\pi^0$ production data, for
which no significant level of BSA was measured."
There are couple of issues here. First, this claim is contradicted by CLAS data on protons, for example
https://arxiv.org/pdf/0711.4736.pdf, which observed 5-10\% BSA over a wide kinematic range.
Second, even if you check with your own exclusive $\pi^0$ data and see no BSA, this claim is only so good as
the precision of your check, and there must be some systematic uncertainty associated with the possibility
of a BSA below your ability to detect. This needs to be explained in the paper.

\textcolor{blue}{The results are not contradictory, our target is half proton half neutron and will experience,
at least, some nuclear effects. There is clearly no solid ground to build a correction from the 
existing data. Moreover, our CLAS measurement shows results consistent with 0, so if we were to make a correction
it would be very small. On your second point, there is indeed a systematic error associated with the $\pi^0$ 
substraction as specified in the Tab. 1. We clarified in the text that 
this correction includes a possible undetected asymmetry in the exclusive $\pi^0$ data.}

\item It would be helpful for you to write a few words explaining the differences in the assumed $\pi^0$ distributions
for the coherent and incoherent cases.

   \textcolor{blue}{There are no assumptions on this, we adjusted the parameters to fit the measured 
distributions. We clarify this point in the text of the paper.}

\end{enumerate}

\item  Regarding the final state interaction discussion starting at approximately line 100, the issue is a nucleonic
final state interaction issue, and we are not sure why DIS is mentioned here as a reference. There must be
people who spent their time investigating the A(e, e 0 p)X nucleonic final state calculations, for example, that
might inform us on these effects. Perhaps adding a more relevant reference would help.

   \textcolor{blue}{We added a modern reference about quasi-elastic FSI. We welcome suggestions if you know 
of more relevant publications on the topic.}

\item  In Section 4.1, your choice to use the present tense here is quite jarring, especially since you don't make a
distinction between the original CLAS, the upgraded CLAS-12 detector. Please consider switching to past
tense here and making that distinction explicit. Furthermore, it would be good to state that these data came
from the EG6 run period, and give the year that they were collected.

   \textcolor{blue}{This section has been modified.}

5. Line 147: "Its historic use to measure DVCS in many different configurations made it an ideal place for this
new DVCS measurement."
This is not an informative claim. CLAS can be ideal for DVCS for many reasons, but you're pretty much
giving a tautology here.

   \textcolor{blue}{The information is that CLAS has been used for other DVCS measurements in the past. It seems
to be a relevant information when showing results for DVCS on a new target.}

\item  Section 5.1: "In particular, it serves for the identification of the protons."
Are you referring to the time-of-flight detection of the electron or the proton? Perhaps separate sections for
electron ID and proton ID are warranted. Furthermore, you have no section on the ID of 4 He in the RTPC.
We know that this is covered in the RTPC NIM, but a paragraph would be nice.

   \textcolor{blue}{We made the change and clarified the detection of protons and helium nuclei.}

\item  Line 238: "In principle, a selection based on two or three variables can be used to guaranty the exclusivity of
the process."
In exclusive DVCS, you measure the momentum vectors of 3 particles of known ID, i.e. 9 quantities. Exclu-
sivity requires both energy and momentum conservation, i.e. 4 constraints. Where do "2 or 3" come from?
That is a vague statement.

   \textcolor{blue}{This is incorrect, one constraint is enough, you can ensure exclusivity by only cutting on missing 
energy, if it is 0 the process is exclusive. Due to the effects of resolution, this is usually never a good solution
and 2, 3 or more variables are used, often missing mass and missing momentum for instance. Anyway, we agree the statement was 
vague and not very useful, we changed this sentence completely.}

\item  Line 302: "As it is not obvious which solution is best. . ."
What is obvious is that this is not an issue of correct or incorrect, but one of possible bias introduced to your
BSA from your poor t resolution. Please rephrase the paragraph in these terms, and state what systematic
uncertainty is introduced from the limitations in your ability to reconstruct t. If this effect introduces negligible
uncertainty on the BSA, that's fine, but make that argument clear.

   \textcolor{blue}{We added a sentence about the fact that no error is associated to 
this study.}

\item  Table 1 is mildly misleading, simply because the beam polarization is a relative uncertainty, while the others
are absolute uncertainties (if we understand correctly). Perhaps it is better to pull beam polarization out of
the middle of the table and put it somewhere else so it can be understood as distinct.
That 3.5\% is a multiplicative factor, right?

   \textcolor{blue}{We changed the order of the lines and the presentation in the table to make it clearer.}

\item  Eq. 25 appears to have a mis-print. xA should be $x_A$.

   \textcolor{blue}{Done.}

\item  English and Typographical suggestions.

   \textcolor{blue}{All done.}

\end{enumerate}


\end{document}
