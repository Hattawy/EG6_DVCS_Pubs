\documentclass[nofootinbib,twocolumn,showpacs,prl,superscriptaddress,secnumarabic,amssymb,nobibnotes,aps,floatfix]{revtex4}
\usepackage{graphicx}
\usepackage{hyperref}
\usepackage{amsmath}
\usepackage{dcolumn}% Align table columns on decimal point
\usepackage{bm}% bold math

%
\def\Dirac#1{#1\hskip-5pt/}
%
\begin{document}

\title{First Exclusive Measurement of Deep Virtual Compton Scattering off $^4$He: Toward the 3D tomography of nuclei}

\newcommand*{\ANL}{Argonne National Laboratory, Argonne, Illinois 60439}
\newcommand*{\ANLindex}{1}
\affiliation{\ANL}
\newcommand*{\ORSAY}{Institut de Physique Nucl\'eaire, CNRS/IN2P3 and Universit\'e Paris Sud, Orsay, France}
\newcommand*{\ORSAYindex}{2}
\affiliation{\ORSAY}
\newcommand*{\JLAB}{Thomas Jefferson National Accelerator Facility, Newport News, Virginia 23606}
\newcommand*{\JLABindex}{3}
\affiliation{\JLAB}
 
\newcommand*{\NOWJLAB}{Thomas Jefferson National Accelerator Facility, Newport News, Virginia 23606}
\newcommand*{\NOWODU}{Old Dominion University, Norfolk, Virginia 23529}
 %%%%%%%%%%%%%%% END OF Latex Macros for institute addresses  %%%%%%%%%%%%%%%%%%%%%%%%% 

\author {M.~Hattawy}
\affiliation{\ANL}
\affiliation{\ORSAY}
\author {R.~Dupr\'{e}} 
\affiliation{\ANL}
\affiliation{\ORSAY}
\author {N.A.~Baltzell} 
\affiliation{\ANL}
\affiliation{\JLAB}
\author {K.~Hafidi} 
\email[corresponding author: ]{kawtar@anl.gov}
\affiliation{\ANL}


\collaboration{The CLAS Collaboration}
\noaffiliation

%
\date{\today}
%
\begin{abstract}
We report the first fully exclusive measurement of deeply virtual Compton 
scattering (DVCS) off a nucleus for $A>1$. The experiment used the 6 GeV 
electron beam from Jefferson Lab projected onto a $^4$He target in the center of the CEBAF 
Large Acceptance Spectrometer (CLAS). An additional radial time projection 
chamber (RTPC) was used to detect the recoiling $^4$He nuclei and ensure the 
exclusivity of the process. We measured beam spin asymmetries significantly
larger than the one observed on proton targets and extract,
in a completely model independent way, the chiral even generalized parton 
distribution (GPD) of the $^4$He nucleus. This pioneering measurement 
leads the way toward the 3D imaging of the partonic structure of nuclei.
\end{abstract}
\pacs{Valid PACS appear here}

\maketitle 

A wealth of information on the quantum chromodynamics (QCD) lies in the 
internal structure of hadrons. In the recent past, the development of the 
generalized parton distributions (GPDs) framework has offered the possibility
to obtain new information about the momentum and spatial degrees of 
freedom of the quark and gluons inside hadrons [Add ref Mueller, Ji, Rad.]. 
In impact parameter space, the GPDs are indeed interpreted as a tomography of the 
transverse plane for partons carrying a certain longitudinal momentum 
\cite{Burkardt:2000za,Diehl:2002he,Belitsky:2002ep}. 

The main access to GPDs is through the measurement of deep virtual Compton 
scattering (DVCS), i.e. the hard exclusive electroproduction of a real photon. 
While other processes are known to be sensitive to GPDs, the measurement of the
DVCS is considered as the cleanest probe and has been the focus of a worldwide effort 
\cite{Stepanyan:2001sm,Airapetian,Chekanov:2003ya,Aktas:2005ty,Chen:2006na,Munoz 
Camacho:2006hx,Girod:2007aa,Gavalian:2009,Seder:2015,Pisano:2015,Jo:2015ema} 
[Should we cite less publications on the topic or a review?]
involving several accelerator facilities such as Jefferson Lab (JLab), HERA and  
CERN. The vast majority of these measurements was focused on the study of the 
proton structure and allow the extraction of the tomography of the nucleon [refs].
This framework is also applicable to nuclei, giving access to completely new 
information about the nuclear structure in terms of quarks and gluons[refs to add].
The study of the 3D structure of the nuclei appears to be especially important
in light of the large nuclear effects observed in nuclear parton distribution 
functions [ref to EMC reviews]. 

\begin{figure}[tb]
\includegraphics[width=6.5cm]{figs/DVCS_diagram.pdf}
\caption{Representation of the leading-order, twist-2, handbag diagram of the 
DVCS process off $^4$He.}
\label{fig:diags}
\end{figure}

Experimentally, the coherent nuclear DVCS channel is difficult to measure because of 
its small cross section and the necessity to ensure that the target remains 
intact while emitting a hard photon ($eA \rightarrow e' A' \gamma$). Figure 
\ref{fig:diags} illustrates the hand bag diagram for the coherent DVCS on 
$^4$He. Similarly to the proton case, at large photon's 4-momentum square 
$Q^2$ ($= -(k-k')^{2}$) and small squared momentum transfer $t$ 
($= (p-p')^{2}$), the DVCS handbag diagram can be factorized into two parts 
\cite{Freund_Collins,Ji_Osborne}. The hard part includes photons-quark 
interaction and is calculable with perturbative QED, while the non-perturbative 
part is parametrized in terms of GPDs, which embed the partonic structure of 
the hadron. The GPDs depend on three variables $x$, $\xi$ and $t$, which are 
indicated in Figure \ref{fig:diags} for the DVCS process. In DVCS, $\xi$ can 
be related to the Bjorken variable $x_{B}$: $\xi\approx {{x_B}\over{2-x_B}}$, 
where $x_B=\frac{Q^2}{2M\nu}$ with $M$ the proton mass and 
$\nu=E_e-E_{e^\prime}$. One can easily access $t$ as the squared momentum 
transfered to the target, however the $x$ variable cannot be experimentally 
accessed in a DVCS reaction. Hence, we measure only complex Compton Form 
Factors (CFF) [need reference] defined as: [add FORMULA for CFFs]. 

The main challenge to study nuclear DVCS is to ensure the exclusivity of the 
reaction. Until now, the only available data set of nuclear DVCS
was from the HERMES experiment \cite{Ellinghaus:2002zw}, where exclusivity is 
based on kinematic cuts on the measured scattered electron and real photon only. 
The measurement was performed on a large set of nuclei ($^4$He, N, Ne, Kr and 
Xe) but has been critisized for mixing coherent and incoherent processes 
[ref]. In this regard, the direct detection of the low-energy recoil nuclei is the best way to
garanty that the nuclei is intact and that the reaction did not occur on a 
bound nucleon. The CEBAF Large Acceptance Spectrometer (CLAS) setting  in the
Hall-B of JLab being already optimized for DVCS measurments 
[refs], we built a dedicated radial time projection chamber (RTPC) for 
the detection of the recoiling low energy nuclei in order to complete this
experimental setting. The $^4$He nucleus is an ideal experimental target in
this regard as it is light enough to be detected in such a setting, while it
is subject to significant nuclear effects \cite{JSeely} and has rather high 
density. Using a helium target leads to another important advantage, the number of GPDs 
defined for a hadron depends on its spin. The structure of a spin zero nuclei, such as 
$^4He$, is parametrized by only one chiral even GPD ($H_{A}(x,\xi,t)$) at 
leading twist, while 4 GPDs arise in the nucleon case. That significantly
simplifies the interpretation of the data and allow a model independent
extraction of the $^4$He CFF ($\mathcal{H}_{A}$) that will be presented at the
end of this letter. 

%experimental setup

The experiment E08-024 took place in the experimental Hall-B of Jefferson 
laboratory (JLab) in 2009 using the nearly 100\% duty factor longitudinally 
polarized electron beam (83$\%$ polarization) at its full energy of 6.064 
GeV. The data were collected over three months using a 6 atm gaseous $^4$He 
target placed in the center of CLAS. For DVCS experiments, the CLAS base line 
design \cite{CLAS_ref} is supplemented with an inner calorimeter (IC) and a
solenoid. The IC extends the photon detection acceptance of CLAS, which is 
originally from 15$^{\circ}$ to 45$^{\circ}$, to polar angle as low 
as 4$^{\circ}$. At these small angles the low-energy M\o{}ller 
electrons produced in the target form a very high rate background that is
suppressed by a 5 Tesla solenoid placed around the target. 

Because the coherent DVCS cross section is peaked at low $-t$, the recoil 
$^4$He nuclei produced in DVCS have a low average momentum around 250 MeV/c.
The CLAS cannot detect such low energy $\alpha$ particles, so in order to 
ensure the exclusivity of the measurement, we 
built a small and light RTPC to complement CLAS. Figure~\ref{fig:RTPC} presents 
the design of this RTPC that is surrounding the 
$^4$He gaseous target and inside the solenoid magnet.
The detector was specifically calibrated for the detection of
$^4$He nuclei using elastic scattering produced with a 1.2~GeV electron beam.


\begin{figure}[tb]
\vspace{-1.1cm}
\includegraphics[width=7.0cm,angle=-90]{figs/RTPC.pdf}
\vspace{-1.1cm}
\caption{Left: A picture of the E08-024 RTPC before insertion into the 
   solenoid. Right: A cross section of the E08-024 RTPC perpendicular to the 
   beam direction. An illustration of a $^4$He track originating from the 
   pressurized straw target is shown along with the electrons produced in the 
   drift region.}
\label{fig:RTPC}
\end{figure}

%DVCS selection
To identify the coherent DVCS events, we first select events where one 
electron, one $^4$He and at least one photon are detected in the final state. 
Electrons are identified by using fiducial cuts and detecting a good
signals in all the sub-detectors of the base line CLAS (drift chambers, 
Cherenkov counters, electromagnetic calorimeter, and time of flight 
scintillators). The $^4$He tracks are identified with fiducial, 
timing and track quality cuts in the RTPC detector. In addition,
we apply a vertex matching cut to make sure the electron and the helium
nuclei originate from the same place. The photons are detected either
in the IC or the CLAS electromagnetic calorimeter, note that even though the 
DVCS reaction has only one real photon in the final state, 
events with more than one good photon are not discarded at this stage. This is 
motivated by the fact that soft photons are likely to be produced in random 
coincidences, but cannot be mistaken for DVCS photons which are always at high
energies ($>2$ GeV).

Then, events with one or 
more $\pi^{0}$ are removed from the coherent DVCS sample. After that, the most 
energetic IC photon was considered as the DVCS photon candidate. Next, a 
$Q^{2}>1~[GeV^{2}/c^{2}]$ cut is applied on the DVCS candidates in order to 
ensure that the interaction occurs at the partonic level and the applicability 
of the factorization in the DVCS handbag diagram.  Once the three final state 
particles were identified with their 3-momentum vectors, the exclusivity of the 
coherent DVCS events were ensured by applying a set of exclusive cuts, which 
are: the co-planarity angle ($\Delta \phi$), missing energy, missing mass 
squared and missing transverse momentum in the $e'^4He'\gamma X$ final state 
configuration, the missing mass squared in the $e'^4He'X$ and $e'\gamma X$ 
configurations, and finally the cone angle ($\theta$) between the measured real 
photon and the missing particle in the $e'^4He'X$ configuration.  
Figure~\ref{fig:kin-cuts} presents four of the applied exclusivity cuts, where 
3$\sigma$ cuts are applied on all the exclusive quantities except the missing 
energy, for which a [-0.45,0.5] GeV cut was adopted to reduce the background 
contribution. Finally, 

\begin{figure}[tb]
\includegraphics[width=8.9cm]{figs/coh_exc_cuts.pdf}
\caption{Four of the seven coherent DVCS exclusivity cuts. The black 
distributions represent the coherent DVCS events candidate. The shaded 
distributions represent the events which passed all the exclusivity cuts except 
the quantity plotted. The vertical red lines represent the applied exclusivity 
cuts. The distributions from left to right and from top to bottoms are: $\Delta 
\phi$, missing energy and missing mass squared in $e'^4He'\gamma X$, and the 
cone angle ($\theta$) between the measured and the calculated photons.}
\label{fig:kin-cuts}
\end{figure}
After all the requirements on the individual final state particles of the 
coherent DVCS events and the exclusivity cuts, we ended up with about 3500 
events. Figure \ref{fig:kin-coverage} presents the ($Q^{2}$,$x_{B}$) and 
($Q^{2}$,$-t$) kinematic coverage of the collected DVCS events. 

\begin{figure}[tb]
\hspace{-0.45cm}
\includegraphics[width=9.0cm]{figs/Q2_xB_t_Coh.pdf}
\caption{The $Q^{2}$ as a function of $x_{B}$ (left) and the $Q^{2}$ as a 
function of $-t$ (right) for the identified coherent DVCS events after the 
exclusivity cuts.}
\label{fig:kin-coverage}
\end{figure}

Even with all the previously presented exclusive cuts, the selected events are
not all true DVCS events. We identified several background contributions to the 
coherent DVCS process, in particular accidental events and exclusive Deeply 
Virtual $\pi^0$ Production (DV$\pi^0$P). The accidental events where the 
different particles come from different events are suppressed by the limited 
phase space allowed by the exclusivity cuts. We estimate the accidental events 
to represent 4.1\% of our sample. This relatively large number is due to the 
small cross section of the DVCS. We evaluated this contribution by selecting 
events passing all our cuts but with particles originating from different 
vertices. Regarding the DV$\pi^0$P, which can easily be mistaken with DVCS when 
one of the two photons of the $\pi^0$ decay is produced at low energy in the 
laboratory frame. To estimate the importance of this background, we developed 
an event generator that we calibrated to match our measured experimental yield 
of exclusive $\pi^0$. We used this generator together with a GEANT3 simulation 
of our detection system to estimate the ratio of acceptance between DV$\pi^0$P 
where the two photons are detected and those where only one photon is detected 
and would pass our DVCS selection cuts. This ratio obtained from simulation is 
then multiplied by the measured yield of DV$\pi^0$P events, indicating a 
contamination of 2 to 4\%. The study of systematics errors showed that the main 
contributions come from the choice of the DVCS exclusivity cuts (8\%) and the 
large binning size (5.1\%). However added quadratically, these errors sum up to 
10\%, which remain for all bins well below the statistical errors.

%asymmetries
Different observables are available in DVCS, the cross sections, the BSA...
This work presents the first exclusive measurement of the beam-spin asymmetry 
of the reaction $e~^{4}He\rightarrow e'~^{4}He~\gamma$. This DVCS sensitive 
observable is measurable using a polarized lepton beam on an unpolarized target 
(U). It is convenient to use the beam-spin asymmetry as a DVCS observable 
because most of the experimental normalization and acceptance issues cancel out 
in the asymmetry ratio. It is defined in terms of the cross sections as:
  \begin{equation}
  A_{LU} = \frac{d^{5}\sigma^{+} - d^{5}\sigma^{-} }
                {d^{5}\sigma^{+} + d^{5}\sigma^{-}}.
    \label{BSA_equation}
  \end{equation}
where $d^{5}\sigma^{+}$($d^{5}\sigma^{-}$) is the DVCS differential cross 
section for a positive (negative) beam helicity. Experimentally, the DVCS 
reaction is indistinguishable from the Bethe-Heitler (BH) process, where the 
final photon is emitted either from the incoming or the outgoing leptons. The 
BH process is not sensitive to GPDs and does not carry information about the 
partonic structure of the hadronic target, and it's amplitude is calculable 
from the well-known electromagnetic form factors.

At leading twist order, the photon-electroproduction cross section can be 
decomposed into a BH, a DVCS, and an interference terms. The amplitudes of the 
three terms is approximated as a finite sum of Fourier harmonics, the so-called 
BMK approximation, as shown for the nucleon DVCS in \cite{Belitsky:2001ns} and 
for the spinless nuclei in \cite{Kirchner:2003wt,Belitsky:2008bz}. Therefore, 
the beam-spin asymmetry ($A_{LU}$) with the two opposite helicities of a 
longitudinally-polarized electron beam (L) on a spin-zero target (U) can be 
simplified as:
\small
\begin{equation}
\begin{split}
A_{LU}(\phi) =~~~~~~~~~~~~~~~~~~~~~~~~~~~~~~~~~~~~~~~~~~~~~~~~~~~~~~~~~\\
 \frac{\alpha_{0}(\phi) \, \Im m(\mathcal{H}_{A})}
{\alpha_{1}(\phi) + \alpha_{2}(\phi) \, \Re e(\mathcal{H}_{A}) + \alpha_{3}(\phi) \, 
\big( 
\Re e(\mathcal{H}_{A})^{2} + \Im m(\mathcal{H}_{A})^{2} \big)}
\label{eq:A_LU-coh}
\end{split}
\end{equation}
\normalsize
where $\Im m(\mathcal{H}_{A})$ and $\Re e(\mathcal{H}_{A})$ are the imaginary 
and real parts of the $^4$He CFF $\mathcal{H}_{A}$ associated to the GPD $H_A$. 
$\phi$ is the azimuthal angle between the ($e$,$e'$) and 
($\gamma^{*}$,$^4$He$'$) planes. The $\alpha_{i}$'s are $\phi$-dependent 
kinematical factors that depend on the nuclear form factor ($F_{A}(t)$) and the 
independent variables $Q^2$, $x_{B}$ and $t$. These factors are simplified as:
\small
\begin{eqnarray}
   \alpha_0 (\phi) & = &\frac{x_{A}(1+\epsilon^2)^2}{y} S_{++}(1) \sin(\phi)\\
    \alpha_1 (\phi) & = & c_0^{BH}+c_1^{BH} \cos({\phi})+c_2^{BH} \cos(2\phi)\\ 
   \alpha_2 (\phi) & = & \frac{x_{A}(1+\epsilon^2)^2}{y}  \left( C_{++}(0) +  
C_{++}(1) cos(\phi) \right)\\
\alpha_3 (\phi) &=& \frac{x^{2}_{A}t(1+\epsilon^2)^2}{y} {\mathcal P}_1(\phi) 
{\mathcal P}_2(\phi) \cdot 2 \frac{2-2y+y^2 + \frac{\epsilon^2}{2}y^2}{1 + 
\epsilon^2}
\end{eqnarray}
\normalsize

Where $x_{A} = \frac{x_{B}M_{N}}{M_{A}}$ with $M_{A}$ is the $^4$He mass, 
$\mathcal{P}_1(\phi)$ and $\mathcal {P}_2(\phi)$ are the Bethe-Heitler
propagators. The factors: $c_{0,1,2}^{BH}$, $c_0^{DVCS}$, $c_{0,1}^{INT}$ and
$s_1^{INT}$ are the Fourier coefficients of the BH, $S{++}(1)$, $C_{++}(0)$, 
and $C_{++}(1)$ are the Fourier harmonics in the leptonic tensor. Their 
explicit expressions can be found in \cite{Belitsky:2008bz}.  Therefore, Using 
the $\alpha_{i}$ factors, one can obtain in a totally model-independent way 
$\Im m(\mathcal{H}_{A})$ and $\Re e(\mathcal{H}_{A})$ from fitting the 
experimental $A_{LU}$ as a function of $\phi$ for given values of $Q^2$, 
$x_{B}$, and $t$.

$A_{LU}$ can be simplified in terms of the collected number of events in each 
beam-helicity state ($N^{+}$, $N^{-}$) as:
\begin{equation}
A_{LU} = \frac{1}{P_{B}} \frac{N^{+} - N^{-}}{N^{+} + N^{-} }.
\end{equation}
where $P_{B}$ is the beam polarization, and $N^{+}$ and $N^{-}$ are the number 
of DVCS events detected with positive and negative electron helicity with 
respect to the beam direction. The statistical uncertainty of $A_{LU}$ is
\begin{equation}
   \sigma_{A_{LU}} = \frac{1}{P_{B}} \sqrt{ \frac{1 - (P_{B}A_{LU})^{2}}{N}}
\end{equation}
where $N (N^{+} + N^{-}) $ is the total number of measured events.

Due to our limited statistics only, a two-dimensional binning is carried out in 
this work. The strongest dependence of $A_{LU}$ is on the azimuthal angle 
($\phi$). Thus, the coherent measured ranges of $Q^{2}$, $x_{B}$ and $t$ are 
binned statistically into three bins. Then, the identified DVCS events in each 
bin are binned into nine bins in $\phi$. Therefore, we are left
with $Q^{2}$-$\phi$ bins integrated over the full ranges of $x_{B}$ and $t$, 
$x_{B}$-$\phi$ bins integrated over $Q^{2}$ and $t$, and $t$-$\phi$ bins 
integrated over $Q^{2}$ and $x_{B}$.

Figure \ref{fig:alu} presents the coherent $A_{LU}$ for the three
sets of two-dimensional bins. The asymmetries are fitted with the form of
equation \ref{eq:A_LU-coh}, where the real and the imaginary part of the CFF
$\mathcal{H}_{A}$ are the free parameters in the fit. Figure \ref{fig:alu90} 
shows the $Q^2$, $x_{B}$, and $-t$-dependencies of the fitted $A_{LU}$ signals 
at $\phi$~=~90$^{\circ}$. The $x_{B}$ and $-t$-dependencies are compared to 
theoretical calculations performed by S.~Liuti and K.~Taneja 
\cite{simonetta_2}. Their model relies on the impulse approximation and uses 
advanced spectral function of the nuclei to calculate the nuclear GPDs and then 
the observables. The calculations were carried out at slightly different 
kinematics than ours but provide already some guidance. The experimental 
results appear to have larger asymmetries compared to the calculations.  These 
differences may arise from nuclear effects which are not taken into account in 
the model, such as long-range interactions. Our measurements also agree with 
those of HERMES, considering their large uncertainties.


\begin{figure}[tb]
\includegraphics[width=8.9cm]{figs/coherent-ALU.pdf}
\caption{The coherent $A_{LU}$ as a function of $\phi$ in
   $Q^{2}$(top panel), $x_{B}$ (middle panel), and $-t$ (bottom panel) bins.  
   The error bars represent the statistical uncertainties. The bluish-gray 
   bands represent the systematic uncertainties, including the normalisation 
systematic uncertainties. The red curves represent fits in the form of equation 
\ref{eq:A_LU-coh}.}
\label{fig:alu}
\end{figure}

\begin{figure}[tb]
\includegraphics[width=8.9cm]{figs/coherent-ALU_90.pdf}
\caption{The $Q^{2}$ (left), $x_{B}$ (middle), and $-t$-dependencies (right) of
   the coherent $A_{LU}$ at $\phi$~=~90$^{\circ}$ (black squares). On the 
   middle plot: the full-red and the dashed-blue curves are theoretical 
   calculations from \cite{simonetta_2}. On the right: the green circles are 
   the HERMES $-A_{LU}$ (positron beam was used) inclusive measurements 
   \cite{HERMES_BSA}, the colored curves represent theoretical calculations 
from \cite{simonetta_2}.}
\label{fig:alu90}
\end{figure}

As has been advertised previously, the $^4$He CFF $\mathcal{H}_A$ can be 
extracted from fitting the experimentally measured coherent $A_{LU}$ in a 
totally model independent way, which is not the case of nucleon targets. For 
the later, four CFFs exist and extracting them is always made by making 
limitations according some theoretical models and by neglecting some CFFs 
\cite{Jo:2015ema}. Figure \ref{fig:CFF_HA} presents our extracted imaginary and 
real parts of the CFF $\mathcal{H}_A$ as function of the kinematical variable 
($Q^{2}$, $x_B$,$-t$), and compared to some theoretical calculations.

\begin{figure}[tb]
\includegraphics[width=8.9cm]{figs/Coherent_CFF.pdf}
\caption{The model-independent extraction of the imaginary (top panel) and
real (bottom panel) parts of the $^4$He CFF $\mathcal{H}_A$, as functions of
$Q^{2}$ (right panel), $x_B$ (middle panel), and $t$ (left panel). The full red 
curves are calculations based on an on-shell model from
\cite{Vadim_priv}. The black-dashed curves are calculations from a convolution 
model based on the VGG model for the nucleons' GPDs \cite{Guidal_priv}. The 
blue long-dashed curve on the top-right plot is from
an off-shell model based on \cite{GonzalezHernandez:2012jv}.}
\label{fig:CFF_HA}
\end{figure}

We display in figure \ref{fig:CFF_HA} calculations for $\mathcal{H}_A$ from 
three GPD models: On-shell, Convolution, and off-shell models. In the On-shell 
model \cite{Vadim_priv}, a nucleus is assumed to be composed of 
non-relativistic non-interacting nucleons, and these nucleons interact 
independently with the probe. For the nucleons, the GPDs are modeled according 
to the dual parametrization \cite{Guzey:2006xi}. In the Convolution model, the 
same assumption has been made for the nucleus, while the nucleon GPDs were 
extracted from the VGG model, which is based on the double distributions ansatz 
\cite{DD_model}. The Off-shell model \cite{GonzalezHernandez:2012jv} relies on 
the impulse approximation also, but uses advanced spectral function of the 
nuclei that accounts for all configurations of the final nuclear system and the 
binding effects between the nucleons.

In figure \ref{fig:CFF_HA}, within the given uncertainties, the extracted CFF 
shows a slight dependence on $Q^{2}$, $x_B$, and $-t$, which are in agreement 
with the theoretical calculations. One can see a difference between the 
precision of the extracted imaginary and real parts, which is expected because 
$\alpha_2$ is suppressed compared to $\alpha_0$ contribution. However, we note 
that the error bars are finite and that the fit converge without placing any 
bound on the CFF, which is necessary on proton targets.


%conclusion
In summary, we presented the first exclusive measurement of the coherent DVCS 
off $^4$He using CLAS spectrometer, supplemented with an inner calorimeter, a 5 
Tesla solenoid, and a specially designed radial TPC. This dataset represents a 
unique source for the nuclear DVCS global dataset, which will be used to 
constrain GPD models. The measured beam-spin asymmetries show a very strong 
signal and allowed to perform the first fully model-independent extraction of 
the $^4$He CFF $\mathcal{H}_A$. The extracted CFFs, while limited by 
statistics, are in a good agreement with the available GPD models. This opens 
many new perspectives to study the nuclear structure within the GPDs framework 
and pave the way for future more precise measurements at JLab 12~GeV program 
and possibly at the Electron-Ion Collider (EIC) to achieve better understanding 
of the nuclear effects.

The incoherent DVCS channel 
measurement from the same data set in in preparation and will reported in 
another publication. 
%Acknowledgments

We acknowledge the staff of the Accelerator and Physics Divisions at Jefferson 
Lab for making this experiment possible. This work is supported by the U.S.  
Department of Energy, Office of Science, Office of Nuclear Physics contract 
DE-AC05-06OR23177.

\begin{thebibliography}{99}

\bibitem{Burkardt:2000za} 
  M.~Burkardt,
  Phys.\ Rev.\ D {\bf 62}, 071503 (2000)
  Erratum: [Phys.\ Rev.\ D {\bf 66}, 119903 (2002)]

\bibitem{Diehl:2002he} 
  M.~Diehl,
  Eur.\ Phys.\ J.\ C {\bf 25}, 223 (2002)
  Erratum: [Eur.\ Phys.\ J.\ C {\bf 31}, 277 (2003)]
 
\bibitem{Belitsky:2002ep} 
  A.~V.~Belitsky and D.~Mueller,
  Nucl.\ Phys.\ A {\bf 711}, 118 (2002)

\bibitem{Burkardt:2005hp} 
  M.~Burkardt,
  Phys.\ Rev.\ D {\bf 72}, 094020 (2005)

\bibitem{Stepanyan:2001sm}
S.~Stepanyan {\it et al.} [CLAS Collaboration],
Phys.\ Rev.\ Lett. {\bf 87}, 182002 (2001).

\bibitem{Airapetian}
A. Airapetian {\it et al.} [HERMES Collaboration],
Phys.\ Rev.\ Lett. {\bf 87}, 182001 (2001);
JHEP {\bf 1207}, 032 (2012);
JHEP {\bf 1006}, 019 (2010);
JHEP {\bf 0806}, 066 (2008);
Phys.\ Lett.\ B {\bf 704}, 15 (2011);
Phys.\ Rev.\  D {\bf 75}, 011103 (2007);
JHEP {\bf 0911}, 083 (2009);
Phys.\ Rev.\ C {\bf 81}, 035202 (2010);
JHEP {\bf 1210}, 042 (2012).

\bibitem{Chekanov:2003ya}
S. Chekanov {\it et al.} [ZEUS Collaboration],
Phys.\ Lett.\  B {\bf 573}, 46 (2003).

\bibitem{Aktas:2005ty}
A. Aktas {\it et al.} [H1 Collaboration],
Eur.\ Phys.\ J.\ C {\bf 44}, 1 (2005).

\bibitem{Chen:2006na} 
S.~Chen {\it et al.} [CLAS Collaboration],
Phys.\ Rev.\ Lett.\ {\bf 97}, 072002 (2006).

\bibitem{Munoz Camacho:2006hx} 
C. Mu\~noz Camacho {\it et al.} [Jefferson Lab Hall A Collaboration],
Phys.\ Rev.\ Lett. {\bf 97}, 262002 (2006).

\bibitem{Girod:2007aa} 
F.X. Girod {\it et al.} [CLAS Collaboration],
Phys.\ Rev.\ Lett. {\bf 100}, 162002 (2008).

\bibitem{Gavalian:2009} 
G. Gavalian {\it et al.} [CLAS Collaboration],
Phys.\ Rev.\ C {\bf 80}, 035206 (2009).

\bibitem{Seder:2015} 
E. Seder {\it et al.} [CLAS Collaboration],
Phys.\ Rev.\ Lett. {\bf 114}, 032001 (2015).

\bibitem{Pisano:2015} 
S.~Pisano {\it et al.} [CLAS Collaboration],
Phys.\ Rev.\ D {\bf 91}, 052014 (2015).

\bibitem{Jo:2015ema} H.~S.~Jo {\it et al.} [CLAS Collaboration],
  Phys.\ Rev.\ Lett.\  {\bf 115}, no. 21, 212003 (2015)

\bibitem{Mazouz:2007aa} 
  M.~Mazouz {\it et al.} [Jefferson Lab Hall A Collaboration],
   Phys.\ Rev.\ Lett.\  {\bf 99}, 242501 (2007)

\bibitem{Freund_Collins}
A.~Freund and J.C.~Collins, 
Phys.\ Rev.\ D {\bf 59}, 074009 (1998)

\bibitem{Ji_Osborne}
X.-D.~Ji and J.~Osborne, 
Phys.\ Rev.\ D {\bf 58}, 094018 (1998)

\bibitem{Belitsky}
A.~V.~Belitsky and A.~V.~Radyushkin, 
Phys.\ Rept.\ vol. 418 (2005)

\bibitem{Guidal:2013rya}
M.~Guidal, H.~Moutarde and M.~Vanderhaeghen,
Rept.\ Prog.\ Phys.\  {\bf 76}, 066202 (2013)

\bibitem{JSeely}
J. Seely {\it et al.} 
Phys.\ Rev.\ Lett.\ {\bf 103}, 202301 (2009)

\bibitem{CLAS_ref}
B.A. Mecking {\it et al.}, 
Nucl.\ Inst.\ and Meth.\ A 503, 513 (2003)

\bibitem{Belitsky:2001ns}
A.~V.~Belitsky, D.~Mueller and A.~Kirchner,
Nucl.\ Phys.\ B {\bf 629}, 323 (2002)

\bibitem{Kirchner:2003wt}
A.~Kirchner and D.~Mueller, 
Eur.\ Phys.\ J.\ C {\bf 32}, 347 (2003)

\bibitem{Belitsky:2008bz}
A.~V.~Belitsky and D.~Mueller,
Phys.\ Rev.\ D {\bf 79}, 014017 (2009)


\bibitem{Ellinghaus:2002zw}
F.~Ellinghaus {\it et al.} [HERMES Collaboration],
AIP Conf.\ Proc.\  {\bf 675}, 303 (2003)

\bibitem{eg6_note}
M. Hattawy {\it et al.} (CLAS-EG6 Working Group), 
CLAS internal analysis note, 2016.

\bibitem{simonetta_2}
S.~Liuti and K.~Taneja, 
Phys.\ Rev.\ C {\bf 72}, 032201 (2005)

\bibitem{HERMES_BSA}
A. Airapetian et al. (HERMES Collaboration), 
Phys.\ Rev.\ C {\bf 81}, 035202 (2010)

\bibitem{Vadim_priv}
Private communications with V.~Guzey based on: 
V.~Guzey, Phys.\ Rev.\ C {\bf 78}, 025211 (2008).

\bibitem{Guzey:2006xi}
V.~Guzey and T.~Teckentrup,
Phys.\ Rev.\ D {\bf 74}, 054027 (2006)

\bibitem{Guidal_priv}
Private communications with M.~Guidal based on: 
M.~Guidal, M.~V.~Polyakov, A.~V.~Radyushkin and M.~Vanderhaeghen, 
Phys.\ Rev.\ D {\bf 72}, 054013 (2005).

\bibitem{DD_model}
I.~V.~Musatov and A.~V.~Radyushkin, 
Phys.\ Rev.\ D {\bf 61}, 074027 (2000).

\bibitem{GonzalezHernandez:2012jv}
Private communications with S.~Liuti based on: 
J.~O.~Gonzalez-Hernandez, S.~Liuti, G.~R.~Goldstein and K.~Kathuria,
Phys.\ Rev.\ C {\bf 88}, no. 6, 065206, (2013).


\end{thebibliography}

\end{document}
