\documentclass[nofootinbib,twocolumn,showpacs,prl,superscriptaddress,secnumarabic,amssymb,nobibnotes,aps,floatfix]{revtex4}
\usepackage{graphicx}
\usepackage{hyperref}
\usepackage{amsmath}
\usepackage{dcolumn}% Align table columns on decimal point
\usepackage{bm}% bold math

%
\def\Dirac#1{#1\hskip-5pt/}
%
\begin{document}

\title{First Exclusive Measurement of Deep Virtual Compton Scattering off $^4$He: Toward the 3D tomography of nuclei}

\newcommand*{\ANL}{Argonne National Laboratory, Argonne, Illinois 60439}
\newcommand*{\ANLindex}{1}
\affiliation{\ANL}
\newcommand*{\ORSAY}{Institut de Physique Nucl\'eaire, CNRS/IN2P3 and Universit\'e Paris Sud, Orsay, France}
\newcommand*{\ORSAYindex}{2}
\affiliation{\ORSAY}
\newcommand*{\JLAB}{Thomas Jefferson National Accelerator Facility, Newport News, Virginia 23606}
\newcommand*{\JLABindex}{3}
\affiliation{\JLAB}
 
\newcommand*{\NOWJLAB}{Thomas Jefferson National Accelerator Facility, Newport News, Virginia 23606}
\newcommand*{\NOWODU}{Old Dominion University, Norfolk, Virginia 23529}
 %%%%%%%%%%%%%%% END OF Latex Macros for institute addresses  %%%%%%%%%%%%%%%%%%%%%%%%% 

\author {M.~Hattawy}
\affiliation{\ANL}
\affiliation{\ORSAY}
\author {R.~Dupr\'{e}} 
\affiliation{\ANL}
\affiliation{\ORSAY}
\author {N.A.~Baltzell} 
\affiliation{\ANL}
\affiliation{\JLAB}
\author {K.~Hafidi} 
\email[corresponding author: ]{kawtar@anl.gov}
\affiliation{\ANL}


\collaboration{The CLAS Collaboration}
\noaffiliation

%
\date{\today}
%
\begin{abstract}
We report the first exclusive measurements of deeply virtual Compton scattering 
(DVCS) off a nucleus, where all the products of the reaction including the 
recoil $^4$He nucleus were detected. The experiment was performed using the 
Jefferson Lab CEBAF Large Acceptance Spectrometer (CLAS) enhanced with a radial 
time projection chamber (RTPC) to detect the recoiling $^4$He nuclei. We 
measure large beam spin asymmetries comparable to the proton's ones and extract 
in a model independent way, the single chirally-even generalized parton 
distribution of the $^4$He nucleus. These are pioneering measurements and will 
lead the way toward the 3D imaging of the partonic structure of nuclei.  
\end{abstract}
\pacs{Valid PACS appear here}

\maketitle 

A wealth of information on the quantum chromodynamics (QCD) structure of 
hadrons lies in the correlations between the momentum and spatial degrees of 
freedom of the fundamental constituent partons, quarks and gluons. Such 
correlations are accessible via the generalized parton distributions (GPDs).  
The GPDs correspond to the coherence between quantum states of different (or 
same) helicity, longitudinal momentum, and transverse position. In an impact 
parameter space, they can be interpreted as a distribution in the transverse 
plane of partons carrying a certain longitudinal momentum 
\cite{Burkardt:2000za,Diehl:2002he,Belitsky:2002ep}. A crucial feature of GPDs 
is the access to the transverse position of partons which, combined with their 
longitudinal momentum, leads to the total angular momentum of partons 
\cite{Burkardt:2005hp}. Deep virtual Compton scattering (DVCS) corresponding to 
hard exclusive electroproduction of a real photon, which is considered as the 
cleanest probe to access GPDs and thus study the 3D imaging of nucleons and 
nuclei.

DVCS measurements have been the focus of a worldwide effort 
\cite{Stepanyan:2001sm,Airapetian,Chekanov:2003ya,Aktas:2005ty,Chen:2006na,Munoz 
Camacho:2006hx,Girod:2007aa,Gavalian:2009,Seder:2015,Pisano:2015,Jo:2015ema} 
involving several accelerator facilities such as Jefferson Lab (JLab), HERA and  
CERN. The vast majority of the experiments focused on the study of the 
nucleon's structure. The deuterium was also investigated at HERMES and JLab 
\cite{Mazouz:2007aa} mainly as a neutron target. However, studying the 3D 
imaging of the nucleon is a very important goal, understanding how these 
distributions are modified to provide the binding and structure in a nucleus is 
as fascinating of a question and an integral part of our quest of using QCD to 
explore nuclear matter. 

A DVCS process on a nuclear target differ from single nucleon scattering in 
providing access to the measure two DVCS channels. In the coherent DVCS 
channel, the target nucleus ($A$) remains intact and recoils as a whole while 
emitting a real photon ($eA \rightarrow e' A' \gamma$), allowing to measure the 
nuclear GPDs of the target. In the incoherent channel, the nucleus breaks up 
and the DVCS takes place on a bound nucleon ($N$) that emits the final photon 
($eA \rightarrow e' N' \gamma$ X), enabling the GPDs measurement of the bound 
nucleons and to study their modifications in the nuclear medium via the GPDs.  
Figure \ref{fig:diags} illustrates the dominant mechanism for the coherent DVCS 
channel on $^4$He. At sufficiently large squared electron momentum transfer 
$Q^2$ ($= -(k-k')^{2}$) and small squared momentum transfer $t$ ($= 
(p-p')^{2}$), the QCD factorization theorem predicts that the DVCS handbag 
diagram can be factorized into two parts, hard and soft parts 
\cite{Freund_Collins,Ji_Osborne}. The hard part includes photons-quark 
interaction and it is calculable through perturbative methods, while the 
soft/non-perturbative part is parametrized in terms of GPDs, which embed the 
partonic structure of the hadron.  
\begin{figure}[tb]
\includegraphics[width=6.5cm]{figs/DVCS_diagram.pdf}
\vspace{-0.cm}
\caption{Deep virtual Compton scattering process in the handbag approximation.}
\label{fig:diags}
\end{figure}

%introduction
The GPDs are defined for each quark flavor and gluon as matrix
elements of the light cone operators \cite{Belitsky}, describing the transition 
between the initial and final states of a hadron. The GPDs depend on two 
longitudinal momentum fraction variables ($x$, $\xi$) and on the momentum 
transfer $t$ to the target. $x$ is the average longitudinal momentum fraction 
of the parton involved in the process and $\xi$ is the longitudinal fraction of 
the momentum transfer $t$, which is related to the Bjorken variable $x_{B}$: 
$\xi\approx {{x_B}\over{2-x_B}}$, where $x_B=\frac{Q^2}{2M\nu}$ with
the proton mass $M$ and $\nu=E_e-E_{e^\prime}$. The GPDs $x$ variable cannot be 
measured experimentally in a DVCS reaction. Hence, we measure the their 
convolutions on $x$, the so-called Compton Form Factors (CFF) 
\cite{Guidal:2013rya}.  In a DVCS process, the number of GPDs needed to 
parametrize the partonic structure of a hadron depends on the different 
configurations between the spin of the hadron and the helicity direction of the 
struck quark.  Therefore, the partonic structure of spin zero nuclei, such as 
$^4He$ and $^{12}C$, is parametrized by only one GPD ($H_{A}(x,\xi,t)$) at 
leading twist, while 4 GPDs arise in the nucleon case.  In this work, we have 
chose the $^4$He nucleus as our target of interest because of it's spinless 
nature and it shows a clear EMC effect \cite{JSeely}, in addition of having a 
high density and it is a well-known few-body system.

The study of nuclear DVCS is still in its infancy due to the challenging 
detection of the low-energy recoil nuclei in fixed target experiments. Until 
very recently, the HERMES experiment \cite{Ellinghaus:2002zw} was the only one 
to measure DVCS off heavier nuclei such as $^4$He, N, Ne, Kr and Xe, where only 
the scattered electron and the real photon are detected. In this paper, we 
report the first exclusive measurements of the coherent DVCS channel off $^4$He 
where all products of the reaction are detected including the recoiling $^4$He 
nucleus. Following this exclusive measurement, the $^4$He CFF 
($\mathcal{H}_{A}$) will be extracted experimentally in a fully model 
independent way for the first time ever. The incoherent DVCS channel 
measurement from the same data set in in preparation and will reported in 
another publication. 

This work presents the first exclusive measurement of the beam-spin asymmetry 
of the reaction $e~^{4}He\rightarrow e'~^{4}He~\gamma$.This DVCS sensitive 
observable is measurable using a polarized lepton beam on an unpolarized target 
(U). It is convenient to use the beam-spin asymmetry as a DVCS observable 
because most of the experimental normalization and acceptance issues cancel out 
in the asymmetry ratio. It is defined in terms of the cross sections as:
  \begin{equation}
  A_{LU} = \frac{d^{5}\sigma^{+} - d^{5}\sigma^{-} }
                {d^{5}\sigma^{+} + d^{5}\sigma^{-}}.
    \label{BSA_equation}
  \end{equation}
where $d^{5}\sigma^{+}$($d^{5}\sigma^{-}$) is the DVCS differential cross 
section for a positive (negative) beam helicity. Experimentally, the DVCS 
reaction is indistinguishable from the Bethe-Heitler (BH) process, where the 
final photon is emitted either from the incoming or the outgoing leptons. The 
BH process is not sensitive to GPDs and does not carry information about the 
partonic structure of the hadronic target, and it's amplitude is calculable 
from the well-known electromagnetic form factors.

The photon-electroproduction cross section can be decomposed into a BH,
a DVCS, and an interference terms. At leading twist, the amplitudes of the 
three terms can be decomposed into a finite sum of Fourier harmonics, the 
so-called BMK formalism, as shown for the nucleon DVCS in 
\cite{Belitsky:2001ns} and for the spinless nuclei in 
\cite{Kirchner:2003wt,Belitsky:2008bz}. Therefore, the beam-spin asymmetry 
($A_{LU}$) with the two opposite helicities of a longitudinally-polarized 
electron beam (L) on a spin-zero target (U) can be written as:
\small
\begin{equation}
\begin{split}
A_{LU}(\phi) =~~~~~~~~~~~~~~~~~~~~~~~~~~~~~~~~~~~~~~~~~~~~~~~~~~~~~~~~~\\
 \frac{\alpha_{0}(\phi) \, \Im m(\mathcal{H}_{A})}
{\alpha_{1}(\phi) + \alpha_{2}(\phi) \, \Re e(\mathcal{H}_{A}) + \alpha_{3}(\phi) \, 
\big( 
\Re e(\mathcal{H}_{A})^{2} + \Im m(\mathcal{H}_{A})^{2} \big)}
\label{eq:A_LU-coh}
\end{split}
\end{equation}
\normalsize
where $\Im m(\mathcal{H}_{A})$ and $\Re e(\mathcal{H}_{A})$ are the imaginary 
and real parts of the $^4$He CFF $\mathcal{H}_{A}$ associated to the GPD $H_A$.  
The $\alpha_{i}$'s are $\phi$-dependent kinematical factors that depend on the 
nuclear form factor ($F_{A}(t)$) and the independent variables $Q^2$, $x_{B}$ 
and $t$. These factors are simplified as:
\small
\begin{eqnarray}
   \alpha_0 (\phi) & = &\frac{x_{A}(1+\epsilon^2)^2}{y} S_{++}(1) \sin(\phi)\\
    \alpha_1 (\phi) & = & c_0^{BH}+c_1^{BH} \cos({\phi})+c_2^{BH} \cos(2\phi)\\ 
   \alpha_2 (\phi) & = & \frac{x_{A}(1+\epsilon^2)^2}{y}  \left( C_{++}(0) +  
C_{++}(1) cos(\phi) \right)\\
\alpha_3 (\phi) &=& \frac{x^{2}_{A}t(1+\epsilon^2)^2}{y} {\mathcal P}_1(\phi) 
{\mathcal P}_2(\phi) \cdot 2 \frac{2-2y+y^2 + \frac{\epsilon^2}{2}y^2}{1 + 
\epsilon^2}
\end{eqnarray}
\normalsize

Where $S{++}(1)$, $C_{++}(0)$, and $C_{++}(1)$ are the Fourier harmonics in the 
leptonic tensor. Their explicit expressions can be found in 
\cite{Belitsky:2008bz}. Therefore, Using the $\alpha_{i}$ factors, one can 
obtain in a totally model-independent way $\Im m(\mathcal{H}_{A})$ and $\Re 
e(\mathcal{H}_{A})$ from fitting the experimental $A_{LU}$ as a function of 
$\phi$, the azimuthal angle between the ($e$,$e'$) and ($\gamma^{*}$,$^4$He$'$) 
planes, for given values of $Q^2$, $x_{B}$, and $t$.

%experimental setup

The experiment, CLAS-EG6, took place in the experimental Hall-B of Jefferson 
laboratory (JLab) in 2009. JLab delivers, simultaneously, a nearly 100\% duty 
factor polarized electrons into three experimental Halls (A, B, C). The data 
were collected over three months via projecting a 6.064 GeV longitudinally 
polarized beam, (83$\%$ polarization), on a 6 atm gaseous $^4$He target.  The 
Hall-B Large Acceptance Spectrometer (CLAS) basic design \cite{CLAS_ref} was 
supplemented, during the CLAS-E1DVCS1 experimental run \cite{Girod:2007aa} in 
2005, with a specially designed electromagnetic calorimeter, Inner calorimeter 
(IC). The IC has extended the photon detection acceptance of CLAS, which is 
originally from 15$^{\circ}$ to 45$^{\circ}$, to polar angle reach as minimum 
as 4$^{\circ}$.  During the same experiment, a 5 Tesla solenoid was added 
around the target to shield the inner detectors from the low-energy M\o ller 
electrons.

At 6 GeV incident electron beam energy, the recoil $^4$He nuclei, from the 
coherent DVCS channel, have an average momentum (per charge) around 125 MeV/c, 
while the CLAS spectrometer detects charged particles with a threshold of 250 
MeV/c. In order to ensure the exclusivity of the our coherent DVCS channel, we 
built a small and light radial time projection chamber (RTPC) to detect 
recoiling nuclei down to energies of few MeVs. Figure~\ref{fig:RTPC} presents 
our cylindrical RTPC, which is 20 cm long and 15 cm diameter, surrounding the 
$^4$He gaseous target and being inside the available space inside the solenoid 
, with a 3 cm radial drift length. The detector was specifically calibrated for 
$^4$He nuclei using elastic scattering produced with a 1.2~GeV electron beam.


\begin{figure}[tb]
\includegraphics[width=7.0cm,angle=-90]{figs/RTPC.pdf}
\vspace{-1.1cm}
\caption{Left: A picture of the CLAS-EG6 RTPC before insertion into the 
   solenoid. Right: A cross section of the CLAS-EG6 RTPC perpendicular to the 
   beam direction. An illustration of a $^4$He track originating from the 
   pressurized straw target is shown along with the electrons produced in the 
drift region.}
\label{fig:RTPC}
\end{figure}

%DVCS selection
Identifying the coherent DVCS candidates is the first step of the data 
analysis. These events have one electron, one $^4$He, and at least one photon 
in the final state. Electrons were identified by passing the fiducial cuts and 
having signals in all the sub-detectors of CLAS spectrometer (drift chambers, 
Cherenkov counters, the standard CLAS electromagnetic calorimeter, and 
scintillators). $^4$He tracks were identified by passing all the geometrical, 
timing and quality cuts in the RTPC detector. The most energetic IC photon was 
considered as the DVCS photon candidate. Next, a $Q^{2}>1~[GeV^{2}/c^{2}]$ cut 
is applied on the DVCS candidates in order to ensure that the interaction 
occurs at the partonic level and the applicability of the factorization in the 
DVCS handbag diagram. Once the three final state particles were identified with 
their 3-momentum vectors, the exclusivity of the coherent DVCS events were 
ensured by applying a set of exclusive cuts, which are: the co-planarity angle 
($\Delta \phi$), missing energy, missing mass squared and missing transverse 
momentum in the $e'^4He'\gamma X$ final state configuration, the missing mass 
squared in the $e'^4He'X$ and $e'\gamma X$ configurations, and finally the cone 
angle ($\theta$) between the measured real photon and the missing particle in 
the $e'^4He'X$ configuration. Figure~\ref{fig:kin-cuts} presents four of the 
applied exclusivity cuts, where 3$\sigma$ cuts are applied on all the exclusive 
quantities except the missing energy, for which a [-0.45,0.5] GeV cut was 
adopted to reduce the background contribution.

\begin{figure}[tb]
\includegraphics[width=8.9cm]{figs/coh_exc_cuts.pdf}
\caption{Four of the seven coherent DVCS exclusivity cuts. The black 
distributions represent the coherent DVCS events candidate. The shaded 
distributions represent the events which passed all the exclusivity cuts except 
the quantity plotted. The vertical red lines represent the applied exclusivity 
cuts. The distributions from left to right and from top to bottoms are: $\Delta 
\phi$, missing energy and missing mass squared in $e'^4He'\gamma X$, and the 
cone angle ($\theta$) between the measured and the calculated photons.}
\label{fig:kin-cuts}
\end{figure}
After all the requirements on the individual final state particles of the 
coherent DVCS events and the exclusivity cuts, we ended up with about 3500 
events. Figure \ref{fig:kin-coverage} presents the ($Q^{2}$,$x_{B}$) and 
($Q^{2}$,$-t$) kinematic coverage of the collected DVCS events. 

\begin{figure}[tb]
\hspace{-0.45cm}
\includegraphics[width=9.0cm]{figs/Q2_xB_t_Coh.pdf}
\caption{The $Q^{2}$ as a function of $x_{B}$ (left) and the $Q^{2}$ as a 
function of $-t$ (right) for the identified coherent DVCS events after the 
exclusivity cuts.}
\label{fig:kin-coverage}
\end{figure}

%asymmetries
Even with all the previously presented exclusive cuts, the selected events are
not all true DVCS events. We identified several background contributions to the 
coherent DVCS process, in particular accidental events and exclusive Deeply 
Virtual $\pi^0$ Production (DV$\pi^0$P). The accidental events where the 
different particles come from different events are suppressed by the limited 
phase space allowed by the exclusivity cuts. We estimate the accidental events 
to represent 4.1\% of our sample. This relatively large number is due to the 
small cross section of the DVCS. We evaluated this contribution by selecting 
events passing all our cuts but with particles originating from different 
vertices. Regarding the DV$\pi^0$P, which can easily be mistaken with DVCS when 
one of the two photons of the $\pi^0$ decay is produced at low energy in the 
laboratory frame. To estimate the importance of this background, we developed 
an event generator that we calibrated to match our measured experimental yield 
of exclusive $\pi^0$. We used this generator together with a GEANT3 simulation 
of our detection system to estimate the ratio of acceptance between DV$\pi^0$P 
where the two photons are detected and those where only one photon is detected 
and would pass our DVCS selection cuts. This ratio obtained from simulation is 
then multiplied by the measured yield of DV$\pi^0$P events, indicating a 
contamination of 2 to 4\%. The study of systematics errors showed that the main 
contributions come from the choice of the DVCS exclusivity cuts (8\%) and the 
large binning size (5.1\%). However added quadratically, these errors sum up to 
10\%, which remain for all bins well below the statistical errors.

$A_{LU}$ can be simplified in terms of the collected number of events in each 
beam-helicity state ($N^{+}$, $N^{-}$) as:
\begin{equation}
A_{LU} = \frac{1}{P_{B}} \frac{N^{+} - N^{-}}{N^{+} + N^{-} }.
\end{equation}
where $P_{B}$ is the beam polarization, and $N^{+}$ and $N^{-}$ are the number 
of DVCS events detected with positive and negative electron helicity with 
respect to the beam direction. The statistical uncertainty of $A_{LU}$ is
\begin{equation}
   \sigma_{A_{LU}} = \frac{1}{P_{B}} \sqrt{ \frac{1 - (P_{B}A_{LU})^{2}}{N}}
\end{equation}
where $N (N^{+} + N^{-}) $ is the total number of measured events.

Due to our limited statistics only, a two-dimensional binning is carried out in 
this work. The strongest dependence of $A_{LU}$ is on the azimuthal angle 
($\phi$). Thus, the coherent measured ranges of $Q^{2}$, $x_{B}$ and $t$ are 
binned statistically into three bins. Then, the identified DVCS events in each 
bin are binned into nine bins in $\phi$.  Therefore, we are left
with $Q^{2}$-$\phi$ bins integrated over the full ranges of $x_{B}$ and $t$, 
$x_{B}$-$\phi$ bins integrated over $Q^{2}$ and $t$, and $t$-$\phi$ bins 
integrated over $Q^{2}$ and $x_{B}$.

Figure \ref{fig:A_LU-coh} presents the coherent $A_{LU}$ for the three
sets of two-dimensional bins. The asymmetries are fitted with the form of
equation \ref{eq:A_LU-coh}, where the real and the imaginary part of the CFF
$\mathcal{H}_{A}$ are the free parameters in the fit. Figure \ref{fig:alu90} 
shows the $Q^2$, $x_{B}$, and $-t$-dependencies of the fitted $A_{LU}$ signals 
at $\phi$~=~90$^{\circ}$. The $x_{B}$ and $-t$-dependencies are compared to 
theoretical calculations performed by S.~Liuti and K.~Taneja 
\cite{simonetta_2}.  Their model relies on the impulse approximation and uses 
advanced spectral function of the nuclei to calculate the nuclear GPDs and then 
the observables. The calculations were carried out at slightly different 
kinematics than ours but provide already some guidance. The experimental 
results appear to have larger asymmetries compared to the calculations.  These 
differences may arise from nuclear effects which are not taken into account in 
the model, such as long-range interactions. Our measurements also agree with 
those of HERMES, considering their large uncertainties.


\begin{figure}[tb]
\includegraphics[width=8.9cm]{figs/coherent-ALU.pdf}
\caption{The coherent $A_{LU}$ as a function of $\phi$ in
   $Q^{2}$(top panel), $x_{B}$ (middle panel), and $-t$ (bottom panel) bins.  
   The error bars represent the statistical and the systematic uncertainties 
   added quadratically, shown on top in green are error bars representing only 
   the statistical uncertainties. The bluish-gray bands represent the 
   systematic uncertainties, including the normalisation systematic 
uncertainties. The red curves represent fits in the form of equation 
\ref{eq:A_LU-coh}.}
\label{fig:alu}
\end{figure}

\begin{figure}[tb]
\includegraphics[width=8.9cm]{figs/coherent-ALU_90.pdf}
\caption{The $Q^{2}$ (left), $x_{B}$ (middle), and $-t$-dependencies (right) of
   the coherent $A_{LU}$ at $\phi$~=~90$^{\circ}$ (black squares). On the 
   middle plot: the full-red and the dashed-blue curves are theoretical 
   calculations from \cite{simonetta_2}. On the right: the green circles are 
   the HERMES $-A_{LU}$ (positron beam was used) inclusive measurements 
   \cite{HERMES_BSA}, the colored curves represent theoretical calculations 
from \cite{simonetta_2}.}
\label{fig:alu90}
\end{figure}

\begin{figure}[tb]
\includegraphics[width=8.9cm]{figs/Coherent_CFF.pdf}
\caption{The model-independent extraction of the imaginary (blue points) and
real (red points) parts of the $^4$He CFF $\mathcal{H}_A$, as functions of
$Q^{2}$ (on the top right), $x_B$ (on the top left), and $t$ (on the bottom).  
The full red curves are calculations based on the impulse approximation from
\cite{Vadim_priv}. The black-dashed curves in the right plots are calculations 
from a convolution model based on the VGG model for the nucleons' GPDs 
\cite{Guidal_priv}. The blue long-dashed curve on the top-right plot is from
the off-shell calculations based on reference \cite{GonzalezHernandez:2012jv}.}
\label{fig:CFF_HA}
\end{figure}

%interpretations 




%conclusion





%Acknowledgments

We thank the staff of the Accelerator and Physics Divisions
at Jefferson Lab for making this experiment possible.

\begin{thebibliography}{99}

\bibitem{Burkardt:2000za} 
  M.~Burkardt,
  Phys.\ Rev.\ D {\bf 62}, 071503 (2000)
  Erratum: [Phys.\ Rev.\ D {\bf 66}, 119903 (2002)]

\bibitem{Diehl:2002he} 
  M.~Diehl,
  Eur.\ Phys.\ J.\ C {\bf 25}, 223 (2002)
  Erratum: [Eur.\ Phys.\ J.\ C {\bf 31}, 277 (2003)]
 
\bibitem{Belitsky:2002ep} 
  A.~V.~Belitsky and D.~Mueller,
  Nucl.\ Phys.\ A {\bf 711}, 118 (2002)

\bibitem{Burkardt:2005hp} 
  M.~Burkardt,
  Phys.\ Rev.\ D {\bf 72}, 094020 (2005)

\bibitem{Stepanyan:2001sm}
S.~Stepanyan {\it et al.} [CLAS Collaboration],
Phys.\ Rev.\ Lett. {\bf 87}, 182002 (2001).

\bibitem{Airapetian}
A. Airapetian {\it et al.} [HERMES Collaboration],
Phys.\ Rev.\ Lett. {\bf 87}, 182001 (2001);
JHEP {\bf 1207}, 032 (2012);
JHEP {\bf 1006}, 019 (2010);
JHEP {\bf 0806}, 066 (2008);
Phys.\ Lett.\ B {\bf 704}, 15 (2011);
Phys.\ Rev.\  D {\bf 75}, 011103 (2007);
JHEP {\bf 0911}, 083 (2009);
Phys.\ Rev.\ C {\bf 81}, 035202 (2010);
JHEP {\bf 1210}, 042 (2012).

\bibitem{Chekanov:2003ya}
S. Chekanov {\it et al.} [ZEUS Collaboration],
Phys.\ Lett.\  B {\bf 573}, 46 (2003).

\bibitem{Aktas:2005ty}
A. Aktas {\it et al.} [H1 Collaboration],
Eur.\ Phys.\ J.\ C {\bf 44}, 1 (2005).

\bibitem{Chen:2006na} 
S.~Chen {\it et al.} [CLAS Collaboration],
Phys.\ Rev.\ Lett.\ {\bf 97}, 072002 (2006).

\bibitem{Munoz Camacho:2006hx} 
C. Mu\~noz Camacho {\it et al.} [Jefferson Lab Hall A Collaboration],
Phys.\ Rev.\ Lett. {\bf 97}, 262002 (2006).

\bibitem{Girod:2007aa} 
F.X. Girod {\it et al.} [CLAS Collaboration],
Phys.\ Rev.\ Lett. {\bf 100}, 162002 (2008).

\bibitem{Gavalian:2009} 
G. Gavalian {\it et al.} [CLAS Collaboration],
Phys.\ Rev.\ C {\bf 80}, 035206 (2009).

\bibitem{Seder:2015} 
E. Seder {\it et al.} [CLAS Collaboration],
Phys.\ Rev.\ Lett. {\bf 114}, 032001 (2015).

\bibitem{Pisano:2015} 
S.~Pisano {\it et al.} [CLAS Collaboration],
Phys.\ Rev.\ D {\bf 91}, 052014 (2015).

\bibitem{Jo:2015ema} H.~S.~Jo {\it et al.} [CLAS Collaboration],
  Phys.\ Rev.\ Lett.\  {\bf 115}, no. 21, 212003 (2015)

\bibitem{Mazouz:2007aa} 
  M.~Mazouz {\it et al.} [Jefferson Lab Hall A Collaboration],
   Phys.\ Rev.\ Lett.\  {\bf 99}, 242501 (2007)

\bibitem{Freund_Collins}
A.~Freund and J.C.~Collins, Phys.\ Rev.\ D {\bf 59}, 074009 (1998)

\bibitem{Ji_Osborne}
   X.-D.~Ji and J.~Osborne, Phys.\ Rev.\ D {\bf 58}, 094018 (1998)

\bibitem{Belitsky}
A.~V.~Belitsky and A.~V.~Radyushkin, Phys.\ Rept.\ vol. 418 (2005)

\bibitem{Guidal:2013rya}
 M.~Guidal, H.~Moutarde and M.~Vanderhaeghen,
 Rept.\ Prog.\ Phys.\  {\bf 76}, 066202 (2013)

\bibitem{JSeely}
 J. Seely {\it et al.} Phys.\ Rev.\ Lett.\ {\bf 103}, 202301 (2009)

\bibitem{CLAS_ref}
   B.A. Mecking {\it et al.}, Nucl.\ Inst.\ and Meth.\ A 503, 513 (2003)

\bibitem{Belitsky:2001ns} A.~V.~Belitsky, D.~Mueller and A.~Kirchner,
     Nucl.\ Phys.\ B {\bf 629}, 323 (2002)

\bibitem{Kirchner:2003wt} 
   A.~Kirchner and D.~Mueller, Eur.\ Phys.\ J.\ C {\bf 32}, 347 (2003)

\bibitem{Belitsky:2008bz} A.~V.~Belitsky and D.~Mueller,
     Phys.\ Rev.\ D {\bf 79}, 014017 (2009)


\bibitem{Ellinghaus:2002zw} F.~Ellinghaus {\it et al.} [HERMES Collaboration],
  AIP Conf.\ Proc.\  {\bf 675}, 303 (2003)

\bibitem{simonetta_2}
   S.~Liuti and K.~Taneja, Phys.\ Rev.\ C {\bf 72}, 032201 (2005)

\bibitem{HERMES_BSA}
   A. Airapetian et al. (HERMES Collaboration), Phys.\ Rev.\ C {\bf 81}, 035202 
   (2010)

\bibitem{Vadim_priv}
   Private communications with V.~Guzey based on: 
   V.~Guzey, Phys.\ Rev.\ C {\bf 78}, 025211 (2008).

\bibitem{Guidal_priv}
   Private communications with M.~Guidal based on: M.~Guidal, M.~V.~Polyakov, 
   A.~V.~Radyushkin and M.~Vanderhaeghen, Phys.\ Rev.\ D {\bf 72}, 054013
   (2005).

\bibitem{GonzalezHernandez:2012jv} 
   Private communications with S.~Liuti based on: J.~O.~Gonzalez-Hernandez, 
   S.~Liuti, G.~R.~Goldstein and K.~Kathuria,
   Phys.\ Rev.\ C {\bf 88}, no. 6, 065206, (2013).


\end{thebibliography}

\end{document}
