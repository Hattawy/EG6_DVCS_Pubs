\appendix
\chead[]{\let\uppercase\relax\leftmark}

%\chapter{Appendix}
\chapter{$e ~^4He \rightarrow e ~^4He ~\gamma$ cross section} \label{app:Helium_cross_section}
The differential cross section for a longitudinally-polarized electron beam ($\lambda$) and an unpolarized $^4$He target can written as:
\small
\begin{equation}
\frac{d^{5}\sigma_{\lambda}}{dx_{A} dQ^{2} dt d\phi_{e} d\phi} = 
\frac{\alpha^{3}}{16 \pi^{2}} \frac{x_{A} \, y^{2}}{Q^{4} \sqrt{1 + \epsilon ^{2}}} 
\frac{
|\mathcal{T}_{BH}|^{2} + |{\mathcal{T}}_{DVCS}^{\lambda}|^{2} + {\mathcal{I}}_{BH*DVCS}^{\lambda}}{e^{6}}
\label{eq:sigdiff}
\end{equation}
\normalsize
where $y = \frac{p \cdot q}{p \cdot k}$, $\epsilon  =  \frac{2 x_{A} M_{A}}{Q}$ 
and $x_A  =  \frac{Q^2}{2 p \cdot q}$. The different amplitudes can be written 
as \cite{BM_2009}:
\small
\begin{equation}
|\mathcal{T}_{BH}|^{2} =  
\frac{e^{6} (1 + \epsilon^{2})^{-2}}{x^{2}_{A} y^{2} t \mathcal{P}_{1}(\phi) \mathcal{P}_{2}(\phi)} \left[ c_{0}^{BH} + c_{1}^{BH} \cos(\phi) + c_{2}^{BH} \cos(2\phi)\right] 
\label{TTBH}
\end{equation}

\begin{equation}
|\mathcal{T}_{DVCS}|^{2} =  \frac{e^{6}}{y^{2} Q^{2}} \left[ c_{0}^{DVCS} + \sum_{n=1}^{2} \Bigg( c_{n}^{DVCS} \cos(n \phi) + \lambda s_{n}^{DVCS} \sin(n \phi)\Bigg) \right] 
\label{TTDVCS}
\end{equation}

\begin{equation}
\mathcal{I}_{BH*DVCS} =  \frac{\pm e^{6}}{x_A y^{3} t \, \mathcal{P}_{1}(\phi) 
\mathcal{P}_{2}(\phi)} \left[ c_{0}^{I} + \sum_{n=0}^{3} \Bigg( c_{n}^{I} \cos(n \phi) + \lambda s_{n}^{I} \sin(n \phi) \Bigg) \right] 
\label{TTinter} 
\end{equation}
\\
\normalsize
where $\mathcal{P}_{1}(\phi)$ and $\mathcal{P}_{2}(\phi)$ are BH propagators and defined as:
\small
\begin{align}
&\mathcal{P}_{1}(\phi) = \frac{(k - q')^{2}}{Q^{2}} = - \frac{1}{y (1 + \epsilon^{2})} 
\big[ J + 2 K \cos(\phi) \big] \\
&\mathcal{P}_{2}(\phi) = \frac{(k - \Delta)^{2}}{Q^{2}} = 1 + \frac{t}{Q^{2}} + 
\frac{1}{y (1 + \epsilon^{2})} \big[ J + 2 K \cos(\phi) \big]
\end{align}
~~~~~~~~~~~with,
\begin{align}
& J = \bigg( 1 - y - \frac{y \epsilon^{2}}{2} \bigg) \bigg(1 + \frac{t}{Q^{2}} \bigg) - 
(1 - x_{A})(2 - y) \frac{t}{Q^{2}} \\
& K^{2} = - \delta t \, (1 - x_{A}) \bigg( 1 - y - \frac{y^{2} \epsilon^{2}}{4} \bigg) 
\bigg\{ \sqrt{1 + \epsilon^{2}} + \frac{4 x_{A} (1-x_{A}) + \epsilon^{2}}{4 (1 - x_{A})}
\delta t \bigg\} \\
& \delta t = \frac{t - t_{min}}{Q^{2}} = \frac{t}{Q^{2}} + \frac{2(1-x_{A}) \left(1- \sqrt{1 + 
\epsilon^{2}} \right) + \epsilon^{2}}{4 x_{A} (1- x_{A}) + \epsilon^{2}}
\end{align}
\normalsize
where $t_{min}$ represents the kinematic boundary of the process and defined as:
\small
\begin{equation}
t_{min} = Q^2 \frac{2(1-x_A)(1 - \sqrt{1+\epsilon^2}) + \epsilon^2}{4 x_A(1-x_A) + \epsilon^2}
\end{equation}
\normalsize
The Fourier coefficients, in equations \ref{TTBH}, \ref{TTDVCS} and \ref{TTinter}, of a spin-0 target are defined as:
\small
\begin{eqnarray}
c_0^{BH} = & \bigg[ & \left\{ {(2-y)}^2 + y^2{(1+\epsilon^2)}^2 \right\} 
\left\{ \frac{\epsilon^2 Q^2}{t} + 4 (1-x_A) + (4x_A+\epsilon^2) \frac{t}{Q^2} 
\right\} \nonumber \\
& \phantom{\bigg[} & + 2 \epsilon^2 \left\{ 4(1-y)(3+2\epsilon^2) + y^2(2-\epsilon^4) 
\right\} - 4 x_A^2{(2-y)}^2 (2+\epsilon^2) \frac{t}{Q^2} \nonumber \\
& \phantom{\bigg[} & + 8 K^2 \frac{\epsilon^2 Q^2}{t} \,\,\,\,\,\,\, \bigg] F_A^2(t)  \\
c_1^{BH} = & \phantom{\bigg[} & -8 (2-y) K \left\{ 2 x_A + \epsilon^2 - 
\frac{\epsilon^2 Q^2}{t} \right\} F_A^2(t)  \\
c_2^{BH} = & \phantom{\bigg[} & 8 K^2 \frac{\epsilon^2 Q^2}{t} F_A^2(t) 
\end{eqnarray} 
\normalsize
where $F_A(t)$ is the electromagnetic form factor of the $^4$He. At leading twist, the $|\mathcal{T}_{DVCS}|^{2}$ writes as a function of only one CFF according to
\small
\begin{equation}
%c_0^{DVCS} = 2 (2-2y+y^2) \, {\mathcal H}_A {\mathcal H}^{\star}_A 
   c_0^{DVCS}= 2 \frac{2-2y+y^2 + \frac{\epsilon^2}{2}y^2}{1 + \epsilon^2} \, 
   {\mathcal H}_A {\mathcal H}^{\star}_A 
   \label{eq:c0DVCS}
\end{equation}
\normalsize
and the interference amplitude coefficients are written as:
\small
\begin{equation}
s_{1}^{INT} = F_{A}(t) \Im m(\mathcal{H}_{A}) S_{++}(1),
\end{equation}
with
\begin{eqnarray}
   S_{++}(1) &=& \frac{-8K(2-y)y}{1+\epsilon^2} \left( 1 + 
\frac{1-xA+\frac{\sqrt{1+\epsilon^2}-1}{2}}{1+\epsilon^2} 
\frac{t-t_{min}}{Q^{2}} \right) \label{eq:s1I}
\end{eqnarray}

%c_0^{INT} & = & - 8 (2 - y) \frac{t}{Q^2} F_A \, \Re e \{{\mathcal H}_A\} 
%\label{eq:c0I} \\
% & \times & \left\{ (2-x_A)(1-y) - (1-x_A){(2-y)}^2 \left( 1 - 
% \frac{t_{min}}{Q^2} \right) \right\} \nonumber \\

%c_1^{INT} & = & 8 K (2y-y^2-2) F_A \, \Re e \{{\mathcal H}_A\} \label{eq:c1I} 

\begin{eqnarray}
\small
c_0^{INT} &=& F_A(t) \Re e(\mathcal{H}_{A}) C_{++}(0),
\end{eqnarray}
with \begin{eqnarray}  C_{++}(0) &=&
\frac{-4(2-y)(1+\sqrt{1+\epsilon^{2}})}{(1+\epsilon^{2})^2}  \bigg\{ 
   \frac{\widetilde{K}^2}{Q^2}  \frac{(2-y)^2}{\sqrt{1+\epsilon^{2}}} \, \\
   &+& \frac{t}{Q^2}  \left( 1 - y - \frac{\epsilon^2}{4} y^2 \right)  
(2-x_{A}) \left(  1 + \frac{2x_A(2-x_A + \frac{\sqrt{1+\epsilon^{2}}-1}{2} + 
\frac{\epsilon^{2}}{2x_A})\frac{t}{Q^2} + \epsilon^{2}}{(2-x_A) 
(1+\sqrt{1+\epsilon^{2}})}  \right)  \bigg\} \nonumber
 \label{eq:c0I} 
 \end{eqnarray}

\begin{eqnarray}
   c_1^{INT} &=&  F_A(t) \Re e(\mathcal{H}_{A}) C_{++}(1),
\end{eqnarray}
with  
   
   \begin{eqnarray}
   C_{++}(1) &=&
   \frac{-16K(1-y+\frac{\epsilon^{2}}{4}y^2)}{(1+\epsilon^{2})^{5/2}}\bigg\{\left(1+(1-x_A)\frac{\sqrt{1+\epsilon^{2}}-1}{2x_A} 
   + \frac{\epsilon^{2}}{4x_A}\right) 
\frac{x_At}{Q^2}-\frac{3\epsilon^{2}}{4.0} \bigg\} \nonumber \\&-& 4K \left( 
2-2y+y^2+\frac{\epsilon^{2}}{2}y^2\right)\frac{1+\sqrt{1+\epsilon^{2}}-\epsilon^{2}}{(1+e2)^{5/2}}\bigg\{1-(1-3x_A)\frac{t}{Q^2}\nonumber\\&\,\,\,\,&\,\,\,\,\,\,\,\,\,\,\,\,\,\,\,\,\,\,\,\,\,+\frac{1-\sqrt{1+\epsilon^{2}}+3\epsilon^{2}}{1+\sqrt{1+\epsilon^{2}}-\epsilon^{2}} 
\frac{x_A*t}{Q^2}\bigg\} \label{eq:c1I}
\end{eqnarray}





\chapter{ Angular dependence of the nucleon DVCS beam-spin asymmetry}

Note: All the formulas have been derived from A.~V.~Belitsky, D.~Mueller and 
A.~Kirchner,
  %``Theory of deeply virtual Compton scattering on the nucleon,''
Nucl.\ Phys.\ B {\bf 629}, 323 (2002).
[hep-ph/0112108].
%%CITATION = hep-ph/0112108%% 

The four-fold cross section for the process $e (k) h (P_1) \to e (k^\prime) h 
(P_2) \gamma (q_2)$,

\begin{eqnarray}
\label{WQ}
\frac{d\sigma}{d\Bx dy d|\Delta^2| d\phi }
=
\frac{\alpha^3  \Bx y } {16 \, \pi^2 \,  {\cal Q}^2 \sqrt{1 + \epsilon^2}}
   \frac{ |{\cal T}_{\rm BH}|^2 + |{\cal T}_{\rm DVCS}|^2 + {\cal I} }{e^6} \, 
   .
   \eqname{eq.22\&23 in ref}
\end{eqnarray}
This cross section depends on the Bjorken variable $\Bx$, the squared
momentum transfer $\Delta^2 = (P_2 - P_1)^2$, the lepton energy fraction $y
= P_1\cdot q_1/P_1\cdot k$, with $q_1 = k - k'$, $\phi$ is the angle between 
the lepton and hadron scattering planes, and $epsilon$ is defined as
$\epsilon \equiv 2 \Bx \frac{M}{{\cal Q}}$

The BH term $|{\cal T}_{\rm BH}|^2$, squared DVCS amplitude
$|{\cal T}_{\rm DVCS}|^2$, and  interference term ${\cal I}$ read
\begin{eqnarray}
\label{Par-BH}
&&|{\cal T}_{\rm BH}|^2
= \frac{e^6}
{\Bx^2 y^2 (1 + \epsilon^2)^2 \Delta^2\, {\cal P}_1 (\phi) {\cal P}_2 (\phi)}
\left\{
c^{\rm BH}_0
+  \sum_{n = 1}^2
c^{\rm BH}_n \, \cos{(n\phi)} + s^{\rm BH}_1 \, \sin{(\phi)}
\right\} \, ,
   \eqname{eq.25 in ref}
\\
\label{AmplitudesSquared}
&& |{\cal T}_{\rm DVCS}|^2
=
\frac{e^6}{y^2 {\cal Q}^2}\left\{
c^{\rm DVCS}_0
+ \sum_{n=1}^2
\left[
c^{\rm DVCS}_n \cos (n\phi) + s^{\rm DVCS}_n \sin (n \phi)
\right]
\right\} \, ,
   \eqname{eq.26 in ref}
\\
\label{InterferenceTerm}
&&{\cal I}
= \frac{\pm e^6}{\Bx y^3 \Delta^2 {\cal P}_1 (\phi) {\cal P}_2 (\phi)}
\left\{
c_0^{\cal I}
+ \sum_{n = 1}^3
\left[
c_n^{\cal I} \cos(n \phi) +  s_n^{\cal I} \sin(n \phi)
\right]
\right\} \, ,
   \eqname{eq.27 in ref}
\end{eqnarray}
where the $+$ ($-$) sign in the interference stands for the negatively
(positively) charged lepton. 


\section{Definition of some variables}

The lepton BH propagators,

\begin{eqnarray}
\label{Par-BH-Pro}
{\cal P}_1
   &=&- \frac{1}{y (1 + \epsilon^2)} \left\{ J + 2 K \cos(\phi) \right\}
\,\,
   \eqname{eq.32 in ref}
   \\
{\cal P}_2
   &=&
1 + \frac{\Delta^2}{\cQ^2} +
\frac{1}{y (1 + \epsilon^2)}\left\{J   + 2 K \cos(\phi)
\right\}
   \eqname{eq.32 in ref}
\, ,
\end{eqnarray}

where
\begin{eqnarray*}
J =
\left( 1 - y - \frac{y \epsilon^2}{2} \right)
\left( 1 + \frac{\Delta^2}{\cQ^2} \right)
-
(1 - x) (2 - y) \frac{\Delta^2}{\cQ^2} \, .
   \eqname{eq.32 in ref}
\end{eqnarray*}




\begin{equation}
K^2 = -\frac{\Delta^2}{{\cal Q}^2} (1 - \Bx)
\left( 1 - y - \frac{y^2\epsilon^2}{4} \right)
\left( 1 - \frac{\Delta^2_{\rm min}}{\Delta^2} \right)
\left\{
\sqrt{1 + \epsilon^2}
+ \frac{4\Bx (1 - \Bx) + \epsilon^2}{4(1 - \Bx)}
\frac{\Delta^2 - \Delta^2_{\rm min}}{{\cal Q}^2}
\right\} \, ,
   \eqname{eq.30 in ref}
\end{equation}

\begin{eqnarray}
\label{Def-tmin}
-\Delta_{\rm min}^2
=  {\cal Q}^2
\frac{2(1 - \Bx) \left(1 - \sqrt{1 + \epsilon^2}\right) + \epsilon^2}
{4\Bx (1 - \Bx) + \epsilon^2}
\approx \frac{M^2 \Bx^2}{1 - \Bx + \Bx M^2/{\cQ}^2}
\, ,
   \eqname{eq.31 in ref}
\end{eqnarray}

\begin{eqnarray}
y(x,\cQ^2)
=
2 \frac{\sqrt{1 + \epsilon^2} - 1}{\epsilon^2}
\approx
1 - \frac{M^2 \Bx^2}{\cQ^2} \, .
\end{eqnarray}


\section{Beam-spin asymmetry}
 At leading twist, the beam-spin asymmetry ($A_{LU}$) with two opposite 
 helicities of a  longitudinally-polarized electron beam (L) on an unpolarized 
 target (U) can be written as:        
 
 \begin{eqnarray}
    \label{eq:coh_BSA}
    \eqname{eq.176 in ref}
A_{LU}& =& \frac{d^{5}\sigma^{+} - d^{5}\sigma^{-} }
                {d^{5}\sigma^{+} + d^{5}\sigma^{-}}\\
    \eqname{}
      & =& \frac{x_A(1+\epsilon^2)^2}{y} \, s_1^{\cal I} \sin(\phi) \, \bigg/ 
      \, \bigg[ \, \sum_{n=0}^{n=2}c_n^{BH}\cos{(n\phi)} +  \\
    & & \frac{x_A^2 t {(1+\epsilon^2)}^2}{Q^2} {\cal P}_1(\phi) {\cal 
    P}_2(\phi) \, c_0^{DVCS} + \frac{x_A (1+\epsilon^2)^2}{y} \sum_{n=0}^{n=1} 
    c_n^{\cal I} \cos{(n\phi)} \bigg].  
 \nonumber 
 \end{eqnarray}

\subsection{Rewriting $A_{LU}$ in term of $\phi$ harmonics}
$A_{LU}$ can be expressed as following

 \begin{eqnarray}
A_{LU}& =& \frac{\alpha_0 \sin(\phi)}
                {\alpha_1 +  \alpha_2 \cos(\phi) + \alpha_3 \cos(2\phi) }   
                \label{eq:coh_BSA_2}
\end{eqnarray}
with the kinematical variables $\alpha_{0,1,2,3}$ as

 \begin{eqnarray}
\alpha_{0}& =& \frac{x_A(1+\epsilon^2)^2}{y} \, {\color{blue} s_1^{\cal I}},
\nonumber\\
\alpha_{1}& =& {\color{blue} c_0^{BH}} - \frac{x_A^2 t {(1+\epsilon^2)}^2}{Q^2}  \frac{1}{y (1 + \epsilon^2)}
\left( (J+K) (1 + \frac{\Delta^2}{\cQ^2}) + \frac{J^2 + 2K^2}{y (1 + \epsilon^2)} \right) {\color{blue} c_0^{DVCS}} + \frac{x_A (1+\epsilon^2)^2}{y} {\color{blue} c_0^{\cal I}},\\
\nonumber\\
\alpha_{2}& =& {\color{blue} c_1^{BH}} + \frac{x_A (1+\epsilon^2)^2}{y} {\color{blue} c_1^{\cal I}},
\nonumber\\
\alpha_{3}& =& {\color{blue} c_2^{BH}} - \frac{x_A^2 t {(1+\epsilon^2)}^2}{Q^2}  \frac{K}{y (1 + \epsilon^2)}
\left( 1 + \frac{\Delta^2}{\cQ^2} + \frac{2K}{y (1 + \epsilon^2)} \right) {\color{blue} c_0^{DVCS} }
\nonumber\\
%\alpha_{4}& =& - \frac{x_A^2 t {(1+\epsilon^2)}^2}{Q^2}  \frac{2K}{y (1 + \epsilon^2)}
%\left( 1 + \frac{\Delta^2}{\cQ^2} + \frac{2K}{y (1 + \epsilon^2)} \right) {\color{blue} c_0^{DVCS} }.
%\nonumber\\
 \label{eq:alphas_}
\end{eqnarray}


%%%%%%%%%%%%%%%%%%%%%%%%%%%%%%%%%%%%%%%%%%%%%%%%%%%%%%%%%%%%%%%%%%%%%
\subsection{Leading twist Fourier coefficients for unpolarized nucleon target}

\begin{itemize}
\item Bethe-Heitler coefficients
\label{BHcrosssection}
%%%%%%%%%%%%%%%%%%%%%%%%%%%%%%%%%%%%%%%%%%%%%%%%%%%%%%%%%%%%%%%%%%%%%

This part of the leptoproduction cross section is expressed solely in
terms of $F_1(\Delta^2)$ and $F_2(\Delta^2)$, the known Dirac and Pauli
form factors of the nucleon. For unpolarized target the Fourier
coefficients are:

\begin{eqnarray}
\label{Def-FC-BH-unp0}
    \eqname{eq.35 in ref}
c^{\rm BH}_{0,{\rm unp}}
\!\!\!&=&\!\!\!
8 K^2
\left\{
\left( 2 + 3 \epsilon^2 \right)
\frac{{\cal Q}^2}{\Delta^2}
\left( F_1^2 - \frac{\Delta^2}{4 M^2} F_2^2 \right)
+ 2 \Bx^2 \left( F_1 + F_2 \right)^2
\right\}
\\
&+&\!\!\! (2 - y)^2
\Bigg\{
\left( 2 + \epsilon^2 \right)
\Bigg[
\frac{4 \Bx^2 M^2}{\Delta^2}
\left( 1 + \frac{\Delta^2}{{\cal Q}^2} \right)^2
+ 4 (1 - \Bx)
\left( 1 + \Bx  \frac{\Delta^2}{{\cal Q}^2} \right)
\Bigg]
\left( F_1^2 - \frac{\Delta^2}{4 M^2} F_2^2 \right)
\nonumber\\
&+&\!\!\!
4 \Bx^2
\Bigg[
\Bx + \left(1 - \Bx + \frac{\epsilon^2}{2} \right)
\left(1 -  \frac{\Delta^2}{Q^2} \right)^2
- \Bx (1 - 2 \Bx) \frac{\Delta^4}{Q^4}
\Bigg]
\left( F_1 + F_2 \right)^2
\Bigg\}
\nonumber\\
&+&\!\!\! 8 \left( 1 + \epsilon^2 \right)
\left(1 - y - \frac{\epsilon^2 y^2}{4} \right)
\Bigg\{
2 \epsilon^2
\left( 1 - \frac{\Delta^2}{4 M^2} \right)
\left( F_1^2 - \frac{\Delta^2}{4 M^2} F_2^2 \right)
\nonumber\\
&&\qquad\qquad\qquad\qquad\qquad\qquad\qquad\qquad\qquad
- \Bx^2
\left( 1 -  \frac{\Delta^2}{Q^2} \right)^2
\left( F_1 + F_2 \right)^2
\Bigg\} \, ,
\nonumber\\
\label{Def-FC-BH-unp1}
    \eqname{eq.36 in ref}
c^{\rm BH}_{1,{\rm unp}}
\!\!\!&=&\!\!\!
8 K (2 - y)
\Bigg\{
\left(
\frac{4\Bx^2 M^2}{\Delta^2} - 2\Bx - \epsilon^2
\right)
\left(
F_1^2 - \frac{\Delta^2}{4 M^2} F_2^2
\right)
\\
&&\qquad\qquad\qquad\qquad\qquad\qquad\qquad\qquad\qquad
+ \, 2 \, \Bx^2
\left( 1 - (1 - 2\Bx) \frac{\Delta^2}{{\cal Q}^2} \right)
\left( F_1 + F_2 \right)^2
\Bigg\},
\nonumber\\
\label{Def-FC-BH-unp2}
    \eqname{eq.37 in ref}
c^{\rm BH}_{2,{\rm unp}} \!\!\!&=&\!\!\!
8 \Bx^2 K^2  \left\{ \frac{4 M^2}{\Delta^2}
\left(F_1^2 - \frac{\Delta^2}{4 M^2} F_2^2\right) + 2 \left(F_1 +
F_2\right)^2 \right\} \, .
\end{eqnarray}

%%%%%%%%%%%%%%%%%%%%%%%%%%%%%%%%%%%%%%%%%%%%%%%%%%%%%%%%%%%%%%%%%%%%%
\item DVCS coefficients
\label{SubSec-AziAngDep-DVCS}
%%%%%%%%%%%%%%%%%%%%%%%%%%%%%%%%%%%%%%%%%%%%%%%%%%%%%%%%%%%%%%%%%%%%%

$|{\cal T}_{\rm DVCS}|^2$ is bilinear in the CFFs, and its coefficients read
at leading twist and leading order for unpolarized target:

\begin{eqnarray}
\label{Def-C-DVCS-unp}
    \eqname{eq.43\&66 in ref}
c^{\rm DVCS}_{0,{\rm unp}}
\!\!\!&=&\!\!\!
   \frac{2 ( 2 - 2 y + y^2 )}{(2 - \Bx)^2}
\Bigg\{
4 (1 - \Bx)
\left(
{\cal H} {\cal H}^\ast
+
\widetilde{\cal H} \widetilde {\cal H}^\ast
\right)- \Bx^2
\bigg(
{\cal H} {\cal E}^\ast
+ {\cal E} {\cal H}^\ast
+ \widetilde{{\cal H}} \widetilde{{\cal E}}^\ast
+ \widetilde{{\cal E}} \widetilde{{\cal H}}^\ast
\bigg)
\nonumber\\
&&\qquad\qquad\;
-
\left( \Bx^2 + (2 - \Bx)^2 \frac{\Delta^2}{4M^2} \right)
{\cal E} {\cal E}^\ast
- \Bx^2 \frac{\Delta^2}{4M^2}
\widetilde{{\cal E}} \widetilde{{\cal E}}^\ast
\Bigg\},
\end{eqnarray}




%%%%%%%%%%%%%%%%%%%%%%%%%%%%%%%%%%%%%%%%%%%%%%%%%%%%%%%%%%%%%%%%%%%%%
\item Interference coefficients
\label{SubSec-AziAngDep-INT}
%%%%%%%%%%%%%%%%%%%%%%%%%%%%%%%%%%%%%%%%%%%%%%%%%%%%%%%%%%%%%%%%%%%%%

For the phenomenology of GPDs, ${\cal I}$ is the most interesting quantity
since it is linear in CFFs. This simplifies their disentanglement from
experimental measurements. At leading twist and leading order, the Fourier 
harmonics have the following form for unpolarized target:

\begin{eqnarray}
    \eqname{eq.53 in ref}
   c^{\cal I}_{0,\rm{unp}}
   \!\!\!&=&\!\!\!
   - 8  (2 - y)
   \Re{\rm e}
   \Bigg\{
      \frac{(2 - y)^2}{1-y} K^2 {\cal C}^{\cal I}_{\rm unp} \left({\cal 
      F}\right)
      +
      \frac{\Delta^2}{{\cal Q}^2}  (1-y)(2 - \Bx)
      \left(
      {\cal C}^{\cal I}_{\rm unp} + \Delta {\cal C}^{\cal I}_{\rm unp}
      \right)
      \left( {\cal F} \right)
      \Bigg\} \, ,
    \end{eqnarray}


\begin{eqnarray}
   \label{Res-IntTer-unp_c}
    \eqname{eq.54 in ref}
   c^{\cal I}_{1, \rm unp} &\!\!\!=\!\!\!&
   -8 K (2 - 2y + y^2)~\Re{\rm e}\left\{
       {\cal C}^{\cal I}_{\rm unp}\left({\cal F} \right) \right\}
    \end{eqnarray}


\begin{eqnarray}
   \label{Res-IntTer-unp_s}
    \eqname{eq.54 in ref}
    s^{\cal I}_{1, \rm unp}
   &\!\!\!=\!\!\!&
   8 K \lambda y (2-y)~\Im{\rm m} \left\{
      {\cal C}^{\cal I}_{\rm unp}\left({\cal F} \right) \right\} \end{eqnarray}








In the lowest twist approximation, ${\cal C}^{\cal I}_{\rm unp}\left({\cal F} 
\right)$ reads

\begin{eqnarray}
   \label{Def-C-Int-unp}
    \eqname{eq.69 in ref}
   {\cal C}^{\cal I}_{\rm unp}
   \!\!\!&=&\!\!\!
   F_1 {\cal H} + \frac{\Bx}{2 - \Bx}
   (F_1 + F_2) \widetilde {\cal H}
   -
   \frac{\Delta^2}{4M^2} F_2 {\cal E} \, ,
\end{eqnarray}

The addenda arising in power-suppressed contribution is defined as

\begin{eqnarray}
   \label{Def-C-IntAdd-unp}
    \eqname{eq.72 in ref}
   \Delta {\cal C}^{\cal I}_{{\rm unp}}
   \!\!\!&=&\!\!\!
   - \frac{\Bx}{2-\Bx}  (F_1 + F_2)
   \left\{
      \frac{\Bx}{2 - \Bx} ({\cal H} + {\cal E})
      + \widetilde {\cal H}
      \right\} \, ,
\end{eqnarray}


\end{itemize}


\subsection{Definition of the CFFs}
The Compton form factors (CFFs), $\mathcal{F = \{ H, E, \widetilde{H}, 
\widetilde{E}}\}$, are defined in terms of the GPDs, $F = \{H, E, 
\widetilde{H}, \widetilde{E}\}$ ,as:

\begin{align}
\begin{split}
    \eqname{}
\Re(&\mathcal{F}) = \mathcal{P} \int_{0}^{1}dx[F(x,\xi,t)-F(-x,\xi,t)] \, 
C^{\pm}(x,\xi), \end{split} \\
\Im(&\mathcal{F}) = - \pi [F(\xi,\xi,t)-F(-\xi,\xi,t)],
\end{align}


\subsection{Definition of $\phi$}
The angle between the leptonic and the hadronic planes, $\phi$, is defined as:

\begin{equation}
  \cos(\phi) = \frac{\vec{k} \times \vec{k'}}{| \vec{k} \times \vec{k'} |} 
   \cdot \frac{\vec{q} \times \vec{p'}}{| \vec{q} \times \vec{p'} |}
\end{equation}
and $\phi \in [0,\phi] iff (\vec{k} \times \vec{k'}).\vec{q'} \geq 0$.

Where $\vec{k}/\vec{k'}$, $\vec{q}/\vec{q'}$, $\vec{p}/\vec{p'}$ are the 
four-vectors of the incident/scatter electrons, virtual/real photons, 
initial/final protons. 

\normalsize



\chapter{The parametrizations for the RTPC} \label{app:RTPC_appendix}
\begin{itemize}

\item The parametrizations of the mean ($\mu$) and the width ($\sigma$) of 
   $\Delta z$ distributions shown in figure \ref{fig:rrtpc_delta_z}, with L and 
   R stand for the left and the right modules of the RTPC, and $z$ in mm:
\small
\begin{eqnarray}
\hspace{-0.3in} \mu_{\Delta z}^{L}(z) &=& 2.80051 -0.0624556*z +0.00035567*z^{2} +5.25789e-06*z^{3}\\
\hspace{-0.3in} \sigma_{\Delta z}^{L}(z) &=& 7.48614 -0.00776678*z -3.66892e-05*z^{2}\\
\hspace{-0.3in} \mu_{\Delta z}^{R}(z) &=& -3.85725 -0.061265*z +0.000324528*z^{2} +4.28801e-06*z^{3}\\
\hspace{-0.3in} \sigma_{\Delta z}^{R}(z) &=& 8.67335 -0.00975138*z +8.01378e-05*z^{2}
\end{eqnarray}  
\normalsize
\item  The parametrizations of the mean ($\mu$) and the width ($\sigma$) of $\Delta \phi$ distributions shown in figure \ref{fig:delta_phi_elastic} are:
\small
\begin{eqnarray}
\hspace{-0.2in} \mu_{\Delta \phi}^{L}(z) &=& 178.053 +0.0298072*z -0.000362634*z^{2} -2.32442e-07*z^{3}\\
\hspace{-0.2in} \sigma_{\Delta \phi}^{L}(z) &=& 2.00365 +0.0011081*z +4.1589e-05*z^{2} -2.95347e-07*z^{3}\\
\hspace{-0.2in} \mu_{\Delta \phi}^{R}(z) &=& 181.3 +0.00749361*z -0.000338728*z^{2} +6.37882e-06*z^{3}\\
\hspace{-0.2in} \sigma_{\Delta \phi}^{R}(z) &=& 2.0939 +9.59331e-05*z +2.16727e-05*z^{2} -5.69296e-08*z^{3}
\end{eqnarray} 
\normalsize

\item The parametrizations of the mean ($\mu$) and the width ($\sigma$) of $\Delta \theta$ distribution shown in figure \ref{fig:delta_theta_elastic} are: 
\small
\begin{eqnarray}
\hspace{-0.2in} \mu_{\Delta \theta}(z) &=& -1.02349 -0.0487393*z +0.000219641*z^{2} +3.84156e-06*z^{3}\\
\hspace{-0.2in} \sigma_{\Delta \theta}(z) &=& 3.57854 +0.00639663*z
\end{eqnarray}
\normalsize

~\newpage
\item The drift speed parametrization:
\begin{figure}[!h]
\centering
\includegraphics[scale=0.28]{fig_rtpc/p0_RunNumber.png}
\includegraphics[scale=0.28]{fig_rtpc/p1_RunNumber.png}
\includegraphics[scale=0.28]{fig_rtpc/p2_RunNumber.png}
\includegraphics[scale=0.28]{fig_rtpc/p3_RunNumber.png}
\caption{ The fit parameters: $p_{0}$, $p_{1}$, $p_{2}$, and $p_{3}$ for the individual runs. The red lines represent their piece-wise fits. }
\label{fig:Drift_speed_fit_par}
\end{figure} 


\begin{equation}
TDC_{max/2} (z)= p_{0} +  p_{1} * e^{p_{2}*(z-p_{3})^{2}}
\end{equation}

\begin {table}[!h]
\begin{center}
\begin{tabular}{|l|l|l|l|l|}
\hline
run range &  $p_{0}$ &  $p_{1}$ \\&  $p_{2}$ &  $p_{3}$ \\
\hline
61448 - 61481 & 
\small{ 1.14312e+03 - 1.75217e-02 *$r_{N}$} & 
\small{ 3.27339e+00 + 5.32577e-05 *$r_{N}$} \\&
\small{-7.55131e-02 + 1.22429e-06 *$r_{N}$} &  
\small{ 3.76627e+03 - 6.14148e-02 *$r_{N}$}\\
\hline
61483 - 61611 &  
\small{ 1.21405e+02 - 9.40705e-04 *$r_{N}$} & 
\small{ 3.89644e+00 + 6.33813e-05 *$r_{N}$}  \\& 
\small{ 4.09308e-03 - 6.97839e-08 *$r_{N}$} & 
\small{-9.04583e+02 + 1.45062e-02 *$r_{N}$}\\
\hline
61612 - 61646 & 
\small{ 1.39733e+03 - 2.16496e-02 *$r_{N}$} & 
\small{-3.07845e+02 + 5.10814e-03 *$r_{N}$}  \\&
\small{-8.23774e-02 + 1.33384e-06 *$r_{N}$} & 
\small{-9.05752e+02 + 1.44872e-02 *$r_{N}$}\\
\hline
61655 - 61779 &
\small{ 1.45093e+02 - 1.33438e-03 *$r_{N}$} &
\small{ 1.63746e+02 - 2.54273e-03 *$r_{N}$} \\&
\small{-5.10501e-04 + 5.64359e-09 *$r_{N}$} &
\small{ 4.26282e+02 - 7.12408e-03 *$r_{N}$}\\
\hline
61791 - 61930&
\small{ 2.18243e+02 - 2.51495e-03 *$r_{N}$} &
\small{ 4.92691e+01 - 6.90443e-04 *$r_{N}$} \\&
\small{-1.11909e-02 + 1.78407e-07 *$r_{N}$} &
\small{ 4.26297e+02 - 7.12383e-03 *$r_{N}$}\\
\hline
61931 - 61961)&
\small{ 2.18152e+02 - 2.51641e-03 *$r_{N}$} &
\small{ 4.92921e+01 - 6.90070e-04 *$r_{N}$} \\&
\small{-1.11766e-02 + 1.78639e-07 *$r_{N}$} &
\small{ 4.23668e+02 - 7.16628e-03 *$r_{N}$} \\
\hline
\end{tabular}
\caption[]{The parameters of $TDC_{max/2}$ used in EG6 experiment reconstruction codes.}
\label{Table:TDCmax}
\end{center}
\end{table}

~\newpage
\item Drift paths' parametrization:
\begin{equation}
  \Delta \phi (TDC, z)=  \sum\limits_{i=0}^{4} p_{i}(z)*TDC^{i}
\end{equation}
\begin{table}[!h]
\begin{center}
\begin{tabular}{|l|l|l|l|}
\hline
Parameter & ~~~~~constant & ~~~~~~~*z & ~~~~~~~*$z^{2}$\\
\hline
~~~~$p_{0}$   & 0.14222     &-6.52562e-05 & 4.06768e-06\\
\hline
~~~~$p_{1}$   &-0.00147368  & 5.64924e-06 &-7.31944e-07\\
\hline
~~~~$p_{2}$   & 0.000216222 & 6.25749e-09 & 1.8923e-08 \\
\hline
~~~~$p_{3}$   &-3.82450e-06 &-6.29825e-09 &-1.89627e-10\\
\hline
~~~~$p_{4}$   & 3.22973e-08 & 7.52017e-11 & 1.08564e-12\\
\hline
\end{tabular}
\caption{The drift paths extracted in the EG6 experiment.}
\label{table:drift_paths}
\end{center}
\end{table}

\end{itemize}

%\part{SIDIS generalized cross section}
%\begin{align}
%\lefteqn{\frac{d\sigma}{dx \, dy\, d\phi_{S} \,dz\, d\phi_h\, d P_{h\perp}^2}
%=}
%\nonumber \\ & \quad 
%\frac{\alpha^2}{x y Q^2}\,
%\frac{y^2}{2\,(1-\varepsilon)}\,  \biggl( 1+\frac{\gamma^2}{2x} \biggr)\, \Biggl\{ F_{UU ,T} +  %\varepsilon F_{UU ,L} + \sqrt{2\,\varepsilon (1+\varepsilon)}\,\cos\phi_h\,
%F_{UU}^{\cos\phi_h} \nonumber \\  & \quad \qquad + \varepsilon \cos(2\phi_h)\, 
%F_{UU}^{\cos 2\phi_h} + \lambda_e\, \sqrt{2\,\varepsilon (1-\varepsilon)}\,  \sin\phi_h\, 
%F_{LU}^{\sin\phi_h} \phantom{\Bigg[ \Bigg] }
%\nonumber \\  & \quad \qquad + S_\parallel\, \Bigg[ \sqrt{2\, \varepsilon (1+\varepsilon)}\,
%  \sin\phi_h\, F_{UL}^{\sin\phi_h} +  \varepsilon \sin(2\phi_h)\,  F_{UL}^{\sin 2\phi_h} \Bigg]
%\nonumber \\  & \quad \qquad
%+ S_\parallel \lambda_e\, \Bigg[ \, \sqrt{1-\varepsilon^2}\; 
%F_{LL} +\sqrt{2\,\varepsilon (1-\varepsilon)}\,
%  \cos\phi_h\,  F_{LL}^{\cos \phi_h} \Bigg] \nonumber \\  & \quad \qquad
%+ |{S}_\perp|\, \Bigg[ \sin(\phi_h-\phi_S)\, \Bigl(F_{UT ,T}^{\sin(\phi_h -\phi_S)}
%+ \varepsilon\, F_{UT ,L}^{\sin(\phi_h -\phi_S)}\Bigr) \nonumber \\  & \quad  \qquad \qquad
%+ \varepsilon\, \sin(\phi_h+\phi_S)\,  F_{UT}^{\sin(\phi_h +\phi_S)}
%+ \varepsilon\, \sin(3\phi_h-\phi_S)\, F_{UT}^{\sin(3\phi_h -\phi_S)}
%\phantom{\Bigg[ \Bigg] } \nonumber \\  & \quad \qquad \qquad + \sqrt{2\,\varepsilon (1+\varepsilon)}\, 
%  \sin\phi_S\,  F_{UT}^{\sin \phi_S }
%+ \sqrt{2\,\varepsilon (1+\varepsilon)}\,  \sin(2\phi_h-\phi_S)\,  
%F_{UT}^{\sin(2\phi_h -\phi_S)} \Bigg] \nonumber \\  & \quad \qquad 
%+ |{S}_\perp| \lambda_e\, \Bigg[  \sqrt{1-\varepsilon^2}\, \cos(\phi_h-\phi_S)\, 
%F_{LT}^{\cos(\phi_h -\phi_S)}
%+\sqrt{2\,\varepsilon (1-\varepsilon)}\,   \cos\phi_S\,  F_{LT}^{\cos \phi_S}
%\nonumber \\  & \quad \qquad \qquad +\sqrt{2\,\varepsilon (1-\varepsilon)}\,  \cos(2\phi_h-\phi_S)\,  %F_{LT}^{\cos(2\phi_h - \phi_S)} \Bigg] \Biggr\}
%\label{crossmaster}
%\end{align}

%\begin{align}
%\varepsilon &= \frac{1-y -\frac{1}{4} \gamma^2 y^2}{1-y
%  +\frac{1}{2} y^2 +\frac{1}{4} \gamma^2 y^2} ,
%\end{align}  


\chapter{The parametrization of the IC-photons energy corrections} \label{App:AppendixA}
\begin{itemize}
\item $\alpha$ parametrization
\vspace{-0.2in}
\begin{equation}
\alpha (x) = c_{0} + c_{2} \left[ e^{-c_{3}(x - c_{1})} - e^{-c_{4}(x - c_{1})} \right],
\end{equation}

\vspace{-0.2in}
\begin {table}[!h]
\begin{center}
\begin{tabular}{|l|l|l|l|l|l|l|l|}
\hline
$Fun_{N}$ & xmin  &   xmax    &  $c_{0}$     &   $c_{1}$    &  $c_{2}$      &   $c_{3}$  &   $c_{4}$\\
\hline
1  &  61510    & 61514   &  -0.00671929   &  61493     & -0.00868856    &   -0.114874  &  -0.117953\\
\hline
2   &  61519   &  61525  &    -0.00160398   &  61585.7  &  -2.1e-09    &   0.38362    &  0.38362\\
\hline
3   &  61531   &  61545  &    -0.00956513   &  61876.8  & -0.0295579   &    1.23958e-06 &  0.000704112\\
\hline
4   &  61546   &  61556  &    -0.000414459  &   61521.8  &-0.0179471   &    -0.0316302  &  -0.030899\\
\hline
5   &  61558   &  61580  &    -0.00731749  &   61532.4 &  -0.465254   &    0.0200131   &   0.0193213\\
\hline
6   &  61581   & 61590   &   0.0604759    & 61561.7  & -19.3314    & 0.0407449   & 0.0411026\\
\hline
7   &  61604   & 61608    &  -0.00320342  &   61521.1  &  -0.00074357   &    -0.018373  &    -0.0243743\\
\hline
8   &  61609   & 61622     & -0.00205987  &   61649.1  &   5.94004e-05   &    0.0760846  &    -2.64412\\
\hline
9   &  61623   & 61637    &  -0.00153458  &   61640.7  &   -1.02541e-10 &   0.965545   &   0.959411\\
\hline
10  &   61638   &  61646  &    -0.000735223  &   61763.1  & -0.00476067   & 0.0423013  &    0.0422956\\
\hline
11  &   61655  &  61675   &   -0.00123166   &  61561.1  &  -2.63733e-06  &  -0.159566  &    -0.159566\\
\hline
12  &  61678    & 61711   &   -0.00294886   &  61670.6   & -0.0079126   &    0.052612  &    0.0252883\\
\hline
13   &  61712   &  61713   &   -0.00102705   &  61711.9 &   -0.0034062 &  -3.1972   &   -3.19763\\
\hline
14   &  61714   &  61724   &   -0.00109184   &  61731.2&    -1.14221e-06  &     0.308061   &-4.24803\\
\hline
15   &  61725   &  61729   &   -0.00915958   &  61716.6 &     -0.0240355 &      0.137179  & 0.0524492\\
\hline
16   &  61731   &  61779   &   -0.00295947   &  61669.4 &  0.0130552    & 0.00768358   &   0.0117274\\
\hline
17   &  61791   &  61796   &   -0.00117275   &  61791.9  &  -0.00210188  &  5.63266   &   5.63003\\
\hline
18   &  61797   &  61826   &   0.00155106   &  61787.7 &  -0.980189  &     0.0514066 &     0.0518518\\
\hline
19   &  61829   &  61843   &   -0.0015276  &   61826 &  -0.00204473 &      2.72258   &   0.275824\\
\hline
20   &  61848  &   61874   &   0.0265578   &  61973.2 &  -0.0249309&    0.0010753   &   -0.186452\\
\hline
21   &  61876 &  61895    &  -0.00205642   &  61715.7  & -3.3277e-05 &   -0.065808   &   -0.0658091\\
\hline
22   &  61904  & 61915    &  -0.00274984   &  62178.9  &   -0.107207&     0.00976882  &    0.00977186\\
\hline
23   &  61925   &61930    &  0.0064528    & 61924.8   & -0.00849265  & 0.0278942  &   9.3268 \\
\hline
\end{tabular}
\caption{$\alpha$ parametrization shown in figure \ref{fig:ic_alpha_beta_fitting}.}
\label{Table:ic_alpha_corrections}
\end{center}
\end{table}

~\newpage
\item $\beta$ parametrization

\begin{equation}
\beta (x) = p_{0} + p_{2} \left[ e^{-p_{3}(x - p_{1})} - e^{-p_{4}(x - p_{1})} \right],
\end{equation}

\begin {table}[!h]
\begin{center}
\begin{tabular}{|l|l|l|l|l|l|l|l|}
\hline
$Fun_{N}$ & xmin  &   xmax    &  $p_{0}$     &   $p_{1}$    &  $p_{2}$      &   $p_{3}$  &   $p_{4}$\\
\hline
1   &  61510   &61514   &   0.139127   &  61508.7   &   -0.00705827      & -0.127   &   -0.163719\\
\hline
2   &  61519   &  61525  &    0.144057  &   61553.1  &    6.44528e-05   &    0.384066 &     0.384069\\
\hline
3   &  61531   &  61545   &   -0.238301    & 61487    &  -1.28623      & 0.0260483   &   0.011476\\
\hline
4   &  61546    & 61556    &  0.0904474   &  61545.1   &   -0.0367149 &      2.10481  &    -0.0186192\\
\hline
5   &  61558  &   61580     & 0.095243   &  61543.5  &    -0.197639  &     0.0280817  &    0.015222\\
\hline
6   &  61581   &  61590 & 0.0643454     &61557.1    &  -0.394422    &   0.0358816   &   0.0212999\\
\hline
7   &  61604    & 61608  &    0.138436 &    61605.4  &    -0.0142253    &   -0.0323464  &    -0.0103419\\
\hline
8  &   61609    & 61622   &   0.135047    & 61605.8   &   -0.00856825    &   0.157423    &  0.0474859\\
\hline
9   &  61623    & 61637    &  0.137445   &  61646.8    &  -1.46307e-05  &     0.273975  &    -0.0770047\\
\hline
10 &    61638    & 61646    &  0.141909 &    61651.9 &     -0.103873    &   0.00713496   &  0.000949948\\
\hline
11&   61655  & 61675  & 0.127712   &  61652.8   &  -0.00969769     &  0.220956  &    0.00215349\\
\hline
12 &  61678   &  61711 &     0.137774     &61697.7  &    -9.15967e-05    &   0.237965 &     -0.158587\\
\hline
13  & 61712   & 61713   &   0.124706     &61655.1    &  -3.72443e-05    &   -0.0443564 &     -0.100321\\
\hline
14    & 61714  &   61724 &     0.138057 &    61726.8  &    -0.000274029   &    0.234    &  -4.45937\\
\hline
15   &  61725  &   61729  &    0.108381    & 61721.3   &   -0.0289603    &   0.77633  &    -0.00285553\\
\hline
16  &   61731  &   61779   &   0.113727   &  61726   &   -0.0236114     &  0.243927    &  -0.00196355\\
\hline
17 &    61791  &   61791    &  0.136619  &   0.0   &   0.0   &    0.0  &    0.0\\
\hline
18&   61792   &  61796 & 0.138708   &  61914.5   &   1.07991e-09     &  0.181595    &  0.181612\\
\hline
19 &  61797  &   61825  &    0.137559   &  61839.6 &     -0.000312349  &     0.160791&      0.159434\\
\hline
20  &61826   & 61843     & 0.0359758   &  61810.5   &   -0.10032      & 0.215911     & -0.000529924\\
\hline
21 & 61848  &  61874      &0.116396   &  61891.7   &   -0.385562     &  -0.0563216    &  -0.0479573\\
\hline
22&   61876  &   61915  &  0.105816  &   61872.2    &  -0.0312028     &  0.384691    &  -0.00145868\\
\hline
23 & 61925   &  61930    &  0.137809   &  61927.5    &  -0.00071665  &     0.620419   &   0.222406\\

\hline
\end{tabular}
\caption{$\beta$ parametrization shown in figure \ref{fig:ic_alpha_beta_fitting}.}
\label{Table:ic_beta_corrections}
\end{center}
\end{table}
\end{itemize}

~\newpage

\chapter{Exclusive $\pi^{0}$ events selection}\label{app:Exclusive_pi0_selection}

The exclusive selection of the experimental $e^{4}He\pi^{0}$ and $ep\pi^{0}$ events require the detection of only one good electron, one good $\pi^{0}$ in the topology ICIC or ICEC, and one good $^{4}He$ track in the coherent channel or one good proton in the incoherent channel case. Furthermore, in order to ensure that this is a deep process we apply a set of initial requirements. The exclusivity of the reaction is ensured by a set of exclusivity cuts like for the DVCS channels. These requirements and exclusivity cuts are presented for the case of the coherent $e^{4}He\pi^{0}$ events are:

\paragraph{Initial criteria} ~\\
These requirements are made to ensure that the selected events occurre at the partonic level:
\begin{itemize}
\item High virtuality of the exchanged photon ($Q^{2}$>1 $GeV^{2}$).
\item High energy of the emitted $\pi^{0}$ ($E_{\pi^{0}}$ > 2 GeV).
\item The invariant mass of the virtual photon and the target proton is greater than 2 GeV$^{2}$/c$^{2}$ in order to avoid the baryons resonances region.
\item The transfer momentum squared ($-t$) between the initial target and the recoil one is greater than the minimum allowed one ($t_{min}$) defined by the kinematics of the incoming and the scattered electrons. The definition of $t_{min}$ for each channel can be found in the corresponding DVCS channel selection presented previously.
\end{itemize}

\paragraph{Exclusivity requirements}
~\\
~\\The exclusivity of the selected $e^{4}He\pi^{0}$ events is done with the following cuts:
\begin{itemize}
\item The coplanarity cut ($\Delta \phi$) between the recoil $^{4}He$ and the produced $\pi^{0}$.
\item The missing energy, mass and transverse momentum cuts in the configuration $e^{4}He\pi^{0}X$.
\item The missing mass cut in the configuration $e^{4}HeX$.
\item The missing mass cut in the configuration $e\pi^{0}X$.
\item The cone angle cut between $e^{4}HeX$ and the reconstructed $\pi^{0}$.
\end{itemize}
  The same procedure holds for the case of the incoherent $ep\pi^{0}$ events. In the following two subsections, the results of the two channels selection are presented.
~\newpage
\section{$e^{4}He\pi^{0}$ exclusivity cuts}
The events which pass the following exclusivity cuts are assumed to be good $e^{4}He\pi^{0}$ events.

\begin{figure}[h!]
\centering
\includegraphics[scale=0.4]{fig_dvcs/all_coh_pi0_exc_cuts.png}
\caption{The blue distributions represent all the  $e^{4}He\pi^{0}$ events before the exclusivity cuts. The shaded distributions show the events which passed all the exclusivity cuts except for the quantity plotted. The red lines are $3\sigma$ cuts. The mean and sigma values of each distribution are listed in table \ref{Table:cohpi0_exclusivity_cuts}.} 
\label{fig:cohpi0_exclusivty_cuts}
\end{figure}

\paragraph{Comparison with simulation}
~\\
~\\As for the selection of the experimental $e^{4}He\pi^{0}$ events, the simulated events have to pass an equivalent set of exclusivity cuts in addition to the $\pi^{0}$ electroproduction criteria, presented at the beginning of this section. In this section, we show the comparison between the experimental and the simulated selected  $e^{4}He\pi^{0}$ events as a function of the kinematic variables ($Q^{2}$, $x_{B}$, $-t$), figure \ref{fig:coh_pi0_comparison_with_simulation}, and as a function of the variables used for the exclusivity cuts, figure \ref{fig:coh_pi0_comparison_with_simulation_2}.\\

\begin{figure}[h!]
\hspace{-0.1in}\includegraphics[scale=0.37]{fig_dvcs/comp/Q2_Coh_pi0.png}
\hspace{-0.4in}\includegraphics[scale=0.37]{fig_dvcs/comp/xB_Coh_pi0.png}
\hspace*{-0.1in}\includegraphics[scale=0.37]{fig_dvcs/comp/t_Coh_pi0.png}
\hspace{-0.4in}\includegraphics[scale=0.37]{fig_dvcs/comp/phi_h_Coh_pi0.png}
\caption{Comparison between the simulated $e^{4}He\pi^{0}$ events (red lines) and the experimental events (blue shaded distributions) as a function of the kinematic variables: $Q^{2}$, $x_{B}$, $-t$ and $\phi_{h}$ respectively from top to right to right and from top to bottom.}
\label{fig:coh_pi0_comparison_with_simulation}
\end{figure}

\begin{figure}[h!]
\includegraphics[scale=0.35]{fig_dvcs/comp/Coh_pi0_delta_phi.png}
\includegraphics[scale=0.35]{fig_dvcs/comp/Coh_pi0_e4Hepi0_E_Mis.png}
\includegraphics[scale=0.35]{fig_dvcs/comp/Coh_pi0_e4Hepi0_M2_Mis.png}
\includegraphics[scale=0.35]{fig_dvcs/comp/Coh_pi0_e4Hepi0_PT_Mis.png}
\includegraphics[scale=0.35]{fig_dvcs/comp/Coh_pi0_e4He_M2_Mis.png}
\includegraphics[scale=0.35]{fig_dvcs/comp/Coh_pi0_epi0_M2_Mis.png}
\caption{Comparison between the simulated and experimental $e^{4}He\gamma$ events in terms of the exclusivity variables. The vertical black line indicates the theoretically expected value for each exclusive variable.} 
\label{fig:coh_pi0_comparison_with_simulation_2}
\end{figure}

Even with low experimental statistics, figures \ref{fig:coh_pi0_comparison_with_simulation} and \ref{fig:coh_pi0_comparison_with_simulation_2} show a good match between the simulation and the experimental   $e^{4}He\gamma$ events for the different kinematic variables, which is satisfying  for our background subtraction goal. 


~\newpage
~\newpage
\section{$ep\pi^{0}$ exclusivity cuts}
The $ep\pi^{0}$ events which pass the initial deepness criteria and the exclusivity cuts, marked by the red vertical lines in the figure below, are considered as clean events.

\begin{figure}[h!]
\centering
\includegraphics[scale=0.4]{fig_dvcs/all_incoh_pi0_exc_cuts.png}
\caption{The blue distributions represent all the $ep\pi^{0}$ events before any exclusive requirement. The shaded brown distributions show the events which passed all the exclusivity cuts except the quantity plotted. The vertical red lines represent $3\sigma$ cuts on the shaded distribution. The mean and sigma values of each distribution are listed in table \ref{Table:incohpi0_exclusivity_cuts}.} 
\label{fig:incohpi0_exclusivty_cuts}
\end{figure}



\paragraph{Comparison with simulation}
~\\
~\\
In this section, the experimental selected $ep\pi^{0}$ events are compared to the Monte Carlo simulated events. Figure \ref{fig:incoh_pi0_comparison_with_simulation} shows the comparison as a function of the kinematic variables. Figure \ref{fig:incoh_comparison_with_simulation_exclusive} shows the comparison in terms of the different exclusivity variables. One can see an agreement within some degrees of differences, which might come from the fact that our protons are bound ones and the physics of the nuclear process is not fully understood. 
\begin{figure}[h!]
\includegraphics[scale=0.35]{fig_dvcs/comp/Q2_InCoh_pi0.png}
\includegraphics[scale=0.35]{fig_dvcs/comp/xB_InCoh_pi0.png}
\includegraphics[scale=0.35]{fig_dvcs/comp/t_InCoh_pi0.png}
\includegraphics[scale=0.35]{fig_dvcs/comp/phi_h_InCoh_pi0.png}
\caption{Comparison between the Monte Carlo simulated $ep\pi^{0}$ events (red lines) and the experimental ones (blue shaded distributions) as a function of the kinematic variables: $Q^{2}$, $x_{B}$, $-t$, and $\phi$, respectively from top to right to right and from top to bottom.}
\label{fig:incoh_pi0_comparison_with_simulation}
\end{figure}

\begin{figure}[h!]
\includegraphics[scale=0.35]{fig_dvcs/comp/InCoh_pi0_delta_phi_InCoh.png}
\includegraphics[scale=0.35]{fig_dvcs/comp/InCoh_pi0_eppi0_E_Mis.png}
\includegraphics[scale=0.35]{fig_dvcs/comp/InCoh_pi0_eppi0_M2_Mis.png}
\includegraphics[scale=0.35]{fig_dvcs/comp/InCoh_pi0_eppi0_PT_Mis.png}
\includegraphics[scale=0.35]{fig_dvcs/comp/InCoh_pi0_epi0_M2_Mis_InCoh.png}
\includegraphics[scale=0.35]{fig_dvcs/comp/InCoh_pi0_Theta_pi0X_InCoh.png}
\caption{Comparison between the simulated and experimental $ep\pi^{0}$ DVCS events as a function of the variables used for the exclusivity cuts. The simulated distributions are normalized with respect to the experimental ones. The vertical black lines indicate the theoretically expected values.} 
\label{fig:incoh_comparison_with_simulation_exclusive}
\end{figure}


\chapter{Tables list of the exclusive distributions}\label{exclusivity_cuts}
\begin{itemize}

\item Exclusive $e^{4}He\gamma$ distributions
\begin {table}[!h]
\begin{center}
\begin{tabular}{|l|l|l|}
\hline
The quantity &  ~~~mean & ~~~~~$\sigma$ \\
\hline
$\Delta \phi$ &  1.86020e-01 & 4.64936e-01 \\
\hline
$E_{X}$ ($e^{4}He\gamma X$) &  1.48023e-02 & 2.51854e-01 \\ 
\hline
$M^{2}_{X}$ ($e^{4}He\gamma X$) &  -2.98662e-03 & 9.28645e-03 \\ 
\hline
$pt_{X}$ ($e^{4}He\gamma X$) & 4.14664e-02 & 4.24914e-02 \\ 
\hline
$M^{2}_{X}$ ($e^{4}HeX$) & -3.45128e-02 & 2.28247e-01 \\
\hline
$M^{2}_{X}$ ($e\gamma X$) & 1.39930e+01 & 1.61245 \\
\hline
$\theta$ ($\gamma$,$e^{4}HeX$) &  5.08070e-01 & 4.74883e-01\\
\hline
$px_{X}$ ($e^{4}He\gamma X$) & -2.32102e-03  & 4.52945e-02\\
\hline
$py_{X}$ ($e^{4}He\gamma X$) &  -8.97351e-04 & 3.89937e-02\\ 
\hline
\end{tabular}
\caption{ The mean and sigma values of the exclusive coherent quantities drawn in figure \ref{fig:coh_exclusivty_cuts}.}
\label{Table:coh_exclusivity_cuts}
\end{center}
\end{table}


\item Exclusive $e^{4}He\pi^{0}$ distributions
\begin {table}[!h]
\begin{center}
\begin{tabular}{|l|l|l|}
\hline
The quantity &  ~~~mean & ~~~~~$\sigma$  \\
\hline
$\Delta \phi$ & 1.41750e-01  & 3.84202e-01 \\
\hline
$E_{X}$ ($e^{4}He\pi^{0} X$)     &  7.80328e-03 & 1.85770e-01 \\
\hline
$M^{2}_{X}$ ($e^{4}He\pi^{0} X$) &  -2.31650e-03 & 8.65851e-03 \\
\hline
$pt_{X}$ ($e^{4}He\pi^{0} X$)    & 4.36619e-02  & 3.25254e-02 \\
\hline
$M^{2}_{X}$ ($e^{4}HeX$) &  -1.30346e-02 & 2.07791e-01 \\
\hline
$M^{2}_{X}$ ($e\pi^{0} X$)  & 1.39835e+01 & 1.33781\\
\hline
$\theta$ ($\pi^{0}$,$e^{4}HeX$) & 5.30001e-01  & 3.44745e-01\\
\hline
$px_{X}$ ($e^{4}He\pi^{0} X$)    &   -3.79596e-03  &  4.20732e-02\\
\hline
 $py_{X}$ ($e^{4}He\pi^{0} X$)    &   9.41010e-04  & 3.50393e-02 \\
\hline
\end{tabular}
\caption{ The mean and sigma values of the exclusive coherent quantities drawn in figure \ref{fig:cohpi0_exclusivty_cuts}.}
\label{Table:cohpi0_exclusivity_cuts}
\end{center}
\end{table}


~\newpage
\item Exclusive $ep\gamma$ distributions

\begin {table}[!h]
\begin{center}
\begin{tabular}{|l|l|l|}
\hline
The quantity &  ~~~mean & ~~~~~$\sigma$ \\
\hline
$\Delta \phi$ &  4.22584e-02 & 1.39413 \\
\hline
$E_{X}$ ($ep\gamma X$) &  6.27739e-02 & 1.34499e-01 \\
\hline
$M^{2}_{X}$ ($ep\gamma X$) &  -1.00889e-02 & 1.58503e-02 \\ 
\hline
$pt_{X}$ ($ep\gamma X$) & 8.03008e-02 & 4.28511e-02 \\  
\hline
$M^{2}_{X}$ ($epX$) &  2.40257e-01 & 3.66321e-01 \\
\hline
$M^{2}_{X}$ ($e\gamma X$) &  1.01266 & 2.03835e-01 \\
\hline
$\theta$ ($\gamma$, $epX$) &  1.06788 & 6.76469e-01 \\
\hline
$px_{X}$ ($ep\gamma X$) & 3.48024e-03  &  8.19527e-02\\
\hline
$py_{X}$ ($ep\gamma X$) & -1.50911e-03 & 8.16219e-02\\
\hline
\end{tabular}
\caption{ The mean and sigma values of the exclusive incoherent quantities drawn in figure \ref{fig:incoh_exclusivty_cuts}.}
\label{Table:incoh_exclusivity_cuts}
\end{center}
\end{table}


\item Exclusive $ep\pi^{0}$ distributions
\begin {table}[!h]
\begin{center}
\begin{tabular}{|l|l|l|}
\hline
The quantity &  ~~~mean & ~~~~~$\sigma$  \\
\hline
$\Delta \phi$ & -3.25864e-02  & 2.11499 \\ 
  
\hline
$E_{X}$ ($ep\pi^{0} X$) &  9.23934e-02 & 1.50977e-01 \\
  
\hline
$M^{2}_{X}$ ($ep\pi^{0} X$) &  -1.11900e-02 & 2.31963e-02 \\
  
\hline
$pt_{X}$ ($ep\pi^{0} X$) & 1.02247e-01 & 5.31387e-02 \\
  
\hline
$M^{2}_{X}$ ($epX$) & 2.27334e-01  & 2.98775e-01\\
\hline
$M^{2}_{X}$ ($e\pi^{0} X$) & 1.07125  & 2.79845e-01\\
\hline
$\theta$ ($\pi^{0}$, $epX$) &  1.42739 & 8.89072e-01 \\
\hline
$px_{X}$ ($ep\pi^{0} X$) &  2.19686e-03  & 9.01343e-02 \\
\hline
$py_{X}$ ($ep\pi^{0} X$) &  -1.30580e-03 & 9.05090e-02  \\ 
\hline
\end{tabular}
\caption{ The mean and sigma values of the exclusive $ep\pi^{0}$ quantities 
drawn in shaded brown in figure \ref{fig:incohpi0_exclusivty_cuts}.}
\label{Table:incohpi0_exclusivity_cuts}
\end{center}
\end{table}
\end{itemize}



\chapter{$A_{LU}$ tables}
\begin{table}[!h]
   \begin{center}
      \begin{tabular}{||l|l|l|l|l||}
         \hline
 $<Q^{2}>$ & $<x_{B}>$ & $<-t>$ & $<\phi>$ & $A_{LU}$ $\pm$ stat. $\pm$ syst.\\
         \hline
  1.143 & 0.136 & 0.096 & 24.3  &  0.1327 $\pm$ 0.1088 $\pm$ 0.0257 \\                                             
  1.143 & 0.136 & 0.096 & 61.1  &  0.3210 $\pm$ 0.0932 $\pm$ 0.0188 \\                                             
  1.143 & 0.136 & 0.096 & 98.9  &  0.3709 $\pm$ 0.1026 $\pm$ 0.0403 \\                                             
  1.143 & 0.136 & 0.096 & 140.9 &  0.2452 $\pm$ 0.1517 $\pm$ 0.0267 \\                                             
  1.143 & 0.136 & 0.096 & 178.4 &  0.0225 $\pm$ 0.1632 $\pm$ 0.0278 \\                                             
  1.143 & 0.136 & 0.096 & 219.1 & -0.0526 $\pm$ 0.1475 $\pm$ 0.0248 \\                                             
  1.143 & 0.136 & 0.096 & 262.9 & -0.2642 $\pm$ 0.1196 $\pm$ 0.0389 \\                                             
  1.143 & 0.136 & 0.096 & 301.9 & -0.1761 $\pm$ 0.0968 $\pm$ 0.0263 \\                                             
  1.143 & 0.136 & 0.096 & 337.7 & -0.2791 $\pm$ 0.1047 $\pm$ 0.0343 \\                                             
  \hline                                                                                                           
  1.423 & 0.172 & 0.099 & 21.1  &  0.1917 $\pm$ 0.0892 $\pm$ 0.0267 \\                                             
  1.423 & 0.172 & 0.099 & 56.6  &  0.2819 $\pm$ 0.0866 $\pm$ 0.0250 \\                                             
  1.423 & 0.172 & 0.099 & 96.9  &  0.4855 $\pm$ 0.1291 $\pm$ 0.0426 \\                                             
  1.423 & 0.172 & 0.099 & 139.9 &  0.0996 $\pm$ 0.1679 $\pm$ 0.0250 \\                                             
  1.423 & 0.172 & 0.099 & 180.3 &  0.1458 $\pm$ 0.1906 $\pm$ 0.0302 \\                                             
  1.423 & 0.172 & 0.099 & 218.7 & -0.1109 $\pm$ 0.1853 $\pm$ 0.0344 \\                                             
  1.423 & 0.172 & 0.099 & 263.1 & -0.3516 $\pm$ 0.1371 $\pm$ 0.0366 \\                                             
  1.423 & 0.172 & 0.099 & 302.2 & -0.4141 $\pm$ 0.0842 $\pm$ 0.0383 \\                                             
  1.423 & 0.172 & 0.099 & 338.2 & -0.2786 $\pm$ 0.0836 $\pm$ 0.0261 \\                                             
  \hline                                                                                                           
  1.902 & 0.224 & 0.107 & 21.4  &  0.1803 $\pm$ 0.0805 $\pm$ 0.0234 \\                                             
  1.902 & 0.224 & 0.107 & 57.2  &  0.3501 $\pm$ 0.0824 $\pm$ 0.0185 \\                                             
  1.902 & 0.224 & 0.107 & 96.2  &  0.2701 $\pm$ 0.1230 $\pm$ 0.0173 \\                                             
  1.902 & 0.224 & 0.107 & 139.5 &  0.3053 $\pm$ 0.2390 $\pm$ 0.0170 \\                                             
  1.902 & 0.224 & 0.107 & 178.2 &  0.1028 $\pm$ 0.2670 $\pm$ 0.0126 \\                                             
  1.902 & 0.224 & 0.107 & 221.4 & -0.2118 $\pm$ 0.2147 $\pm$ 0.0152 \\                                             
  1.902 & 0.224 & 0.107 & 263.3 & -0.3060 $\pm$ 0.1306 $\pm$ 0.0260 \\                                             
  1.902 & 0.224 & 0.107 & 303.3 & -0.1384 $\pm$ 0.0944 $\pm$ 0.0207 \\                                             
  1.902 & 0.224 & 0.107 & 338.5 & -0.1633 $\pm$ 0.0791 $\pm$ 0.0163 \\                                             
         \hline \hline
      \end{tabular}
      \caption{ The coherent $A_{LU}$ in $Q^2$ bins}
      \label{table:Coh_Q2_BSA}
   \end{center}
\end{table}                    

\begin{table}[!h]
   \begin{center}
      \begin{tabular}{||l|l|l|l|l||}
         \hline
 $<Q^{2}>$ & $<x_{B}>$ & $<-t>$ & $<\phi>$ & $A_{LU}$ $\pm$ stat. $\pm$ syst.\\
         \hline

  1.164 & 0.132 & 0.095 & 25.6  &  0.0174 $\pm$ 0.1440 $\pm$ 0.0221 \\                                            
  1.164 & 0.132 & 0.095 & 61.6  &  0.3478 $\pm$ 0.0867 $\pm$ 0.0198 \\                                            
  1.164 & 0.132 & 0.095 & 99.0  &  0.3808 $\pm$ 0.0948 $\pm$ 0.0408 \\                                            
  1.164 & 0.132 & 0.095 & 141.7 &  0.2943 $\pm$ 0.1379 $\pm$ 0.0333 \\                                            
  1.164 & 0.132 & 0.095 & 178.1 &  0.0433 $\pm$ 0.1519 $\pm$ 0.0293 \\                                            
  1.164 & 0.132 & 0.095 & 218.5 & -0.0352 $\pm$ 0.1316 $\pm$ 0.0243 \\                                            
  1.164 & 0.132 & 0.095 & 262.7 & -0.2771 $\pm$ 0.1050 $\pm$ 0.0372 \\                                            
  1.164 & 0.132 & 0.095 & 300.7 & -0.2138 $\pm$ 0.0836 $\pm$ 0.0259 \\                                            
  1.164 & 0.132 & 0.095 & 335.2 & -0.2339 $\pm$ 0.1221 $\pm$ 0.0319 \\                                            
   \hline                                                                                                         
  1.439 & 0.17 & 0.099 &  22.9  &  0.1579 $\pm$ 0.0853 $\pm$ 0.0230 \\                                            
  1.439 & 0.17 & 0.099 &  57.1  &  0.1733 $\pm$ 0.0880 $\pm$ 0.0195 \\                                            
  1.439 & 0.17 & 0.099 &  96.3  &  0.2259 $\pm$ 0.1331 $\pm$ 0.0295 \\                                            
  1.439 & 0.17 & 0.099 &  139.1 &  0.2454 $\pm$ 0.1758 $\pm$ 0.0187 \\                                            
  1.439 & 0.17 & 0.099 &  180.1 & -0.1021 $\pm$ 0.1918 $\pm$ 0.0197 \\                                            
  1.439 & 0.17 & 0.099 &  218.9 & -0.2876 $\pm$ 0.1909 $\pm$ 0.0270 \\                                            
  1.439 & 0.17 & 0.099 &  263.9 & -0.2943 $\pm$ 0.1360 $\pm$ 0.0292 \\                                            
  1.439 & 0.17 & 0.099 &  302.5 & -0.3979 $\pm$ 0.0921 $\pm$ 0.0331 \\                                            
  1.439 & 0.17 & 0.099 &  338.1 & -0.2692 $\pm$ 0.0834 $\pm$ 0.0254 \\                                            
   \hline                                                                                                         
  1.844 & 0.225 & 0.107 & 20.1  &  0.2626 $\pm$ 0.0755 $\pm$ 0.0248 \\                                            
  1.844 & 0.225 & 0.107 & 55.9  &  0.4281 $\pm$ 0.0887 $\pm$ 0.0224 \\                                            
  1.844 & 0.225 & 0.107 & 96.4  &  0.4925 $\pm$ 0.1390 $\pm$ 0.0271 \\                                            
  1.844 & 0.225 & 0.107 & 137.7 & -0.2736 $\pm$ 0.2804 $\pm$ 0.0166 \\                                            
  1.844 & 0.225 & 0.107 & 179.8 &  0.8470 $\pm$ 0.2497 $\pm$ 0.0204 \\                                            
  1.844 & 0.225 & 0.107 & 224.7 & -0.0513 $\pm$ 0.2811 $\pm$ 0.0270 \\                                            
  1.844 & 0.225 & 0.107 & 262.8 & -0.3416 $\pm$ 0.1694 $\pm$ 0.0353 \\                                            
  1.844 & 0.225 & 0.107 & 304.5 & -0.1359 $\pm$ 0.1025 $\pm$ 0.0263 \\                                            
  1.844 & 0.225 & 0.107 & 339.8 & -0.1658 $\pm$ 0.0770 $\pm$ 0.0175 \\                                            
         
         \hline 
         \hline
      \end{tabular}
      \caption{The coherent $A_{LU}$ in $x_B$ bins}
      \label{table:Coh_xB_BSA}
   \end{center}
\end{table}                        

\begin{table}[!h]
   \begin{center}
      \begin{tabular}{||l|l|l|l|l||}
         \hline
 $<Q^{2}>$ & $<x_{B}>$ & $<-t>$ & $<\phi>$ & $A_{LU}$ $\pm$ stat. $\pm$ syst.\\
  \hline
  1.36 & 0.160 & 0.080  & 22.8  &  0.2378 $\pm$ 0.0928 $\pm$ 0.0263 \\                                            
  1.36 & 0.160 & 0.080  & 57.6  &  0.3014 $\pm$ 0.0874 $\pm$ 0.0243 \\                                            
  1.36 & 0.160 & 0.080  & 98.1  &  0.4896 $\pm$ 0.1118 $\pm$ 0.0394 \\                                            
  1.36 & 0.160 & 0.080  & 139.3 &  0.1965 $\pm$ 0.1595 $\pm$ 0.0252 \\                                            
  1.36 & 0.160 & 0.080  & 178.8 &  0.0577 $\pm$ 0.1923 $\pm$ 0.0371 \\                                            
  1.36 & 0.160 & 0.080  & 222.6 & -0.1651 $\pm$ 0.1640 $\pm$ 0.0370 \\                                            
  1.36 & 0.160 & 0.080  & 265.6 & -0.3467 $\pm$ 0.1339 $\pm$ 0.0403 \\                                            
  1.36 & 0.160 & 0.080  & 300.0 & -0.2891 $\pm$ 0.0928 $\pm$ 0.0294 \\                                            
  1.36 & 0.160 & 0.080  & 339.3 & -0.1849 $\pm$ 0.0856 $\pm$ 0.0281 \\                                            
  \hline                                                                
  1.507 & 0.179 & 0.094 & 21.1  &  0.2481 $\pm$ 0.0931 $\pm$ 0.0267 \\                                            
  1.507 & 0.179 & 0.094 & 56.4  &  0.3386 $\pm$ 0.0828 $\pm$ 0.0284 \\                                            
  1.507 & 0.179 & 0.094 & 97.5  &  0.3467 $\pm$ 0.1156 $\pm$ 0.0307 \\                                            
  1.507 & 0.179 & 0.094 & 141.9 &  0.1464 $\pm$ 0.1887 $\pm$ 0.0215 \\                                            
  1.507 & 0.179 & 0.094 & 179.5 & -0.2814 $\pm$ 0.1864 $\pm$ 0.0360 \\                                            
  1.507 & 0.179 & 0.094 & 218.5 &  0.2104 $\pm$ 0.1998 $\pm$ 0.0147 \\                                            
  1.507 & 0.179 & 0.094 & 262.8 & -0.2399 $\pm$ 0.1280 $\pm$ 0.0283 \\                                            
  1.507 & 0.179 & 0.094 & 303.5 & -0.1989 $\pm$ 0.0955 $\pm$ 0.0286 \\                                            
  1.507 & 0.179 & 0.094 & 338.5 & -0.2098 $\pm$ 0.0875 $\pm$ 0.0201 \\                                            
  \hline                                                                                                                 
  1.610 & 0.193 & 0.127 & 22.3  &  0.0280 $\pm$ 0.0909 $\pm$ 0.0214 \\                                            
  1.610 & 0.193 & 0.127 & 60.7  &  0.3582 $\pm$ 0.0932 $\pm$ 0.0204 \\                                            
  1.610 & 0.193 & 0.127 & 96.8  &  0.2559 $\pm$ 0.1269 $\pm$ 0.0313 \\                                            
  1.610 & 0.193 & 0.127 & 139.9 &  0.2207 $\pm$ 0.1930 $\pm$ 0.0203 \\                                            
  1.610 & 0.193 & 0.127 & 178.5 &  0.5138 $\pm$ 0.1832 $\pm$ 0.0354 \\                                            
  1.610 & 0.193 & 0.127 & 217.5 & -0.2473 $\pm$ 0.1664 $\pm$ 0.0189 \\                                            
  1.610 & 0.193 & 0.127 & 260.9 & -0.2919 $\pm$ 0.1295 $\pm$ 0.0327 \\                                            
  1.610 & 0.193 & 0.127 & 303.4 & -0.2488 $\pm$ 0.0892 $\pm$ 0.0280 \\                                            
  1.610 & 0.193 & 0.127 & 336.8 & -0.2825 $\pm$ 0.0901 $\pm$ 0.0261 \\                                            
         \hline
         \hline
      \end{tabular}
      \caption{The coherent $A_{LU}$ in -t bins}
      \label{table:Coh_t_BSA}
   \end{center}
\end{table}

% incoherent channel

\begin{table}[!h]
   \begin{center}
      \begin{tabular}{||l|l|l|l|l||}
         \hline
 $<Q^{2}>$ & $<x_{B}>$ & $<-t>$ & $<\phi>$ & $A_{LU}$ $\pm$ stat. $\pm$ syst.\\
 \hline 
  1.404 & 0.166 & 0.376  &   21.0 &   0.0541 $\pm$  0.0442  $\pm$ 0.0120 \\
  1.404 & 0.166 & 0.376  &   62.9 &   0.0771 $\pm$  0.0461  $\pm$ 0.0156 \\
  1.404 & 0.166 & 0.376  &   95.4 &   0.1913 $\pm$  0.0472  $\pm$ 0.0263 \\
  1.404 & 0.166 & 0.376  &  140.1 &   0.1076 $\pm$  0.0561  $\pm$ 0.0158 \\
  1.404 & 0.166 & 0.376  &  182.3 &   0.0448 $\pm$  0.0693  $\pm$ 0.0234 \\
  1.404 & 0.166 & 0.376  &  220.4 &  -0.0294 $\pm$  0.0675  $\pm$ 0.0150 \\
  1.404 & 0.166 & 0.376  &  258.4 &  -0.1261 $\pm$  0.0459  $\pm$ 0.0226 \\
  1.404 & 0.166 & 0.376  &  303.1 &  -0.1238 $\pm$  0.0401  $\pm$ 0.0196 \\
  1.404 & 0.166 & 0.376  &  337.4 &  -0.0122 $\pm$  0.0537  $\pm$ 0.0142 \\
                                                                           
  \hline                                                              
  1.889 & 0.232 & 0.415  &   20.3 &   0.0143 $\pm$  0.0355  $\pm$ 0.0124 \\
  1.889 & 0.232 & 0.415  &   60.5 &   0.0668 $\pm$  0.0464  $\pm$ 0.0169 \\
  1.889 & 0.232 & 0.415  &   95.8 &   0.1297 $\pm$  0.0521  $\pm$ 0.0202 \\
  1.889 & 0.232 & 0.415  &  140.9 &   0.1651 $\pm$  0.0771  $\pm$ 0.0285 \\
  1.889 & 0.232 & 0.415  &  179.7 &   0.1589 $\pm$  0.0894  $\pm$ 0.0146 \\
  1.889 & 0.232 & 0.415  &  221.5 &  -0.2594 $\pm$  0.0808  $\pm$ 0.0430 \\
  1.889 & 0.232 & 0.415  &  260.3 &  -0.1280 $\pm$  0.0559  $\pm$ 0.0180 \\
  1.889 & 0.232 & 0.415  &  303.5 &  -0.1762 $\pm$  0.0386  $\pm$ 0.0198 \\
  1.889 & 0.232 & 0.415  &  338.2 &  -0.0303 $\pm$  0.0447  $\pm$ 0.0113 \\
  \hline                                                              
                                                                           
  2.339 & 0.288 & 0.497  &  20.8  &  0.0740  $\pm$  0.0330  $\pm$ 0.0135 \\
  2.339 & 0.288 & 0.497  &  58.2  &  0.1359  $\pm$  0.0458  $\pm$ 0.0101 \\
  2.339 & 0.288 & 0.497  &  94.7  &  0.1836  $\pm$  0.0567  $\pm$ 0.0176 \\
  2.339 & 0.288 & 0.497  & 140.9  & -0.0178  $\pm$  0.1008  $\pm$ 0.0161 \\
  2.339 & 0.288 & 0.497  & 181.8  &  0.0922  $\pm$  0.1329  $\pm$ 0.0236 \\
  2.339 & 0.288 & 0.497  & 225.0  & -0.0751  $\pm$  0.1066  $\pm$ 0.0212 \\
  2.339 & 0.288 & 0.497  & 261.0  & -0.2439  $\pm$  0.0596  $\pm$ 0.0241 \\
  2.339 & 0.288 & 0.497  & 303.0  & -0.1977  $\pm$  0.0384  $\pm$ 0.0153 \\
  2.339 & 0.288 & 0.497  & 338.7  & -0.0889  $\pm$  0.0397  $\pm$ 0.0145 \\
                                                                           
  \hline                                                              
  3.098 & 0.379 & 0.641  & 20.1   & 0.0959   $\pm$  0.0305  $\pm$ 0.0147 \\ 
  3.098 & 0.379 & 0.641  & 57.1   & 0.0817   $\pm$  0.0478  $\pm$ 0.0147 \\
  3.098 & 0.379 & 0.641  & 94.2   & 0.1632   $\pm$  0.0684  $\pm$ 0.0281 \\
  3.098 & 0.379 & 0.641  &137.7   & 0.0930   $\pm$  0.1412  $\pm$ 0.0129 \\
  3.098 & 0.379 & 0.641  &180.2   &-0.2272   $\pm$  0.1918  $\pm$ 0.0270 \\
  3.098 & 0.379 & 0.641  &226.0   &-0.0333   $\pm$  0.1599  $\pm$ 0.0273 \\
  3.098 & 0.379 & 0.641  &263.9   &-0.1630   $\pm$  0.0802  $\pm$ 0.0214 \\
  3.098 & 0.379 & 0.641  &303.2   &-0.1644   $\pm$  0.0405  $\pm$ 0.0221 \\
  3.098 & 0.379 & 0.641  &340.5   &-0.0914   $\pm$  0.0368  $\pm$ 0.0141 \\
 \hline
 \hline
      \end{tabular}
      \caption{The incoherent $A_{LU}$ in $Q^2$ bins}
      \label{table:InCoh_Q2_BSA}
   \end{center}
\end{table}


\begin{table}[!h]
   \begin{center}
      \begin{tabular}{||l|l|l|l|l||}
         \hline
 $<Q^{2}>$ & $<x_{B}>$ & $<-t>$ & $<\phi>$ & $A_{LU}$ $\pm$ stat. $\pm$ syst.\\
         \hline
 1.449 & 0.163 & 0.374 &  21.0  &  0.0935 $\pm$ 0.0463  $\pm$  0.0137 \\ 
  1.449 & 0.163 & 0.374 &  63.4  &  0.0690 $\pm$ 0.0442  $\pm$  0.0165 \\ 
  1.449 & 0.163 & 0.374 &  95.6  &  0.1469 $\pm$ 0.0439  $\pm$  0.0216 \\ 
  1.449 & 0.163 & 0.374 & 140.3  &  0.1017 $\pm$ 0.0519  $\pm$  0.0149 \\ 
  1.449 & 0.163 & 0.374 & 181.3  &  0.0707 $\pm$ 0.0619  $\pm$  0.0235 \\ 
  1.449 & 0.163 & 0.374 & 220.0  & -0.0446 $\pm$ 0.0616  $\pm$  0.0199 \\ 
  1.449 & 0.163 & 0.374 & 258.5  & -0.1145 $\pm$ 0.0433  $\pm$  0.0183 \\ 
  1.449 & 0.163 & 0.374 & 303.0  & -0.0977 $\pm$ 0.0394  $\pm$  0.0154 \\ 
  1.449 & 0.163 & 0.374 & 336.9  &  0.0329 $\pm$ 0.0565  $\pm$  0.0105 \\ 
  \hline                                                              
                                                                         
  1.929 & 0.225 & 0.381 &  21.5  &  0.0016 $\pm$ 0.0383  $\pm$  0.0131 \\
  1.929 & 0.225 & 0.381 &  60.3  &  0.0556 $\pm$ 0.0441  $\pm$  0.0161 \\
  1.929 & 0.225 & 0.381 &  96.0  &  0.1681 $\pm$ 0.0495  $\pm$  0.0183 \\
  1.929 & 0.225 & 0.381 & 141.2  &  0.1422 $\pm$ 0.0788  $\pm$  0.0246 \\
  1.929 & 0.225 & 0.381 & 181.5  &  0.1672 $\pm$ 0.0957  $\pm$  0.0165 \\
  1.929 & 0.225 & 0.381 & 223.1  & -0.2619 $\pm$ 0.0828  $\pm$  0.0343 \\
  1.929 & 0.225 & 0.381 & 260.2  & -0.1845 $\pm$ 0.0517  $\pm$  0.0227 \\
  1.929 & 0.225 & 0.381 & 303.0  & -0.1962 $\pm$ 0.0367  $\pm$  0.0206 \\
  1.929 & 0.225 & 0.381 & 336.8  & -0.0410 $\pm$ 0.0455  $\pm$  0.0100 \\
  \hline                                                              
                                                                         
  2.334 & 0.283 & 0.468 &  20.9  &  0.0687 $\pm$ 0.0327  $\pm$  0.0198 \\
  2.334 & 0.283 & 0.468 &  59.2  &  0.1242 $\pm$ 0.0456  $\pm$  0.0195 \\
  2.334 & 0.283 & 0.468 &  94.1  &  0.1648 $\pm$ 0.0583  $\pm$  0.0188 \\
  2.334 & 0.283 & 0.468 & 139.3  &  0.1118 $\pm$ 0.1102  $\pm$  0.0242 \\
  2.334 & 0.283 & 0.468 & 180.7  & -0.1943 $\pm$ 0.1545  $\pm$  0.0145 \\
  2.334 & 0.283 & 0.468 & 225.2  &  0.0081 $\pm$ 0.1109  $\pm$  0.0140 \\
  2.334 & 0.283 & 0.468 & 261.4  & -0.2416 $\pm$ 0.0660  $\pm$  0.0272 \\
  2.334 & 0.283 & 0.468 & 303.3  & -0.1712 $\pm$ 0.0375  $\pm$  0.0178 \\
  2.334 & 0.283 & 0.468 & 338.4  & -0.0649 $\pm$ 0.0406  $\pm$  0.0101 \\
  \hline                                                              
                                                                         
  2.975 & 0.389 & 0.688 &   19.6 &   0.0809$\pm$  0.0289 $\pm$  0.0143 \\
  2.975 & 0.389 & 0.688 &   55.1 &   0.1214$\pm$  0.0539 $\pm$  0.0148 \\
  2.975 & 0.389 & 0.688 &   93.4 &   0.2530$\pm$  0.0899 $\pm$  0.0313 \\
  2.975 & 0.389 & 0.688 &  135.1 &  -0.2304$\pm$  0.2247 $\pm$  0.0386 \\
  2.975 & 0.389 & 0.688 &  179.9 &  -0.0522$\pm$  0.4248 $\pm$  0.0364 \\
  2.975 & 0.389 & 0.688 &  230.5 &  -0.3766$\pm$  0.3336 $\pm$  0.0469 \\
  2.975 & 0.389 & 0.688 &  265.5 &  -0.0929$\pm$  0.1030 $\pm$  0.0202 \\
  2.975 & 0.389 & 0.688 &  303.4 &  -0.1977$\pm$  0.0453 $\pm$  0.0229 \\
  2.975 & 0.389 & 0.688 &  341.2 &  -0.1084$\pm$  0.0352 $\pm$  0.0156 \\
 \hline
 \hline
 \end{tabular}
 \caption{The incoherent $A_{LU}$ in $x_B$ bins}
 \label{table:InCoh_xB_BSA}
 \end{center}
\end{table}



\begin{table}[!h]
   \begin{center}
      \begin{tabular}{||l|l|l|l|l||}
         \hline
 $<Q^{2}>$ & $<x_{B}>$ & $<-t>$ & $<\phi>$ & $A_{LU}$ $\pm$ stat. $\pm$ syst.\\
  \hline
  1.842 & 0.215 & 0.135  &   21.6  &  0.1197  $\pm$  0.0369 $\pm$ 0.0135 \\ 
  1.842 & 0.215 & 0.135  &   60.6  &  0.0273  $\pm$  0.0416 $\pm$ 0.0190 \\ 
  1.842 & 0.215 & 0.135  &   96.3  &  0.2189  $\pm$  0.0409 $\pm$ 0.0179 \\ 
  1.842 & 0.215 & 0.135  &  141.9  &  0.1502  $\pm$  0.0542 $\pm$ 0.0263 \\ 
  1.842 & 0.215 & 0.135  &  179.8  &  0.0083  $\pm$  0.0670 $\pm$ 0.0150 \\ 
  1.842 & 0.215 & 0.135  &  220.7  & -0.1185  $\pm$  0.0653 $\pm$ 0.0182 \\ 
  1.842 & 0.215 & 0.135  &  259.6  & -0.2041  $\pm$  0.0430 $\pm$ 0.0195 \\ 
  1.842 & 0.215 & 0.135  &  303.3  & -0.1605  $\pm$  0.0369 $\pm$ 0.0137 \\ 
  1.842 & 0.215 & 0.135  &  337.5  & -0.0493  $\pm$  0.0469 $\pm$ 0.0100 \\ 
  \hline                                                              
                                                                           
  2.149 & 0.257 & 0.281  &  21.4   & 0.0356   $\pm$ 0.0400  $\pm$ 0.0131 \\
  2.149 & 0.257 & 0.281  &  60.9   & 0.0927   $\pm$ 0.0454  $\pm$ 0.0125 \\
  2.149 & 0.257 & 0.281  &  95.7   & 0.1488   $\pm$ 0.0510  $\pm$ 0.0338 \\
  2.149 & 0.257 & 0.281  & 139.3   & 0.1112   $\pm$ 0.0732  $\pm$ 0.0178 \\
  2.149 & 0.257 & 0.281  & 183.3   & 0.1052   $\pm$ 0.0933  $\pm$ 0.0298 \\
  2.149 & 0.257 & 0.281  & 222.8   &-0.0744   $\pm$ 0.0760  $\pm$ 0.0183 \\
  2.149 & 0.257 & 0.281  & 259.3   &-0.0912   $\pm$ 0.0549  $\pm$ 0.0253 \\
  2.149 & 0.257 & 0.281  & 302.2   &-0.2004   $\pm$ 0.0390  $\pm$ 0.0248 \\
  2.149 & 0.257 & 0.281  & 337.8   &-0.1041   $\pm$ 0.0468  $\pm$ 0.0171 \\
  \hline                                                              
                                                                           
  2.366 & 0.291 & 0.492  &   20.8  &  0.1113  $\pm$ 0.0371  $\pm$ 0.0154 \\
  2.366 & 0.291 & 0.492  &   59.9  &  0.1542  $\pm$ 0.0449  $\pm$ 0.0175 \\
  2.366 & 0.291 & 0.492  &   94.1  &  0.0957  $\pm$ 0.0598  $\pm$ 0.0147 \\
  2.366 & 0.291 & 0.492  &  137.7  & -0.1066  $\pm$ 0.1051  $\pm$ 0.0166 \\
  2.366 & 0.291 & 0.492  &  182.9  &  0.2483  $\pm$ 0.1187  $\pm$ 0.0392 \\
  2.366 & 0.291 & 0.492  &  223.6  & -0.0691  $\pm$ 0.1101  $\pm$ 0.0125 \\
  2.366 & 0.291 & 0.492  &  261.1  & -0.1898  $\pm$ 0.0616  $\pm$ 0.0234 \\
  2.366 & 0.291 & 0.492  &  302.7  & -0.1744  $\pm$ 0.0363  $\pm$ 0.0162 \\
  2.366 & 0.291 & 0.492  &  337.8  & -0.0666  $\pm$ 0.0466  $\pm$ 0.0178 \\
  \hline                                                              
                                                                           
  2.447 & 0.312 & 1.089  &  19.7   & 0.0317   $\pm$ 0.0304  $\pm$ 0.0178 \\
  2.447 & 0.312 & 1.089  &  56.7   & 0.0910   $\pm$ 0.0584  $\pm$ 0.0108 \\
  2.447 & 0.312 & 1.089  &  90.6   & 0.1629   $\pm$ 0.1117  $\pm$ 0.0110 \\
  2.447 & 0.312 & 1.089  & 131.2   & 0.0424   $\pm$ 0.3072  $\pm$ 0.0177 \\
  2.447 & 0.312 & 1.089  & 175.3   &-0.9355   $\pm$ 0.3971  $\pm$ 0.0181 \\
  2.447 & 0.312 & 1.089  & 230.5   &-1.1889   $\pm$ 0.5170  $\pm$ 0.0151 \\
  2.447 & 0.312 & 1.089  & 264.3   &-0.0721   $\pm$ 0.1091  $\pm$ 0.0171 \\
  2.447 & 0.312 & 1.089  & 304.8   &-0.1190   $\pm$ 0.0475  $\pm$ 0.0129 \\
  2.447 & 0.312 & 1.089  & 340.5   &-0.0442   $\pm$ 0.0364  $\pm$ 0.0164 \\

 \hline
 \hline
 \end{tabular}
 \caption{The incoherent $A_{LU}$ in -t bins}
 \label{table:InCoh_t_BSA}
 \end{center}
\end{table}

\chapter{Investigating the Fermi motion effects on the incoherent DVCS channel}

In our DVCS analysis, we considered the initial proton of the incoherent 
channel to be at rest, while this is not totally true because of the Fermi 
motion. Does this assumption is a good approximation? What other options can be 
used to minimize this effect on the calculated exclusive quantities?\\

Figure \ref{fig:handbag} presents the leading order handbag diagram of the 
incoherent DVCS channel off $^4$He, where the nucleus breaks and the DVCS 
occurs on a bound nucleon. Ideally, the transferred momentum squared ($t$) is 
defined using the initial and the final nucleons' 4-vectors as $(p-p')^{2}$.  
Then as a DVCS requirement, $t$ has to be greater than a certain value 
($t_{min}$), that is defined using the kinematics of the incoming and the 
scattered electrons. Due to the knowledge lack of the initial proton's 
momentum, this cut maybe inaccurate causes loosing some good DVCS events. In 
this work, we try to tackle this issue and check the stability of the 
reconstructed beam-spin asymmetry, as being our DVCS observable.\\ 

\begin{figure}[h!]
   \centering
\includegraphics[height=10.0cm]{fig-incoh/handbag_incoherent.pdf}
\caption{Representation of the leading order handbag diagram of the incoherent 
DVCS process off $^4$He}
\label{fig:handbag}
\end{figure}

Typically, the transferred momentum squared $t$ is equal to the transferred 
momentum squared ($t'$) defined between the virtual photon and the final-state 
real photon, as shown in Figure \ref{fig:handbag}.  While defining the momentum 
transferred as $t'$ is a good way to pass the Fermi motion effect on the 
initial bound proton, it is more smeared than than $t$ due to bigger energy 
resolutions of the photons. Figure \ref{fig:tprime} shows the distributions of 
the coherent and the incoherent relative difference between $t$ and $t'$. One 
can see that the coherent DVCS channel, the distributions is symmetric and 
centered at zero as expected due to our good knowledge on the initial $^4$He 
target, while it is not exactly the case for the incoherent case due to the 
Fermi motion.\\            

\begin{figure}[h!]
\includegraphics[height=5.0cm]{fig-incoh/T_tprime_Coh.png}
\includegraphics[height=5.0cm]{fig-incoh/T_tprime_InCoh.png}
\caption{On the left: the coherent relative difference between $t$ and $t'$.  
On the right: the incoherent relative difference.}
\label{fig:tprime}
\end{figure}

In the following, we check the effects of defining the transferred momentum 
squared as $t$ and $t'$ on the exclusive distributions and on the reconstructed 
beam-spin asymmetries. Figure \ref{fig:cuts_t} presents the exclusive 
distributions of the incoherent DVCS events where the transferred momentum
squared is defined as $t$, while it is define as $t'$ in figure 
\ref{fig:cuts_tprime}. To minimize the smearing effects of $t'$ on the 
exclusive distributions, 2$\sigma$ exclusive cuts were optimized compared to 
3$\sigma$ cuts which were applied in the $t$ case. In terms of collected 
statistics, the 2$\sigma$ cuts using $t'$ gave almost the same statistics using 
$t$. From comparing the exclusive distributions, especially the $ep\gamma$ 
missing energy and the missing mass squared, one sees that using $t'$ gives 
more Gaussian missing energy distributions and cleaner missing mass squared 
distribution.\\      

\begin{figure}[h!]
\includegraphics[height=15.0cm]{fig-incoh/old_all_incoh_exc_cuts.pdf}
\caption{The old set of exclusive cuts with $t$>$t_{min}$. The black
   distributions represent the incoherent DVCS events candidate. The shaded
   distributions represent the events which passed all the exclusivity cuts
   except the quantity plotted. The vertical red lines represent the applied
exclusivity, 3$\sigma$ cuts.}
\label{fig:cuts_t}
\end{figure}

\begin{figure}[h!]
   \includegraphics[height=15.0cm]{fig-incoh/all_incoh_exc_cuts_test.pdf}
   \caption{New set of exclusive cuts with $t'$>$t_{min}$. The black
      distributions represent the incoherent DVCS events candidate. The shaded
      distributions represent the events which passed all the exclusivity cuts
      except the quantity plotted. The vertical red lines represent the applied 
   exclusivity, 2$\sigma$ cuts.}
\label{fig:cuts_tprime}
\end{figure}

\begin{figure}[h!]
   \includegraphics[height=6.0cm]{fig-incoh/ALU_phi_p_Q2.pdf}
   \includegraphics[height=6.0cm]{fig-incoh/ALU_phi_p_x.pdf}
   \includegraphics[height=6.0cm]{fig-incoh/ALU_phi_p_t.pdf}
\caption{The reconstructed incoherent $A_{LU}$ as a function of the angle 
   $\phi$ in $Q^2$, $x_B$, and $-t$ bins, respectively from top to bottom using 
$t$>$t_{min}$ in black squares, and $t'$>$t_{min}$ in red squares.}
   \label{fig:bsa_incoh_bins}
\end{figure}

\begin{figure}[h!]
   \centering
   \includegraphics[height=6.0cm]{fig-incoh/ALU_90_p_vs_Q2_shortscenrario.pdf}\\
   \includegraphics[height=6.0cm]{fig-incoh/ALU_90_p_vs_x_shortscenrario.pdf}\\
   \includegraphics[height=6.0cm]{fig-incoh/ALU_90_p_vs_t_shortscenrario.pdf}
   \caption{The $Q^{2}$ (top), $x_{B}$ (middle), and $-t$-dependencies (bottom) 
      of the incoherent $A_{LU}$ at $\phi$~=~90$^{\circ}$, using the 
   $t$>$t_{min}$ cut and the corresponding exclusive cuts in black squares, and 
in red squares using the $t'$>$t_{min}$ and it's exclusive cuts.  The curves 
are theoretical calculations. On the bottom: the green circles are the HERMES 
$-A_{LU}$ (positron beam was used) inclusive measurements.}
\label{fig:incoh_Q2_xB_t_ALU}
\end{figure}



\section*{Conclusions}
In summary, we have studied the effects of Fermi motion on the measured 
incoherent beam-spin asymmetries by defining the transferred momentum squared 
through the nucleons in one case and through the photons in the other case. In 
terms of collected statistics, 2$\sigma$ exclusive cuts using the photons 
produced the same statistics as 3$\sigma$ cuts. The exclusive distributions 
using $t'$, through the photons, have shown more acceptable distributions in 
terms of being more symmetric in some cases and cleaner on others. The 
reconstructed asymmetries in the two cases of defining the transferred momentum 
squared are compatible and while the given error bars. As a suggestion, we 
would like to consider the $t'$ analysis as the final set for publications 
unless a major objections appear.

\chapter{Free proton beam-spin asymmetry interpolation} 
\label{app:free-proton-alu}
\section{Fitting the experimentally measured asymmetries} \label{fit_int} 
Following the discussion in section \ref{sec:Generalized_EMC}, comparing the 
coherent and the incoherent beam-spin asymmetries to free proton asymmetries 
reveal new information about the nuclear medium modifications. To construct 
such ratio, the published free proton asymmetries \cite{FX_BSA} have been used 
in this analysis, see figure \ref{fig:free-proton-alu}. The measured free 
proton $A_{LU}$ at $\phi = 90^{\circ}$ were fitted as a function of $-t$ for 
the different bins in $x_B$ and $Q^2$, figure \ref{fig:free-proton-alu}.  
Finally, the free proton asymmetries were calculated at the same measured 
incoherent and coherent $-t$ values from the t-dependence fit of the closest 
bin in $x_B$ and $Q^2$.  The statistical errors on the calculated asymmetries 
were calculated using the fit errors.\\ 


\begin{figure}[tpb]
\centering
\includegraphics[scale=0.85]{fig_Dec2016/F_ALU-proton-fits.pdf}
\caption{Black circles are the free proton beam-spin asymmetries at $\phi = 90 
   ^{\circ}$ as a function of $-t$ in different ($x_B, Q^2$) bins as measured 
   at CLAS-E1DVCS experiment. The solid red curves are fits to the free proton 
   asymmetries in the form $a_{0}*(\sqrt{t}-a_{1})*e^{a_{2}*\sqrt{t} - a_{3}}$.
   Blue points are the interpolated points corresponding for the incoherent 
points presented in this analysis. The red points stand for the extrapolated 
free proton asymmetries at the measured coherent kinematics.}
\label{fig:free-proton-alu}
\end{figure}


\section{Model-dependent extraction} \label{model_int}
The proton asymmetries have been adjusted using a dispersion calculation 
including a phenomenological GPD $H$ only. The imaginary part of the amplitude 
is parameterized as in \cite{GPD_cal_free_p}. The real part of the amplitude is 
then obtained from the dispersion relation and the D-term as subtraction 
constant, parameterized as in \cite{GPD_cal_free_p}. The beam spin asymmetries 
mostly constrain the parameters of the imaginary part, and the unpolarized 
cross-section mostly constrain the D-term parameters. Figure 
\ref{fig:FXfree-proton-alu} presents the published free proton asymmetries with 
the model calculations, in solid red lines, and the interpolated free proton 
asymmetries at our measured coherent and incoherent kinematic values.

\begin{figure}[tpb]
   \centering
\includegraphics[scale=0.55]{fig_updated/From_FX_Check_Fit_6GeV_asym.png}
\caption{The free proton beam-spin asymmetries at $\phi = 90^{circ}$ as a 
function of $-t$ in different ($x_B, Q^2$) bins as measured at CLAS-E1DVCS 
experiment. The solid red lines are the model calculations based on the 
kinematics of each bin. The blue points stand for the interpolated free proton 
asymmetries at the kinematics of our incoherent DVCS channels.}
\label{fig:FXfree-proton-alu}
\end{figure}

\section{Summary}
In this analysis we use the interpolated values from fitting the experimentally 
measured free proton asymmetries to construct the beam-spin asymmetry ratios, 
presnted in section \ref{fit_int}.  The differences between the beam-spin 
asymmetry ratios calculated using the free proton asymmetries fits, section 
\ref{fit_int}, and the model-dependent interpolated values, section 
\ref{model_int}, were added as a systematic errors on the constructed beam-spin 
asymmetry ratios.  Table \ref{table:ALU_ratios_} presents the coherent and the 
incoherent $A_{LU}$ ratios extracted using both methods and the associated 
systematic uncertainties.

\small
\begin{landscape}
\begin{table}[!h]
   \begin{center}
      \begin{tabular}{||l|l|l|l|l|l||}
         \hline
 ($Q^{2}$, $x_{B}$, $-t$) & EG6 $A_{LU}$ $\pm$ stat. &  Exp. fit $\pm$ stat. & Exp. $A_{LU}$ ratio   &  Model extr. & Model $A_{LU}$ ratio \\
  \hline
 (1.395, 0.166, 0.407) & 0.147 $\pm$ 0.023 & 0.206 $\pm$ 0.009 & 0.715 $\pm$ 0.117 & 0.205 & 0.717 \\
 (1.886, 0.233, 0.499) & 0.153 $\pm$ 0.028 & 0.219 $\pm$ 0.042 & 0.702 $\pm$ 0.188 & 0.224 & 0.683 \\
 (2.338, 0.290, 0.521) & 0.166 $\pm$ 0.032 & 0.219 $\pm$ 0.030 & 0.758 $\pm$ 0.182 & 0.217 & 0.764 \\
 (3.098, 0.379, 0.650) & 0.132 $\pm$ 0.039 & 0.214 $\pm$ 0.080 & 0.619 $\pm$ 0.296 & 0.201 & 0.656 \\
   \hline \hline  \hline                                                                     
 (1.425, 0.162, 0.397) & 0.115 $\pm$ 0.022 & 0.212 $\pm$ 0.009 & 0.545 $\pm$ 0.106 & 0.188 & 0.611 \\
 (1.922, 0.227, 0.418) & 0.182 $\pm$ 0.025 & 0.231 $\pm$ 0.040 & 0.790 $\pm$ 0.176 & 0.213 & 0.854 \\
 (2.354, 0.287, 0.492) & 0.192 $\pm$ 0.030 & 0.225 $\pm$ 0.030 & 0.851 $\pm$ 0.178 & 0.217 & 0.884 \\
 (2.987, 0.390, 0.714) & 0.149 $\pm$ 0.046 & 0.198 $\pm$ 0.083 & 0.754 $\pm$ 0.396 & 0.210 & 0.709 \\
   \hline  \hline  \hline                                                                    
 (1.823, 0.213, 0.145) & 0.142 $\pm$ 0.023 & 0.246 $\pm$ 0.037 & 0.576 $\pm$ 0.131 & 0.243 & 0.584 \\
 (2.127, 0.255, 0.282) & 0.155 $\pm$ 0.027 & 0.225 $\pm$ 0.018 & 0.688 $\pm$ 0.131 & 0.241 & 0.643 \\
 (2.308, 0.284, 0.490) & 0.183 $\pm$ 0.034 & 0.225 $\pm$ 0.030 & 0.811 $\pm$ 0.188 & 0.219 & 0.835 \\
 (2.406, 0.308, 1.107) & 0.170 $\pm$ 0.051 & 0.126 $\pm$ 0.034 & 1.343 $\pm$ 0.545 & 0.123 & 1.382 \\
   \hline  \hline  \hline                                                                    
 (1.143, 0.132, 0.096) & 0.303 $\pm$ 0.051 & 0.195 $\pm$ 0.102 & 1.552 $\pm$ 0.857 & 0.244 & 1.241 \\
 (1.423, 0.170, 0.099) & 0.364 $\pm$ 0.059 & 0.170 $\pm$ 0.020 & 2.137 $\pm$ 0.431 & 0.251 & 1.450 \\
 (1.902, 0.224, 0.107) & 0.294 $\pm$ 0.060 & 0.181 $\pm$ 0.082 & 1.624 $\pm$ 0.808 & 0.235 & 1.251 \\
     \hline \hline  \hline                                                                   
 (1.164, 0.132, 0.095) & 0.319 $\pm$ 0.044 & 0.195 $\pm$ 0.102 & 1.633 $\pm$ 0.888 & 0.223 & 1.430 \\
 (1.439, 0.170, 0.099) & 0.278 $\pm$ 0.078 & 0.166 $\pm$ 0.020 & 1.670 $\pm$ 0.518 & 0.243 & 1.144 \\
 (1.844, 0.225, 0.107) & 0.319 $\pm$ 0.161 & 0.178 $\pm$ 0.084 & 1.788 $\pm$ 1.235 & 0.251 & 1.270 \\
      \hline  \hline  \hline                                                                 
 (1.493, 0.177, 0.1)   & 0.349 $\pm$ 0.029 & 0.157 $\pm$ 0.034 & 2.214 $\pm$ 0.515 & 0.245 & 1.424 \\                                                               
 \hline
 \end{tabular}
 \caption{$A_{LU}$ ratios. The first three blocks stand for the incoherent bins 
 in $Q^{2}$, $x_{B}$, and $-t$, respectively. The last three blocks are for the 
 coherent bins in $Q^{2}$, $x_{B}$, and $-t$, respectively. The first column 
 contains the mean value of the kinematical variables for each bin. The second 
 column contains EG6 measured $A_{LU}$ with the statistical error bars. The 
 third column ("Exp. fit $\pm$ stat.") contains the interpolated free proton 
 $A_{LU}$ at EG6 kinematical values from fitting the free proton beam-spin 
 asymmetries. The forth column ("Exp. $A_{LU}$ ratio") stands for the 
 calculated $A_{LU}$ ratios using EG6 asymmetries and the fitted free proton 
 asymmetries. The fifth column ("Model extr.") stands for the model-depended 
 extracted free proton asymmetries. The sixth column ("Model $A_{LU}$ ratio") 
 presents the calculated $A_{LU}$ ratios using EG6 asymmetries and the 
 model-dependent extracted free proton asymmetries. }
 \label{table:ALU_ratios_}
 \end{center}
\end{table}
\end{landscape}




\chapter{Binning the incoherent DVCS data to match the free proton DVCS 
published data}
 \label{app_free_proton_4D}

Following the discussion in section \ref{sec:Generalized_EMC} and the 
interpolation technique for the free proton DVCS asymmetries in 
Appendix~\ref{app:free-proton-alu}, here we carry out a different technique to 
validate the calculated incoherent DVCS asymmetry ratios. We apply tuning cuts 
on the kinematical distributions to get similar mean values close to the 
published free proton kinematics published in \ref{fig:FXfree-proton-alu}. For 
instance, regarding the binning in $Q^2$, we start binning our incoherent DVCS 
data into four bins in $Q^2$ with mean values similar to the published free 
proton values, then we apply tuning cuts on the associated distributions in 
$x_B$ and $t$. Figure \ref{fig:incoh_Q2-bins-freep} shows the resulting 
kinematical distributions for the $Q^2$ bins, which are compared to the 
published free proton bins shown in table~\ref{Table:freeproton-x_Q2_bins}.  
Similarly, we apply tuning cuts to the binning in $x_B$ and $-t$ dependence.  
Figure \ref{fig:incoh_xB-bins-freep} shows the resulting kinematical 
distributions for the $x_B$ bins, which are compared to the published free 
proton bins shown in table~\ref{Table:freeproton-x_Q2_bins}. Figure 
\ref{fig:incoh_t-bins-freep} shows the resulting kinematical distributions for 
the $t$ bins, which are compared to the published free proton bins shown in 
table~\ref{Table:freeproton-t_bins}.

The beam-spin asymmetry dependence on $Q^2$, $x_B$, and $t$ bins is constructed 
as a function of the azimuthal angle $\phi$ and fitted for the incoherent DVCS 
data. The $A_{LU}$ ratios at $\phi$~=~90$^{\circ}$ are calculated using the two 
methods; interpolating the free proton data and tunning the incoherent DVCS 
data. The results are presented in 
Figure~\ref{fig:incoh_EMC_ratio_ALU_proton_two}. To conclude, the second method 
provides us with the adequate validation of the extracted asymmetry ratios 
using the interpolation technique. The difference between the two calculated 
asymmetry ratios will be considered as a systematic uncertainty to the first 
method.



\begin{figure}[h!]
   \centering
   \includegraphics[height=17.0cm]{4D-incoh-bin/4D-Q2_InCoh_bins.pdf}
   \caption{First row: the four bins in $Q^2$ of the incoherent DVCS events.  
   Second row: the associated distributions of $t$ for the bins in $Q^2$.  
   Third row: the associated distributions of $x_B$ for the bins in $Q^2$.}
\label{fig:incoh_Q2-bins-freep}
\end{figure}

\begin{figure}[h!]
   \centering
   \includegraphics[height=17.0cm]{4D-incoh-bin/4D-xB_InCoh_bins.pdf}
   \caption{ First row: the four bins in $x_B$ of the incoherent DVCS events.  
   Second row: the associated distributions of $t$ for the bins in $x_B$.    
   Third row: the associated distributions of $Q^2$ for the bins in $x_B$.}
\label{fig:incoh_xB-bins-freep}
\end{figure}


\begin{figure}[h!]
   \centering
   \includegraphics[height=17.0cm]{4D-incoh-bin/4D-t_InCoh_bins.pdf}
   \caption{ First row: the four bins in $t$ of the incoherent DVCS events.  
   Second row: the associated distributions of $Q^2$ for the bins in $t$.  
   Third row: the associated distributions of $x_B$ for the bins in $t$.}
\label{fig:incoh_t-bins-freep}
\end{figure}




\begin {table}[!h]
\begin{center}
\begin{tabular}{|l|l|l|l|l|l|l|l|}
\hline
<$x_{B}$> & <$Q^2$> & <-t> & $\alpha$ & $\Delta \alpha$ &  $\beta$ & $\Delta 
   \beta$ & $\chi^{2}$\\
\hline
0.177 & 1.56 & 0.14 & 0.190 & 0.009 & -0.03 & 0.12  & 2.3\\
\hline
0.249 & 1.95 & 0.28 & 0.275 & 0.014 & -0.06 & 0.09  & 1.3\\
\hline
0.345 & 2.64 & 0.30 & 0.269 & 0.019 & 0.29 & 0.15  & 0.3\\
\hline
0.471 & 3.34 & 0.78 & 0.262 & 0.052 & -0.23 & 0.25  & 0.9\\
\hline
\end{tabular}
   \caption{Free proton DVCS published asymmetries associated with the 
   incoherent $Q^{2}$ and $x_{B}$ bins.}
\label{Table:freeproton-x_Q2_bins}
\end{center}
\end{table}


\begin {table}[!h]
\begin{center}
\begin{tabular}{|l|l|l|l|l|l|l|l|}
\hline
<$x_{B}$> & <$Q^2$> & <-t> & $\alpha$ & $\Delta \alpha$ &  $\beta$ & $\Delta 
   \beta$ & $\chi^{2}$\\
\hline
0.238  & 1.89 & 0.14 & 0.243 & 0.013 & -0.02 & 0.09 & 1.0\\
\hline
0.249  & 1.95 & 0.28 & 0.275 & 0.014 & -0.06 & 0.09 & 1.3\\
\hline
0.249  & 1.95 & 0.49 & 0.200 & 0.026 & -0.36 & 0.15 & 1.3\\
\hline
0.248  & 1.94 & 0.77 & 0.172 & 0.035 & 0.01 &  0.31 & 0.3\\
\hline
\end{tabular}
   \caption{Free proton DVCS published asymmetries associated with the 
   incoherent $t$ bins.}
\label{Table:freeproton-t_bins}
\end{center}
\end{table}





\begin{figure}[tp]
\centering
\includegraphics[scale=0.52]{4D-incoh-bin/comp_ALU_ratioInc_Q2_shortscenrario.pdf}\\
\includegraphics[scale=0.52]{4D-incoh-bin/comp_ALU_ratioInc_x_shortscenrario.pdf}\\
\includegraphics[scale=0.52]{4D-incoh-bin/comp_ALU_ratioInc_t_shortscenrario.pdf}
\caption{ The $A_{LU}$ ratio between the bound and the free proton at 
   $\phi$~=~90$^{\circ}$, as a function of $Q^2$ (on the top), $x_B$ (on the 
   middle), and $t$ (on the bottom). The black squares are our results using 
   the interpolation method to get the free proton asymmetries at our kinematic 
   , the red squares are our results using the method of tuning our kinematics 
   to get kinematics similar to the published free proton from E1DVCS1 
   experiment. The gray band stands for the systematic errors calculated as the 
   difference between the two extraction methods. The green circles are the 
   HERMES inclusive measurement \cite{HERMES_BSA} results. The blue and red 
   curves are from an on-shell calculations from S.  Liuti and K.  Taneja 
   \cite{simonetta_2}. The solid and dashed black curves are from the off-shell 
   calculations \cite{EMC_vadim_2}.} \label{fig:incoh_EMC_ratio_ALU_proton_two}
\end{figure}





